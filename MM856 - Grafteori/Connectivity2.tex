\documentclass{Book}


\usepackage{amsthm}

\newtheorem{theorem}{Theorem}


\begin{document}

\chapter{Connectivity and Paths}

\section{Cuts and Connectivity}%
\label{sec:4.1}

We aim to have networks that remain stable (connected) even when some edges or vertices fail. For the rest of this chapter, we assume all graphs to have \textbf{no loops}, as they are irrelevant to connectivity.

A \textbf{seperating set} (or \textbf{vertex cut}) of a graph $G$ is a set $S \subseteq V(G)$ such that $G - S$ has more than one component. The \textbf{connectivity} of $G$, $\kappa(G)$ is the minimum size of a set $S \subseteq V(G)$, such that $G-S$ is disconnected or has only one vertex. We say that $G$ is $k$-connected if it has connectivity at least $k$.

The connectivity of the complete graph $K_{n}$ is $n-1$. I.e., you would have to remove all vertices save 1. However, if $G$ is not a complete graph, then $\kappa(G) \le n-2$. For complete bipartite graphs, then $\kappa(K_{m,n}) = \min\{n,m\}$.

\textbf{Harary graphs} are well-known graphs in Graph Theory. Let $k < n$, we define $H_{k,n}$ to be a Harary Graph. We distinguish between two cases:

(Note from lecture: Connects to the $\frac{k}{2}$ preceding and $\frac{k}{2}$ following vertices.)
\begin{enumerate}
	\item $k$ is even:
	      \begin{itemize}
		      \item We let $V(H_{k,n}) = \{v_{0}, v_{1}, \ldots, v_{n-1}\}$.
		      \item We add edges from $v_{i}$ to $v_{i-\frac{k}{2}}, v_{i-\frac{k}{2}k+1}, \ldots, v_{i-1}, v_{i+1}, v_{i+2}, \ldots, v_{i+\frac{k}{2}}$.
		      \item Here we treat all indices with modulo $n$.
		      \item In essence, this means we connect $v_{i}$ to the preceding $\frac{k}{2}$ vertices and the following $\frac{k}{2}$ vertices in the cyclic order.
	      \end{itemize}
	\item $k$ is odd and $n$ is even:
	      \begin{itemize}
		      \item Make each vertex adjacent to the nearest $\frac{k-1}{2}$ vertices in each direction and to the diametrically opposite vertex.
		      \item (note from lecture): See it as the unique diametrically oposite vertex. THis is guaranteed as $n$ is even.
	      \end{itemize}
	\item $k$ is odd and $n$ is odd:
	      \begin{itemize}
		      \item Start with $H_{k-1,n}$ and then add edges from $v_{i}$ to $v_{i+(n-1)/2}$ for all $0 \le i \le \frac{n-1}{2}$.
	      \end{itemize}
\end{enumerate}

\begin{theorem}[Harary, 1962]
	\(\kappa(H_{k,n}) = k \), and hence the minimum number of edges in a $k$-connected graph on $n$ vertices is $\lceil kn / 2 \rceil$
\end{theorem}

In a $k$-connected graph, every vertex has a degree of at least $k$ (since if $d(x) < k$, then $N(x)$ is a cut-set of size less than $k$, which is a contradiction). So if $G$ is $k$-connected, then $2|E(G)| = \sum_{u \in V(G)} d(u) \ge kn$. Thus $|E(G)|\ge \frac{kn}{2}$ and as $|E(G)|$ is an integer we always have the following:
\begin{equation*}
	|E(G)| \ge \left \lceil \frac{kn}{2}  \right \rceil
\end{equation*}

The Harary graphs show that this is best possible, as $|E(H_{k,n})| = \lceil \frac{kn}{2}  \rceil$.

A \textbf{disconnecting set} of edges is a set $F \subseteq E(G)$, such that $G - F$ has more than one component. A graph $G$ is $k$\textbf{-edge-connected} if every disconnecting set has at least $k$ edges. The \textbf{edge-connectivity} of $G$, $\kappa'(G)$, is the minimum size of a disconnecting set in $G$. We denote by $[S, T]$ all the edges between vertex sets $S$ and $T$. An \textbf{edge-cut} is an edge set $[S, \overline{S}]$, where $\emptyset \subset S \subset V(G)$ (i.e. $S$ is a non-empty proper subset of $V(G)$). Thus the edge-cut $[S, \overline{S}]$ connects all vertices in $S$ to all other vertices, and if cut, will disconnect the graph.

The difference between and edge-cut and a disconnecting set, is that a disconnecting set \textit{may} not be an edge cut, however, a \textit{minimal disconnecting set} is an edge-cut and vice-versa.

\begin{theorem}
	\(\kappa(G) \le \kappa'(G) \le \delta(G)\)
\end{theorem}

That is, the connectivity is smaller than or equal to the edge connectivity which is smaller than or equal to the minimum degree.

\begin{proof}
	We prove in two parts:

	\begin{enumerate}
		\item \(\kappa'(G) \le \delta(G)\)
		\item \(\kappa(G) \le \kappa'(G)\). Let $[S, \overline{S}]$ be a smallest edge-cut. We will prove the result in the following subcases.
		      \begin{itemize}
			      \item All edges between $S$ and $\overline{S}$ exist in $G$.
			      \item All edges between $S$ and $\overline{S}$ do not exist in $G$.
		      \end{itemize}
	\end{enumerate}
\end{proof}


\begin{theorem}
	If $G$ is a 3-regular graph, then \(\kappa(G) = \kappa'(G)\).
\end{theorem}

We omit the proof.

\begin{proposition}
	Let $S \subseteq V(G)$ be arbitrary. Then

	\begin{equation*}
		|[S, \overline{S}]| = \left( \sum_{v \in S} d(v) \right) - 2|E(G[S])|
	\end{equation*}
\end{proposition}

That is, the edge cut's size is equal to the degree of all vertices in $S$ summed, subtract twice the induced subgraph of $S$.

\begin{corollary}
	If $G$ is a simple graph and $|[S, \overline{S}]| < \delta(G)$, then $|S| > \delta(G)$.
\end{corollary}


We refer to a minimal non-empty edge cut as a \textbf{bond}.

\begin{proposition}
	If $G$ is a connected graph, then an edge cut $F$ is a bond if and only if $G-F$ has exactly two components.
\end{proposition}

We define a \textbf{block} of a graph to be a maximal connected subgraph of $G$ that has no cut-vertex. If $G$ itself has no cut-vertex and is connected, then we say that $G$ itself is a block.

\begin{proposition}
	Two blocks in a graph share at most one vertex.
\end{proposition}

\begin{proof}
	Given two blocks $B_{1}$ and $B_{2}$. Assume for the sake of contradiction, that they share at least two vertices.

	We will show that $B^{*} = B_{1} \cup B_{2}$ is a connected subgraph with no cut-vertex (which contradicts the assumption that $B_{1}$ and $B_{2}$ are maximal, from the definition).

	Let $\{x,y\} \subseteq V(B_{1}) \cap V(B_{2})$ and let $z \in V(B^{*})$ be arbitrary. WLOG, $z \ne y$ (otherwise $z \ne x$).

	All vertices in $B_{1}-z$ can reach $y$ and all vertices in $B_{2}-z$ can reach $y$. So $B^{*}$ is connected. As $z$ was arbitrary, $B^{*}$ cannot be disconnected by deleting one vertex.
\end{proof}

The blocks of the graph \textit{decompose} the graph (i.e., each block shares no edge with other blocks, but may share a vertex (if it wasn't blocks, the definition would instead be \textit{vertices})).

\section{$k$-connected graphs}%
\label{sec:4.2}

We say that two $(u,v)$-paths are \textbf{internally disjoint} if they have no common internal vertex. Recall that an internal vertex is every vertex in the path except $u$ and $v$.

\begin{theorem}
	\label{the:4.2.2}
	A graph $G$ with $n(G) \ge 3$ is $2$-connected if and only if for each $u,v \in V(G)$ there exist two internally disjoint $(u,v)$-paths in $G$.
\end{theorem}
Recall that $k$-connectivity means that at least $k$ vertices need to be deleted, in order for the graph to end up disconnected.


\begin{proof}
	We prove both directions: \\
	\noindent
	\textbf{``\(\forall u,v \in V(G)\): $G$ contains two internally disjoint $(u,v)$-paths \(\Rightarrow\) $G$ is $2$-connected.''}\\
	\noindent
	Choose an arbitrary $z \in V(G)$. We will show that $G - z$ is connected. Choose arbitrary $u, v \in V(G-z)$. From the assumption, we know that there exist two internally disjoint $(u,v)$-paths. \textit{At least} one of these do not contain $z$, thus this path exists in $G-z$. So there is a $(u,v)$-path in $G-z$. As $u$ and $v$ were arbitrary, $G-z$ is connected. As $z$ was arbitrary, $G$ is $2$-connected.\\
	\noindent
	\textbf{``$G$ is $2$-connected \(\Rightarrow\) \(\forall u,v \in V(G)\): $G$ contains two internally disjoint $(u,v)$-paths.''}\\
	\noindent
	We will prove that there exists two internally disjoint $(u,v)$-paths by induction on $d(u,v)$.\\
	\noindent
	\textit{Base Case, $d(u,v) = 1$}: Let $w \in N(u) \setminus \{v\}$. We know such a vertex exists, as it wouldn't be 2-connected otherwise. Then there is a $(w,v)$-path, $P$ in $G-u$. We now have two internally disjoint paths: $uv$ and $uP = uwv$.\\
	\noindent
	\textit{Inductive Step, $d(u,v) > 1$}: We assume the statement holds for all pairs of vertices at distance less than $d(u,v)$ (induction hypothesis). Let $up_{1}\ldots,p_{l}v$ be a shortest $(u,v)$-path in $G$. By induction there exists internally disjoint $(u,p_{l})$-paths, $P$ and $Q$ as $d(u,p_{l}) < d(u,v)$ which we have assumed to be true.

	We can assume that $v \notin V(Q) \cup V(P)$, as if it was it would contradict our assumption that we are talking about the shortest $(u,v)$-path.

	Since $G$ is connected, there exists a $(v,u)$-path: $R = vz_{1}z_{2}\ldots z_{r}u$ in $G-p_{l}$. \textbf{If} $R$ is internally disjoint from $P$ (analog. $Q$) we immediately get internally disjoint $(u,v)$-paths $R$ and $Pv$.

	Thus, if $R$ is \textit{not} internally disjoint from $Pv$ (analog. $Qv$) it must intersect at some point with $V(P) \cup V(Q) \setminus \{u,v\}$. Chose an $z_{i} \in V(P) \cup V(Q)$ such that $i$ is minimum. If $z_{i} \in V(P)$, then $vz_{1}z_{2}\ldots z_{i-1}P[z_{i},u]$ and $Qv$ are internally disjoint $(u,v)$-paths. On the other hand, if $z_{i} \in V(Q)$, then $vz_{1}\ldots z_{i-1}Q[z_{i},u]$ and $Pv$ are internally disjoint paths.

\end{proof}

\begin{lemma}[Expansion Lemma]
	\label{lemma:expansion}
	If $G$ is a $k$-connected graph and $G'$ is obtained from $G$ by adding a new vertex $y$ with at least $k$ neighbours in $G$, then $G'$ is connected.
\end{lemma}

\begin{theorem}
	Let $G$ be a graph with $n(G) \ge 3$. The following conditions are now equivalent:
	\begin{enumerate}
		\item $G$ is connected and has no cut-vertex
		\item \(\forall x, y \in V(G)\), there are internally disjoint $(x,y)$-paths.
		\item \(\forall x,y \in V(G)\), there is a cycle through $x$ and $y$.
		\item \(\delta(G) \ge 1\), and every pair of edges in $G$ lie on a common cycle.
	\end{enumerate}
\end{theorem}

\begin{proof}
	We prove:
	\begin{itemize}
		\item $1 \iff 2$: Was proved in Theorem~\ref{the:4.2.2}.
		\item $2 \iff 3$: Together they form a cycle.
		\item $4 \Rightarrow 3$
		\item $1 \text{ \& } 3 \Rightarrow 4$
	\end{itemize}
\end{proof}

A \textbf{subdivision} of an edge $uv$ in a graph $G$ is the operation of replacing $uv$ with a path $uwv$ through new vertex $v$. I.e., you add a new vertex which will be in the middle of the $uv$ path.

\begin{corollary}
	If $G$ is $2$-connected, then the graph $G'$ obtained by subdividing an edge $G$ is $2$-connected.
\end{corollary}

We define an \textbf{ear} of a graph to be the maximal path whose internal vertices have degree $2$ in $G$. Furthermore, an \textbf{ear decomposition} of $G$ is a decomposition $P_{0},P_{1},\ldots, P_{k}$ such that $P_{0}$ is a cycle and $P_{i}$ is an ear in $P_{0} \cup \cdots \cup P_{i}$ for all $i = 1, 2, \ldots, k$.

\begin{theorem}
	A graph is $2$-connected if and only if it has an ear-decomposition.
\end{theorem}

A \textbf{closed-ear} of a graph is a cycle, $C$, such that all vertices on $C$ have degree 2, except one. A \textbf{closed-ear decomposition} of $G$ is a decomposition $P_{0},P_{1}, \ldots, P_k$ such that $P_{0}$ is a cycle and $P_{i}$ is either a closed-ear or an open ear in $P_{0} \cup \cdots \cup P_{i}$ for all $i = 1, 2, \ldots, k$.

\begin{theorem}
	A graph is $2$-edge-connected if and only if it has a closed-ear decomposition.
\end{theorem}

We now turn our attention to connectivity in digraphs. We define connectivity analogously to graphs. A \textbf{seperating set} or \textbf{vertex cut} of a digraph $D$ is a set $S \subseteq V(D)$ such that $D-S$ is not strongly connected. A digraph is $k$-connected if every vertex cut has at least $k$ vertices. The minimum size of a vertex cut is the connectivity \(\kappa(D)\).

For $S, T \subseteq V(G)$ let $[S, T]$ denote all arcs from $S$ to $T$ (Not the other way around, nor is it interchangeable!). An \textbf{arc cut} is the set $[S, \overline{S}]$ for some $\emptyset \ne S \subset V(D)$. A digraph is $k$-\textbf{arc-connected} if every arc cut has at least $k$ edges. The minimum size of the arc cut is the \textbf{arc-connectivity} \(\kappa'(D)\).

\begin{proposition}
	Adding a directed ear to a strong digraph $D$ produces another (larger) strong digraph $D'$.
\end{proposition}

While we will not show a direct proof, there are two main ways of proving this:
\begin{enumerate}
	\item[Way 1: ] Show that every set $\emptyset \ne S \subset V(D')$ has an arc out of it.
	\item[Way 2: ] Show that for every $x,y \in V(D')$ there is a $(x,y)$-path in $D'$.
\end{enumerate}

An \textbf{orientation} of a graph is a digraph obtained by directing each edge.

\begin{theorem}
	A graph $G$ has a strong orientation if and only if $G$ is $2$-edge-connected.
\end{theorem}

\textbf{``Strong Orientation \(\Rightarrow\) 2-edge-connected''}: We can either prove this directly or by contradiction.

\textbf{``Strong Orientation \(\Leftarrow\) 2-edge-connected''}: We prove this by considering an ear decomposition of $G$.

Given $x, y \in V(G)$, a set $S \subseteq V(G) \setminus \{x,y\}$ is a $(x,y)$-\textbf{seperator} (or $(x,y)$-cut) if $G-S$ has no $(x,y)$-path. We let $\kappa(x,y)$ be the minimum size of a $(x,y)$-cut. Let \(\lambda(x,y)\) be the maximum number of pairwise internally disjoint $(x,y)$-paths. For $X, Y \subseteq V(G)$ we define an $(X,Y)$-path to be a path starting in $X$ and ending in $Y$, and having no internal vertices in $X \cup Y$. We always have \(\kappa(x,y) \ge \lambda(x,y)\) as there cannot be fewer cuts than paths.

\begin{theorem}[Menger, 1927]
	\label{the:menger}
	Let $x,y \in V(G)$ be arbitrary such that $xy \notin E(G)$. Then \(\kappa(x,y) = \lambda(x,y)\)
\end{theorem}

That is, the minimum size of a $(x,y)$-cut is equal to the maximum amount of pairwise internally disjoint $(x,y)$-paths.

\begin{proof}
	We have shown that \(\kappa(x,y) \ge \lambda(x,y)\), thus to complete the proof we shall show that $\kappa(x,y) \le \lambda(x,y)$ thus making \(\kappa(x,y) = \lambda(x,y)\). Let $S$ be a minimum $(x,y)$-cut. Then there is no $(x,y)$-path in $G-S$ and $|S| = \kappa(x,y)$. To complete the proof, we show that there are $|S|$ pairwise internally disjoint $(x,y)$-paths. We prove by induction on $n = n(G)$.\\
	\noindent
	\textit{Base Case $n = 2$}: In this case $V(G) = \{x,y\}$. As $xy \notin E(G)$, we have $\kappa(x,y) = 0 = \lambda(x,y)$.\\
	\noindent
	\textit{Inductive Step $n > 2$}: We assume the theorem holds for graphs of order $< n$. We consider two cases:\\
	1. There exists  a minimum $(x,y)$-cut different than $N(x)$ and $N(y)$. Let $S$ be such a cut. We define $V_{1}$ to contain all vertices on $(x,S)$-paths in $G$, and $V_{2}$ to contain all vertices on $(S,y)$-paths in $G$. We will then show that $S = V_{1} \cap V_{2}$. As $S$ is minimal $(x,y)$-cut, $S \subseteq V_{1} \cap V_{2}$. If $V_{1} \cap V_{2}$, then $v \in S$, as $S$ is a $(x,y)$-cut. So $S = V_{1} \cap V_{2}$, as desired.
\end{proof}


The \textbf{line graph} of a graph $G$, written $L(G)$, is defined as follows:
\begin{equation*}
	V(L(G)) = E(G)
\end{equation*}
\begin{equation*}
	E(L(G)) = \{ef \mid \text{ e and f  have a common end-point}\}
\end{equation*}

We can also define \textit{line digraphs} for directed graphs as follows:

\begin{equation*}
	A(L(D)) = \{(e,f) \mid e = uv \text{ and } f = vw \text{ where } e, f \in A(D)\}
\end{equation*}

\begin{lemma}
	\label{lemma:deletionofedge}
	Deletion of an edge reduces the connectivity by at most 1.
\end{lemma}

\begin{theorem}
	\label{the:coinnectivity}
	The connectivity of $G$ equals the maximum $k$ such that \(\lambda(x,y) \ge k\) for all $x,y \in V(G)$. The edge connectivity of $G$ equals the maximum $k$ such that $\lambda'(x,y) \ge k$ for all $x,y \in V(G)$.
\end{theorem}

Both statements (\ref{lemma:deletionofedge} and \ref{the:coinnectivity}) hold for both graphs and digraphs.

Let's now look at the applications of Menger's Theorem.

Given an arbitrary vertex $x$, and a set $U \subseteq V(G) \setminus \{x\}$, an $(x,U)$\textbf{-fan} is a set of paths from $x$ to $U$ such that any two of them share only the vertex $x$.

\begin{theorem}
	A graph $G$ is $k$-connected if and only if it has at least $k+1$ vertices and for every choice of $x$ and $U \subseteq V(G) \setminus \{x\}$ with $|U| \ge k$, it has an $(x,U)$-fan of size $k$ (i.e. with $k$ paths).
\end{theorem}

\begin{proof}
	$G$ is $k$-connected if and only if there is a $(x,U)$-fan of size $k$ for all $x, U$ ($|U| \ge k$).

	We begin by proving the forward direction: ``If $G$ is $k$-connected then there is a $(x,U)$-fan of size $k$ for all $x,U$''.

	Assume $G$ is $k$-connected and let $x \in V(G)$ and $U \subseteq V(G) \setminus \{x\}$ be arbitrary, such that $|U| \ge k$. We need to show that there is a $(x,U)$-fan in $G$.

	Let $G'$ be a graph obtained by adding a new vertex $y$ to graph $G$, where $N(y) = U$. By the expansion lemma (\ref{lemma:expansion}) $G'$ is $k$-connected. By Menger's theorem (\ref{the:menger}) there exist $k$ pairwise internally disjoint $(x,y)$-paths in $G$. Now, if we remove $y$ from each of these paths, we get an $(x,U)$-fan.

	Now we prove the other way: ``If there is a $(x,U)$-fan of size $k$ for all $x,U$, then $G$ is $k$-connected.''

	We assume now that $G$ satisfies the ``fan''-condition. Let $w,z \in V(G)$ be arbitrary. Let $U = N(z)$. $|U| \ge k$, as there otherwise is no $(z,R)$-fan for $R \subseteq V(G) \setminus \{z\}$ with $|R| \ge k$, which is a contradiction. Thus there exists a $(w,U)$-fan in $G$ of size $k$. We may assume $z$ is not part of the $(w,U)$-fan. For each of the $k$ $(w,U)$-paths in the fan, add the edge from the endpoint in $U$ to $z$, in order to obtain $k$ pairwise internally disjoint $(w,z)$-paths in $G$. As $w$ and $z$ were arbitrary $G$ is $k$-connected, by Menger's theorem.
\end{proof}

\begin{theorem}
	If $G$ is a $k$-connected graph where $k \ge 2$, and $S$ is a set of $k$ vertices in $G$, then $G$ has a cycle containing all vertices from $S$.
\end{theorem}

\begin{proof}
	We prove by induction on the size of $k$.

	\textit{Base Case $k = 2$}: This case follows from Menger's Theorem.\\
	\noindent
	\textit{Inductive Step $k > 2$}: Let $S = \{s_{1}, s_{2}, \ldots, s_{k}\}$. By induction there exists a cycle $C$ in $G$ containing the vertices $\{s_{1}, \ldots, s_{k-1}\}$ (as, if $G$ is $k$-connected, it is also $(k-1)$-connected). If $s_{k} \in V(C)$, then we are done. So assume that $s_{k} \notin V(C)$. We will now consider the following two cases:
	\\
	\noindent
	\textit{Case 1: $|V(C)| = k -1 $}\\
	\noindent
	\textit{Case 2: $|V(C)| \ge k$}\\
	\noindent
	As these cases exhaust all possibilities, the proof is complete.
\end{proof}


\section{Network Flow Problems}%
\label{sec:label}

A \textit{network} is a digraph, with an associated capacity function $c(e)$ on each edge $e$. A network distinguishes between two types of vertices: a \textbf{source vertex} denoted by $s$, and a \textbf{sink vertex} denoted by $t$. A \textbf{flow} $f$ assigns a value $f(e)$ to each edge $e$. We define two values $f^{+}(e)$ and $f^{-}(e)$ to be the total flow on edges leaving $v$ and the total flow on edges entering $v$, respectively.

We say that a flow is \textbf{feasible} if it satisfies the \textbf{capacity constraints} which are $0 \le f(e) \le c(e)$ and the \textbf{conservation constraint} $f^{+}(v) = f^{-}(v)$ for all $v \notin \{s,t\}$. I.e., the flow of an edged should be less than or equal to the capacity of that edge (such that it doest overflow), and the input should be equal to the output. By a simple observation, you can see that if the input was not equal to the output, there would either be coming flow out, that come from nowhere, or flow would transport to the void.

We define the \textbf{value} of a flow to be $val(f) = f^{-}(t) - f^{+}(t)$, where $t$ is the sink vertex. As a side note, one can show that this is equal to the $f^{+}(s) - f^{-}(s)$ where $s$ is the source node. A \textbf{maximum flow} is a feasible flow of maximum value. We distinguish the \textit{zero flow} as a flow which assigns zero to each edge.

Given any flow, we want to be able to decide if we can \textit{push} more flow from $s$ to $t$. We will do this using a new concept, the \textbf{residual network}. Assume we are given a network, $N$, and a flow, $f$. If $f(e) < c(e)$ then we can push up to $c(e)-f(e)$ more flow through $e$. If $e = uv$ and $f(uv) > 0$, then we can push up to $f(vu)$ flow through $e$, as it will ``cancel out''. Thus the residual network shows how much more flow we can push through any arc (or back through an arc). Here comes an important point about residual networks: If we cannot push any more flow through an arc, i.e., from $u$ to $v$, we do not include $e = uv$ in the residual network. Given a network, along with a residual network, given we can find an $(s,t)$-path, we can push more flow by ``augmenting'' (increasing the flow) through this path. Once we have no $(s,t)$-paths we have obtained a ``maximum flow'', that is, a flow which cannot be augmented any more.

The method described above is called the \textit{Ford-Fulkerson} method. As you have already seen, the algorithm is quite simple, but let's break it down into some easy to understand steps:
\begin{enumerate}
	\item Start with any flow $f$ (this may be the zero flow)
	\item Build the residual network $N'$ with relation to the flow $f$
	\item Decide if there is an $(s,t)$-path in $N'$.
	      \begin{itemize}
		      \item If yes, then push the flow through $f$ and go back to step 2.
		      \item If no, then the algorithm terminates.
	      \end{itemize}
\end{enumerate}


If $S \subseteq V(N)$ is a set such that $s \in S$ and $t \notin S$, then let $T = \bar{S}$ and denote $[S, T]$ as a \textbf{source/sink cut}.

The capacity of $[S, T]$ is the sum of the capacities on the arcs in $[S,T]$.

\begin{theorem}[Max-flow, Min-cut]
	In every network, the maximum value of a feasible flow is equal to the minimum capacity of a source/sink cut.
\end{theorem}

Before proving this, we need a lemma. Before getting to the lemma, we need the following definitions: Let $f^{+}(S)$ denote the sum of the flows of the arcs \textit{out of $S$}, i.e., in $[S, \bar{S}]$. Let $f^{-}$ denote the sum of the flows of the arcs \textit{into $S$}, i.e., in $[\bar{S}, S]$.

\begin{lemma}
	\label{lemma:4.3.7}
	Let $[S, T]$ be a source/sink cut in the network $N$, then $val(f) = f^{+}(S)  - f^{-}(S)$.
\end{lemma}

\begin{proof}
	\begin{equation*}
		\sum_{v \in T} (f^{-} - f^{+}(v)) = f^{+}(S) - f^{-}(S)
	\end{equation*}

	We get this formula from the flow conservation property. As much flow comes in, as leaves.
\end{proof}

\begin{proof}[Max-flow, Min-cut]
	We construct a maximum flow using Ford-Fulkerson's Algorithm. As is ensured by the algorithm, there will be no $(s,t)$-path in the residual network $N'$ when the algorithm terminates. Let $S = \{x \mid \exists (s,x)\text{-path in }N' \}$. Let $T = \bar{S}$ and note that $t \in T$.

	We now observer that $f(e) = c(e)$ for all $e \in [S, T]$ and $f(e') = 0$ for all $e' \in [T, S]$. This comes from the fact that we defined $S$ to only contain paths. As there is a path, we know it has been augmented.

	Therefore, $val(f) = f^{+}(S) - f^{-}(S)$ by Lemma~\ref{lemma:4.3.7}.

	So, the capacity of the source/sink cut $[S, \bar{S}]$ is equal to the value of a maximum feasible flow. There cannot be a source/sink cut $[S', \bar{S'}]$ with lower capacity, as then $val(f) = f^{+}(S) - f^{-}(S) \le c^{+}(S') < c^{+}(S) = val(f)$. Thus $[S, \bar{S}]$ is the source/sink cut of minimum capacity and this is equal to the maximum flow $f$.
\end{proof}

\begin{corollary}[Integrality Theorem]
	If all capacities in the network are integers, then there is a maximum flow assigning integral flow to each edge.
\end{corollary}

While we will not show a proof, the main idea of the proof would be that every step of the Ford-Fulkerson algorithm augments the flow by an integer.

We will now look at Menger's Theorem and Max-flow, Min-cut. Let $D$ be a digraph and $s, t \in V(D)$. Construct a network by putting capacity 1 on every arc. Let $f$ be a maximum $(s,t)$-flow (i.e., a maximum valued feasible flow with source $s$ and sink $t$). $f$ corresponds to $val(f)$ arc disjoint paths in $D$ (plus maybe some cycles which we can ignore). This is because all edges have capacity one, and by the max-flow min-cut theorem we have found the min-cut. The capacity of a source/sink cut $[S,T]$ is equal to the number of arcs in the $(s,t)$-arc-cut $[S,T]$, and the number of arcs in a $(s,t)$-arc-cut $[S,T]$ is equal to the capacity of the source/sink-cut $[S,T]$.

We can also use max-flow to find internally disjoint paths. Without modifying the capacity of the edges, we can restrict the amount of flow going through a vertex $v$, by ``splitting'' it into two vertices, $v^{-}$ which takes the incoming arcs, and $v^{+}$ which takes the outgoing arcs. Then, between these two vertices we add an arc with the restricted value $r$, which will then restrict the flow.

We can also use flows on undirected graphs. We replace each edge $e = uv$ with two arcs $a = uv$ and $a = vu$. Now we look at the question posed earlier, about the feasibility of finding internally disjoint paths with max-flow.

Let $G$ be a graph and $s, t \in V(D)$. Make $G$ into a digraph $D$, by replacing each edge with an arc in each direction (as stated earlier). Let all capacities be $1$. Using the splitting transformation described earlier, let each vertex have a bottleneck edge of capacity 1. Now find the maximum flow in this resulting network. This flow transforms $val(f)$ into internally disjoint $(s,t)$-paths. Any $k$ internally disjoint $(s,t)$-paths  would give rise to a flow of value $k$. So this gives us the maximum number of internally disjoint $(s,t)$-paths in $G$.

\end{document}
