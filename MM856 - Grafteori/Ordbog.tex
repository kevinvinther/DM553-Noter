\chapter{Ordbog}

\begin{note}[Kilder]
  \href{https://first.math.aau.dk/dan/2015/dmat/?file=dictionary.pdf}{Ordliste for DISKRET MATEMATIK}\\
  \noindent
  \href{https://www.georgmohr.dk/noter/grafteori2014.pdf}{Grafteori af Kirsten Rosenkilde}\\
  \noindent
  \href{https://noter.math.ku.dk/dis1-01.pdf}{Noter til Kombinatorik og Grafteori af Gunnar Forst}
\end{note}

Ordbogen kommer i rækkefølge fra hvornår jeg har brugt dem. Man kan ikke regne med at de er i samme rækkefølge som noterne, da jeg kan være gået tilbage og har fjernet/tilføjet tekst.

\begin{longtable}[c]{ll}
  \textbf{Engelsk} & \textbf{Dansk} \\ \hline \endfirsthead \endhead
  graph & graf \\
  vertex & knude \\
  edge & kant \\
  set & mængde \\
  endpoints & endepunkter \\
  loop & løkke \\
  multiple edges & multiple kanter \\
  adjacent vertices & naboknuder\footnote{vi bruger ikke \textit{adjacent} på dansk, kun nabo} \\
  neighbours & naboer \\
  null graph & den tomme graf \\
  path & sti \\
  clique & klikke \\
  independent set & uafhængig mængde \\
  bipartite & todelelig \\
  partite sets & delelige mængder \\
  chromatic number & kromatisk tal \\
  $k$-partite & $k$-delelige \\
  subgraph & delgraf \\
  connected & sammenhængende \\
  disconnected & usammenhængende \\
  isomorphism & isomorfi \\
  loopless & løkkefri\footnote{egen oversætning\label{egen}}\\
  adjacency matrix & nabo-matrix \\
  incidence matrix & incidens-matrix \\
  incidence & incidens \\
  incident & incident \\
  degree & valens\footnote{\textit{grad} kan også bruges} \\
  cycle & kreds \\
  complete graph & komplet graf \\
  biclique & biklikke\footref{egen}\\
  self-complementary & selv-komplementær \footref{egen}\\
  decomposition & nedbrydning \footref{egen}\\
  girth & omkreds \footref{egen}\\
  automorphism & automorfi \footref{egen}\\
  vertex-transitiv & knudetransitiv \footref{egen}\\
\end{longtable}

%%% Local Variables:
%%% mode: latex
%%% TeX-engine: luatex
%%% TeX-command-extra-options: "-shell-escape"
%%% TeX-master: "main"
%%% End:
