\chapter{Graf Teori}

\section{Chapter 1.1}

\noindent
\textbf{Exercise 1.1.6}\\
\noindent
\textit{Determine whether the graph below decomposes into copies of $P_{4}$.}
It does. See:

\begin{center}
	\begin{tikzpicture}[node distance={30mm}, thick, main/.style = {draw, circle}]
		\node[main] (1) {};
		\node[main] (2) [right=4cm of 1] {};
		\node[main] (3) [below of=1] {};
		\node[main] (4) [right=4cm of 3] {};
		\node[main] (5) [above right=13mm and 1cm of 3] {};
		\node[main] (6) [above left=13mm and 1cm of 4] {};
		\draw[-, red] (1) to node {} (2);
		\draw[-, red] (1) to node {} (3);
		\draw[-, red] (2) to node {} (4);

		\draw[-, blue] (3) to node {}(5);
		\draw[-, blue] (4) to node {}(6);
		\draw[-, blue] (3) to node {}(4);

		\draw[-, green] (1) to node {}(5);
		\draw[-, green] (5) to node{}(6);
		\draw[-, green] (6) to node{}(2);
	\end{tikzpicture}
\end{center}

\noindent
\textbf{Exercise 1.1.10}\\
\noindent
\textit{Prove or disprove: The complement of a simple disconnected graph must be connected.}

True. Given 2 non-adjacent vertices in $G$; $u, v$ if these are \textbf{not} connected in $G$ they must be in $\overline{G}$. We know such two vertices exist given the non-closedness. Any $z \in V(G) \setminus \{v, u\}$ not connected to $u,v$ will now be connected as well. Assuming the graph can be split up to $k$ components, ,each of the components will now be connected. As one vertex would connect to $k-1$ times the minimum amount of other vertices in a component, the graph would be connected.

Another way to see this, is that if two vertices were not connected in each their component, then they would be connected now. And now enough with that, but say $u$ is in component $A$ while $v$ is in component $B$. Because they are in each component, all vertices in $A$ will become adjacent to $v$, and vice versa. Because of this all vertices are now connected.
\\\\
\noindent
\textbf{Exercise 1.1.23}\\
\noindent

If time.
\\\\
\noindent
\textbf{Exercise 1.1.26}\\
\noindent
\textit{Let $G$ be a graph with girth $4$ in which every vertex has degree $k$. Prove that $G$ has at least $2k$ vertices. Determine all such graphs with exactly $2k$ vertices.}

Given a $v \in V(G)$, $|N(v)| = 4$, and form an independent set. THis means that $\forall n \in N(v)$, $d(n) = k$. We know that each of the neighbours also have $k$ neighbours each. Therefore, we have so far encountered $1 + k + k(k-1)$ vertices. 1 is for the start vertex, $k$ for each of its neighbours, and $k(k-1)$ for the neighbours neighbours. We take $k-1$ as we do not regard the start vertex. That is equal to $1 + k + k^{2} -k \ge 2k$. To be clear, if $k = 2$, then $1 + 2 + 2^{2} - 2 = 3 + 4 - 2 = 5$. If $k = 1$ a girth of $4$ is impossible.

In the exercise class he showed that if $y \leftrightarrow x$, then $N(y) \cap N(x) = \emptyset$. This is $2k$. For the construction, the answer is bipartite graphs, $K_{k,k}$
\\\\
\noindent
\textbf{Exercise 1.1.38}\\
\noindent
\textit{Let $G$ be a simple graph in which every vertex has degree 3. Prove that $G$ decomposes into claws if and only if $G$ is bipartite.}\\\\
\noindent
\textbf{If $G$ is bipartite, it decomposes into claws}.\\
\noindent
For a graph to decompose into claws, we can see it as a vertex with 3 edges emerging from it. As the graph is $3$-regular, we can find these claws in bipartite graphs. Given a vertex in set $A$ (or $B$), take that along with its 3 neighbours. Continue doing this for all $v \in A$, and you have $3|V(G)|$ claws.\\\\
\noindent
\textbf{If $G$ decomposes into claws, $G$ is bipartite.}\\
\noindent
Each claw is itself a bipartite structure, $K_{1,3}$ and thus can only be decomposed by other bipartite graphs.


\section{Chapter 1.2}
\noindent
\textbf{Exercise 1.2.20}\\
\noindent
\textit{Let $v$ be a cut-vertex of a simple graph $G$. Prove that $\overline{G} - v$ is connected.}

As $v$ is a cut-vertex, $N(v)$ is not connected. We can follow the same logic from 1.1.10 showing that $N(v)$ would now be connected, as is the same, with the rest of the graph. We can also just imagine that $G$ is now not connected, and follow the same reasoning.
\\\\
\noindent
\textbf{Exercise 1.2.25}\\
\noindent
\textit{Use ordinary induction on the number of edges to prove that absence of odd cycles is a sufficient condition for a graph to be bipartite.}\\\\
\noindent
\textbf{Base Case: $e(G) = 0$}: \\
\noindent
If a graph has no edges it is trivially bipartite, and has no odd cycles.\\
\noindent
\textbf{Induction hypothesis}: Any graph with $e(G)$ edges that contains no odd cycles is bipartite.\\
\noindent
\textbf{Inductive Step}:\\
\noindent
We show that a graph with $e(G) + 1$ edges and no odd cycles is bipartite.

Let $G = (V,E)$ be a graph with $e(G) + 1$ edges and no odd cycles. Let $e = uv$ be an arbitrary edge. The removal of $e$ from $G$ results in a subgraph $G'$ with $e(G)$ edges. By the induction hypothesis, since $G'$ contains no odd cycles, $G'$ is bipartite.

If $e$ is a cut-edge, the graph $G'$ consists now of two disconnected components, each of which is bipartite. Combining these sets, such that the endpoints of $e$ lie in each their bipartition makes the graph bipartite.

If $e$ is not a cut-edge, removing it leaves a connected graph $G'$, which is bipartite by the induction hypothesis.
\\\\
\noindent
\textbf{Exercise 1.2.26}\\
\noindent


\section{Chapter 1.3}
\noindent
\textbf{Exercise 1.3.12}\\
\noindent
\textit{Prove that an even graph has no cut-edge. For each $k \ge 1$, construct a $2k+1$ regular simple graph having a cut-edge.}

An even graph must have at least $2k, k \in \mathbb{N}$ edges. This implies that every vertex is part of a cycle, and a vertex or edge cannot be a cut-vertex or edge if it is a part of a cycle. This is by proposition 1.2.27: every even graph decomposes into cycles, and by theorem 1.2.14: an edge cannot be a cut-edge if in a cycle.
\\\\
\noindent
\textbf{Exercise 1.3.50}\\
\noindent
\textit{For $n \ge 3$, determine the minimum number of edges in a connected $n$-vertex graph in which every edge belongs to a triangle.}

\[T_{n} = \lceil \frac{3(n-1)}{2} \rceil\]

The minimum number is at least $T(n)$. If $n$ is odd we can just start with $K_{3}$, and then add 2 vertices and edges in such a fashion that it creates a new triangle. Thus we get the correct number from $T(n)$. If $n$ is even, we start with $n = 4$, and then add two vertices following the same strategy as with odd $n$.

$T_{n}$ is minimal. We prove by induction.

\textit{Basis: $n=3$}: The triangle is the smallest such graph, and $T_{n}$ is true here.

\textit{Induction step: $n \rightarrow n+1$}: $T_{n} \le \lceil \frac{3(n-1)}{2} \rceil$. $2T_{n} = \sum_{v \in V} \delta(v) \ge 3n \Rightarrow T_{n} \ge \frac{3n}{2} \Rightarrow T_{n}$ must have a vertex with valency 0, 1, or 2.


\section{Chapter 1.4}
\noindent
\textbf{Exercise 1.4.8}\\
\noindent
\textit{Prove that there is an $n$-vertex tournament with $d^{-}(v) = d^{+}(v) \forall v \in V(G)$ if and only if $n$ is odd.}\\

First, if $n$ is even there is an odd amount of edges, and therefore $d^{+}(v) \ne d^{-}(v) \forall v \in V(G)$. For an example, theorem 1.2.26 says that $K_{n}, n = 2k+1$ has an Eulerian graph, thus we can orient according to the trial. I.e., if going from A to B in the graph, we add a directed edge going from A to B. This way, whenever we go into a vertex, we will come out of it again, and the other way around.
\\\\
\noindent
\textbf{Exercise 1.4.10}\\
\noindent
\textit{Prove that a digraph is strongly connected if and only if for each partition of the vertex set into nonempty sets $S$ and $T$, there is an edge from $S$ to $T$.}\\
\noindent
$\Rightarrow$\\
Follows the same logic. If it was not possible, the underlying graph would not be connected.\\
\noindent
$\Leftarrow$\\
If for each partition set there is an edge from $S$ to $T$, the underlying graph is \textbf{at least} connected, as this is ensured in every partition.
It is strongly connected as there is always a path from $S$ to $T$ and vice versa.
\\\\
\noindent
\textbf{Exercise 1.4.14}\\
\noindent
If time.
\\\\
\noindent
\textbf{Exercise 1.4.38}\\
\noindent
\textit{For $n \in \mathbb{N}$, prove that there is an $n$-vertex tournament in which every vertex is a king if and only if $n \notin \{2, 4\}$}\\

???



\section{Chapter 2.1}
\noindent
\textbf{Exercise 2.1.5}\\
\noindent
\textit{Let $G$ be a graph. Prove that a maximal acyclic subgraph of $G$ consists of a spanning tree from each component of $G$.}

As it is acyclic there are no cycles. If there is a cycle in a component, we do not add the edges making it a cycle. Thus it becomes a spanning tree.
\\\\
\noindent
\textbf{Exercise 2.1.15}\\
\noindent
\textit{Let $G$ be a simple graph with diameter of at least 4. Prove that $\overline{G}$ has a diameter at most 2.}

Consider two vertices, $u, v$ whose length is $\ge 4$. In $\overline{G}$ this is no longer true, as they will be neighbours. Therefore the length will be 2. This happens in general to all $k = 2, 3$. If $k = 1$ in $G$ there will be no line in $\overline{G}$, and thus will not be represented in the diameter.
\\\\
\noindent
\textbf{Exercise 2.1.39}\\
\noindent
\textit{Let $G$ be a tree with $2k$ vertices of odd degree. Prove that $G$ decomposes into $k$ paths. (Hint: Prove the stronger result that claim holds for all forests.)}\\
Missing.
\section{Chapter 2.2}
\noindent
\textbf{Exercise 2.2.1}\\
\noindent
\textit{Determine which trees have Prüfer codes that (a) contain only one value, (b) contain exactly two values, or (c) have distinct values in all positions.}\\

\begin{enumerate}
	\item[(a).] Trees that have Prüfer codes which contain only one value are trees of order 3. Removing one vertex leave only one value in the code.
	\item[(b).] Same logic applies here, but with order 4.
\end{enumerate}
\noindent
\textbf{Exercise 2.2.7}\\
\noindent
\textit{Use Cayley's Formula to prove that the graph obtained from $K_{n}$ by deleting an edge has $(n-2)n^{n-3}$ spanning trees.}
\\\\
\noindent
\textbf{Exercise 2.2.25}\\
\noindent
\textit{Prove that if $G$ is graceful and Eulerian, then $e(G)$ is congruent to 0 or 3 mod 4. (Hint: Sum the absolute edge differences (mod 2) in two different ways.)}
\section{Chapter 2.3}
\textit{Use Dijkstra’s algorithm to find the shortest path from v1 to all other
	vertices in the below graph and use Kruskal’s algorithm to find a minimum
	spanning tree in the below graph.}
\section{Chapter 3.1}
\noindent
\textbf{Exercise 3.1.8}\\
\noindent
\textit{Prove or disprove: Every tree has at most one perfect matching.}
\\\\
\noindent
\textbf{Exercise 3.1.21}\\
\noindent
\textit{Let $G$ be an $X,Y$-bigraph such that $|N(S) > |S|$ whenever $\emptyset \ne S \in X$. Prove that every edge of $G$ belongs to some matching that saturates $X$.}


%%% Local Variables:
%%% mode: latex
%%% TeX-engine: luatex
%%% TeX-command-extra-options: "-shell-escape"
%%% TeX-master: "main"
%%% End:
