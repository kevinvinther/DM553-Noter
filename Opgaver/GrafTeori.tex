\chapter{Graf Teori}

\section{Chapter 1.1}

\noindent
\textbf{Exercise 1.1.6}\\
\noindent
\textit{Determine whether the graph below decomposes into copies of $P_{4}$.}
It does. See:

\begin{center}
	\begin{tikzpicture}[node distance={30mm}, thick, main/.style = {draw, circle}]
		\node[main] (1) {};
		\node[main] (2) [right=4cm of 1] {};
		\node[main] (3) [below of=1] {};
		\node[main] (4) [right=4cm of 3] {};
		\node[main] (5) [above right=13mm and 1cm of 3] {};
		\node[main] (6) [above left=13mm and 1cm of 4] {};
		\draw[-, red] (1) to node {} (2);
		\draw[-, red] (1) to node {} (3);
		\draw[-, red] (2) to node {} (4);

		\draw[-, blue] (3) to node {}(5);
		\draw[-, blue] (4) to node {}(6);
		\draw[-, blue] (3) to node {}(4);

		\draw[-, green] (1) to node {}(5);
		\draw[-, green] (5) to node{}(6);
		\draw[-, green] (6) to node{}(2);
	\end{tikzpicture}
\end{center}

\noindent
\textbf{Exercise 1.1.10}\\
\noindent
\textit{Prove or disprove: The complement of a simple disconnected graph must be connected.}

True. Given 2 non-adjacent vertices in $G$; $u, v$ if these are \textbf{not} connected in $G$ they must be in $\overline{G}$. We know such two vertices exist given the non-closedness. Any $z \in V(G) \setminus \{v, u\}$ not connected to $u,v$ will now be connected as well. Assuming the graph can be split up to $k$ components, ,each of the components will now be connected. As one vertex would connect to $k-1$ times the minimum amount of other vertices in a component, the graph would be connected.

Another way to see this, is that if two vertices were not connected in each their component, then they would be connected now. And now enough with that, but say $u$ is in component $A$ while $v$ is in component $B$. Because they are in each component, all vertices in $A$ will become adjacent to $v$, and vice versa. Because of this all vertices are now connected.
\\\\
\noindent
\textbf{Exercise 1.1.23}\\
\noindent

If time.
\\\\
\noindent
\textbf{Exercise 1.1.26}\\
\noindent
\textit{Let $G$ be a graph with girth $4$ in which every vertex has degree $k$. Prove that $G$ has at least $2k$ vertices. Determine all such graphs with exactly $2k$ vertices.}

Given a $v \in V(G)$, $|N(v)| = 4$, and form an independent set. THis means that $\forall n \in N(v)$, $d(n) = k$. We know that each of the neighbours also have $k$ neighbours each. Therefore, we have so far encountered $1 + k + k(k-1)$ vertices. 1 is for the start vertex, $k$ for each of its neighbours, and $k(k-1)$ for the neighbours neighbours. We take $k-1$ as we do not regard the start vertex. That is equal to $1 + k + k^{2} -k \ge 2k$. To be clear, if $k = 2$, then $1 + 2 + 2^{2} - 2 = 3 + 4 - 2 = 5$. If $k = 1$ a girth of $4$ is impossible.

In the exercise class he showed that if $y \leftrightarrow x$, then $N(y) \cap N(x) = \emptyset$. This is $2k$. For the construction, the answer is bipartite graphs, $K_{k,k}$
\\\\
\noindent
\textbf{Exercise 1.1.38}\\
\noindent
\textit{Let $G$ be a simple graph in which every vertex has degree 3. Prove that $G$ decomposes into claws if and only if $G$ is bipartite.}\\\\
\noindent
\textbf{If $G$ is bipartite, it decomposes into claws}.\\
\noindent
For a graph to decompose into claws, we can see it as a vertex with 3 edges emerging from it. As the graph is $3$-regular, we can find these claws in bipartite graphs. Given a vertex in set $A$ (or $B$), take that along with its 3 neighbours. Continue doing this for all $v \in A$, and you have $3|V(G)|$ claws.\\\\
\noindent
\textbf{If $G$ decomposes into claws, $G$ is bipartite.}\\
\noindent
Each claw is itself a bipartite structure, $K_{1,3}$ and thus can only be decomposed by other bipartite graphs.


\section{Chapter 1.2}
\noindent
\textbf{Exercise 1.2.20}\\
\noindent
\textit{Let $v$ be a cut-vertex of a simple graph $G$. Prove that $\overline{G} - v$ is connected.}

As $v$ is a cut-vertex, $N(v)$ is not connected. We can follow the same logic from 1.1.10 showing that $N(v)$ would now be connected, as is the same, with the rest of the graph. We can also just imagine that $G$ is now not connected, and follow the same reasoning.
\\\\
\noindent
\textbf{Exercise 1.2.25}\\
\noindent
\textit{Use ordinary induction on the number of edges to prove that absence of odd cycles is a sufficient condition for a graph to be bipartite.}\\\\
\noindent
\textbf{Base Case: $e(G) = 0$}: \\
\noindent
If a graph has no edges it is trivially bipartite, and has no odd cycles.\\
\noindent
\textbf{Induction hypothesis}: Any graph with $e(G)$ edges that contains no odd cycles is bipartite.\\
\noindent
\textbf{Inductive Step}:\\
\noindent
We show that a graph with $e(G) + 1$ edges and no odd cycles is bipartite.

Let $G = (V,E)$ be a graph with $e(G) + 1$ edges and no odd cycles. Let $e = uv$ be an arbitrary edge. The removal of $e$ from $G$ results in a subgraph $G'$ with $e(G)$ edges. By the induction hypothesis, since $G'$ contains no odd cycles, $G'$ is bipartite.

If $e$ is a cut-edge, the graph $G'$ consists now of two disconnected components, each of which is bipartite. Combining these sets, such that the endpoints of $e$ lie in each their bipartition makes the graph bipartite.

If $e$ is not a cut-edge, removing it leaves a connected graph $G'$, which is bipartite by the induction hypothesis.
\\\\
\noindent
\textbf{Exercise 1.2.26}\\
\noindent


\section{Chapter 1.3}
\noindent
\textbf{Exercise 1.3.12}\\
\noindent
\textit{Prove that an even graph has no cut-edge. For each $k \ge 1$, construct a $2k+1$ regular simple graph having a cut-edge.}

An even graph must have at least $2k, k \in \mathbb{N}$ edges. This implies that every vertex is part of a cycle, and a vertex or edge cannot be a cut-vertex or edge if it is a part of a cycle. This is by proposition 1.2.27: every even graph decomposes into cycles, and by theorem 1.2.14: an edge cannot be a cut-edge if in a cycle.
\\\\
\noindent
\textbf{Exercise 1.3.50}\\
\noindent
\textit{For $n \ge 3$, determine the minimum number of edges in a connected $n$-vertex graph in which every edge belongs to a triangle.}

\[T_{n} = \lceil \frac{3(n-1)}{2} \rceil\]

The minimum number is at least $T(n)$. If $n$ is odd we can just start with $K_{3}$, and then add 2 vertices and edges in such a fashion that it creates a new triangle. Thus we get the correct number from $T(n)$. If $n$ is even, we start with $n = 4$, and then add two vertices following the same strategy as with odd $n$.

$T_{n}$ is minimal. We prove by induction.

\textit{Basis: $n=3$}: The triangle is the smallest such graph, and $T_{n}$ is true here.

\textit{Induction step: $n \rightarrow n+1$}: $T_{n} \le \lceil \frac{3(n-1)}{2} \rceil$. $2T_{n} = \sum_{v \in V} \delta(v) \ge 3n \Rightarrow T_{n} \ge \frac{3n}{2} \Rightarrow T_{n}$ must have a vertex with valency 0, 1, or 2.


\section{Chapter 1.4}
\noindent
\textbf{Exercise 1.4.8}\\
\noindent
\textit{Prove that there is an $n$-vertex tournament with $d^{-}(v) = d^{+}(v) \forall v \in V(G)$ if and only if $n$ is odd.}\\

First, if $n$ is even there is an odd amount of edges, and therefore $d^{+}(v) \ne d^{-}(v) \forall v \in V(G)$. For an example, theorem 1.2.26 says that $K_{n}, n = 2k+1$ has an Eulerian graph, thus we can orient according to the trial. I.e., if going from A to B in the graph, we add a directed edge going from A to B. This way, whenever we go into a vertex, we will come out of it again, and the other way around.
\\\\
\noindent
\textbf{Exercise 1.4.10}\\
\noindent
\textit{Prove that a digraph is strongly connected if and only if for each partition of the vertex set into nonempty sets $S$ and $T$, there is an edge from $S$ to $T$.}\\
\noindent
$\Rightarrow$\\
Follows the same logic. If it was not possible, the underlying graph would not be connected.\\
\noindent
$\Leftarrow$\\
If for each partition set there is an edge from $S$ to $T$, the underlying graph is \textbf{at least} connected, as this is ensured in every partition.
It is strongly connected as there is always a path from $S$ to $T$ and vice versa.
\\\\
\noindent
\textbf{Exercise 1.4.14}\\
\noindent
If time.
\\\\
\noindent
\textbf{Exercise 1.4.38}\\
\noindent
\textit{For $n \in \mathbb{N}$, prove that there is an $n$-vertex tournament in which every vertex is a king if and only if $n \notin \{2, 4\}$}\\

???



\section{Chapter 2.1}
\noindent
\textbf{Exercise 2.1.5}\\
\noindent
\textit{Let $G$ be a graph. Prove that a maximal acyclic subgraph of $G$ consists of a spanning tree from each component of $G$.}

As it is acyclic there are no cycles. If there is a cycle in a component, we do not add the edges making it a cycle. Thus it becomes a spanning tree.
\\\\
\noindent
\textbf{Exercise 2.1.15}\\
\noindent
\textit{Let $G$ be a simple graph with diameter of at least 4. Prove that $\overline{G}$ has a diameter at most 2.}

Consider two vertices, $u, v$ whose length is $\ge 4$. In $\overline{G}$ this is no longer true, as they will be neighbours. Therefore the length will be 2. This happens in general to all $k = 2, 3$. If $k = 1$ in $G$ there will be no line in $\overline{G}$, and thus will not be represented in the diameter.
\\\\
\noindent
\textbf{Exercise 2.1.39}\\
\noindent
\textit{Let $G$ be a tree with $2k$ vertices of odd degree. Prove that $G$ decomposes into $k$ paths. (Hint: Prove the stronger result that claim holds for all forests.)}\\
Missing.
\section{Chapter 2.2}
\noindent
\textbf{Exercise 2.2.1}\\
\noindent
\textit{Determine which trees have Prüfer codes that (a) contain only one value, (b) contain exactly two values, or (c) have distinct values in all positions.}\\

\begin{enumerate}
	\item[(a).] Trees that have Prüfer codes which contain only one value are trees of order 3. Removing one vertex leave only one value in the code.
	\item[(b).] Same logic applies here, but with order 4.
\end{enumerate}
\noindent
\textbf{Exercise 2.2.7}\\
\noindent
\textit{Use Cayley's Formula to prove that the graph obtained from $K_{n}$ by deleting an edge has $(n-2)n^{n-3}$ spanning trees.}
\\\\
\noindent
\textbf{Exercise 2.2.25}\\
\noindent
\textit{Prove that if $G$ is graceful and Eulerian, then $e(G)$ is congruent to 0 or 3 mod 4. (Hint: Sum the absolute edge differences (mod 2) in two different ways.)}\\
Missing.
\section{Chapter 2.3}
\textit{Use Dijkstra’s algorithm to find the shortest path from v1 to all other vertices in the below graph and use Kruskal’s algorithm to find a minimum spanning tree in the below graph.}\\
Missing.
\section{Chapter 3.1}
\noindent
\textbf{Exercise 3.1.8}\\
\noindent
\textit{Prove or disprove: Every tree has at most one perfect matching.}\\
Missing.
\\\\
\noindent
\textbf{Exercise 3.1.21}\\
\noindent
\textit{Let $G$ be an $X,Y$-bigraph such that $|N(S)| > |S|$ whenever $\emptyset \ne S \subset X$. Prove that every edge of $G$ belongs to some matching that saturates $X$.}\\

This is a restatement of Hall's marriage theorem.



\section{Chapter 3.3}%
\label{subsec:label}

\noindent
\textbf{Exercise 3.3.1}\\
\noindent
\textit{Determine whether the graph below has a 1-factor.}\\
You can prove this with Tutte's theorem. Specifically when $|S| =4$.
\\\\
\noindent
\textbf{Exercise 3.3.6}\\
\noindent
\textit{Prove that a tree $T$ has a perfect matching if and only if $o(T-v)=1$ for every $v \in V(T)$.}\\
\noindent
\textbf{Forward Direction}: If $T$ has a perfect matching, then $o(T-v) = 1$ for all $v \in V(T)$.
\textit{Suppose} that $T$ has a perfect matching $M$. Consider any vertex $v \in V(T)$: Since $M$ is a perfect matching, $v$ is matched to exactly one vertex $u \in M$. When $v$ is removed from $T$, the vertex $u$ becomes unmatched.
You cannot match on odd components, therefore there has to be at most one.\\
\noindent
\textbf{Reverse Direction}: If $o(T-v) = 1$ for every $v \in V(T)$, then $T$ has a perfect matching.\\
\textit{Assume} for every vertex $v \in V(T), \; o(T-v)=1$.\\
\textit{Base Case $|V(T)| = 2$}: This is trivially true, as there is one trivial perfect matching, and $\forall v \in V(T), \; o(T-v) =1$.\\
\textit{Inductive Step $|V(T)| > 2|$, $|V(T)| = n$, assume all trees with fewer than $n$ vertices satisfying $\forall v \in V(T) : o(T-v)=1$ has a perfect matching}:
\\\\
\noindent
\textbf{Exercise 3.3.25}\\
\noindent
\textit{A graph $G$ is \textbf{factor-critical} if each subgraph $G-v$ obtained by deleting one vertex has a 1-factor. Prove that $G$ is factor-critical if and only if $n(G)$ is odd and $o(G-S) \le |S|$ for all non-empty $S \subseteq V(G)$.}\\
We prove both directions:

\textbf{Forward Direction:} ``$G$ is factor-critical \(\Rightarrow\) $n(G)$ is odd and $\forall S \subseteq V(G) : o(G-S) \le |S|$''. \\
To begin with: We know from the earlier exercise that a perfect matching must be even (and in this case, we can consider 1-factors to be equivalent to perfect matchings).

\section{Chapter 4.1}%
\label{sec:4.1}

\noindent
\textbf{Exercise 4.1.2 }\\
\noindent
\textit{Give a counterexample to the following statement, add a hypothesis to correct it, and prove the corrected statement: If $e$ is a cut-edge of $G$, then at least one vertex of $e$ is a cut-vertex of $G$.}\\
Consider the graph $K_{2}$. Removing $e$ (of which there is only one) disconnects the graph. Removing any vertex does not disconnect the graph.\\
We can correct it as follows:
``If $G$ is a connected graph with \textbf{at least three} vertices and $e$ is a cut-edge of $G$, then at least one vertex of $e$ is a cut-vertex of $G$.''. Let's prove it:

Given $e = uv$ is a cut-edge, removing the edge splits the graph into components $G_{1}$ and $G_{2}$. While it is arbitrary, we say that $u \in G_{1}$ and $v \in G_{2}$. If $G_{1} = \{u\}$, then there must be at least two more vertices, but $u$ is not a cut-vertex. However, $v$ is (as $e = uv$). If $u \subset G_{1} \land G_{1} \ne \{u\}$ then $u$ is a cut-vertex.

This is wrong. Rather than updating it to be $n \ge 3$, it should be ``The valency of $v$ is larger than one for both vertices in $K_{2}$ in a non-$K_{2}$-free graph.'' Or not?
\\\\
\noindent
\textbf{Exercise 4.1.4 }\\
\noindent
\textit{Prove that $G$ is $k$-connected if and only if $G \lor K_{r}$ is $k+r$-connected.}\\
\textit{Claim}: Any disconnecting set $S$ of $G \land K_{r}$ must contain $K_{r}$. Assume otherwise: Then any two vertices $v_{1}, v_{2} \in G - S$ are connected in a way such that $K_{r}$ connects them.  If $K_{r} \setminus S$ has at least two vertices, then there is a connection in $K_{r} \setminus S$.

\textit{Claim} \(\Rightarrow S = K_{r} \cup (S' \subseteq V(G)), S \text{ seperates } G \vee K_{r} \iff S' \text{ seperates } G\). This can be rewritten by observing that $(G \vee K_{r}) - S = G - S'$.

So, if $G$ is $k$-connected, then $S' \ge k \Rightarrow S \ge k+r$. By the claim, $(S-K_{r}) = k$, so $S' \ge k \iff S \ge k+r$.
\\\\
\noindent
\textbf{Exercise 4.1.6 }\\
\noindent
\textit{For a graph $G$ with blocks $B_{1}, \ldots, B_{k}$, prove that $n(G) = (\sum_{i=1}^k n(B_{i}))-k+1$.}\\

In essence, this statement says that we sum the number of vertices in the blocks and subtract $k+1$ in order to get the number of vertices in the entire graph.

Let's recall the definition of a \textit{block}:
\begin{definition}[Block]
	A \textbf{block} of a graph $G$ is a maximally connected subgraph of $G$ that has no cut-vertex. If $G$ itself is connected and has no cut-vertex, then $G$ is a block.
\end{definition}

Rather than proving inductively, I will give a short argument. Given $n = 1$, it is trivially true, as $n(G) = |B_{1}| = |B_{1}| + 1 - 1$.

If we have $k$ blocks, then two of the blocks either share a vertex, or the block is an entire maximally connected subgraph.

Given two maximally connected subgraphs with no cut-vertex $G_{1} \sqcup G_{2}$, $n(G) = |B_{1}| + |B_{2}| \ne |B_{1}| + |B_{2}| - 2 + 1$. Thus the statement is false. This would however be true if $G$ was connected.

\textbf{Corrected statement: $G$ is connected}.

\begin{definition}[Block-cutpoint Graph]
	The block-cutpoint graph of a graph $G$ is a bipartite graph with one vertex set being the cut-vertices of $G$, and the other vertex set consisting of one vertex per block.
\end{definition}

Let $v$ be a cut-vertex of $G$, and $b_{i}$ be a vertex corresponding to a block. Then $vb_{i}$ is an edge if and only if $v$ belongs to the block.

\textit{Claim}: There is one block which contains only one cut-vertex of $G$.

This is implied by the fact that block-cutpoint graph is a tree. If not, it would contain a cycle. However, it does not contain a cycle, as you would then just be able to declare that a block in itself.

So the fact that a tree has at least one leaf implies the claim.

Let $b_{k}$ be this block and $v \in B_{k}$ be the cut-vertex of $G$. By induction on the number of blocks you get the formula:

\[
	n(G-(B_{k} \cdot v)) = \sum_{i=1}^{k-1} n(B_{i})-(k-1)+1
\]

This implies that $n(G) = \sum_{i=1}^k n(B_{k}) - k +1$.
\\\\
\noindent
\textbf{Exercise 4.1.15 }\\
\noindent
\textit{Use Proposition 4.1.12 and Theorem 4.1.11 to prove that the Petersen Graph is 3-connected}\\

\textbf{Theorem 4.1.11}: If $G$ is 3-regular, then \(\kappa(G) = \kappa'(G)\).

\textbf{Proposition 4.1.12}:
\[
	|[S, \bar{S}]|| = \sum_{v \in S} d(v) - 2e(G[S])
\]

Let $[S, \bar{S}]$ be a minimal edge cut. Then $|[S, \bar{S}]| \ge 3$. We assume $|S| \le |\bar{S}|$, then $|S| \le 5$. Using Corollary 1.1.40 implies that $G$ has girth 5.

\textit{Case 1}: $|S| < 5 \Rightarrow G[S]$ is a forest. This implies that $e(G[S]) \le |S|-1$.

\textit{Case 2}: $|S| = 5$, and $S$ is a forest. Apply previous argument.

\textit{Case 3}: $|S| = 5$ and is a 5-cycle.

\section{Chapter 4.2}%
\label{sec:label}

\noindent
\textbf{Exercise 4.2.4 }\\
\noindent
\textit{Prove or disprove: If $P$ is a $(u,v)$-path in a 2-connected graph $G$, then there is a $(u,v)$-path $Q$ that is internally disjoint from $P$.}\\
If $P$ is a $(u,v)$-path in a 2-connected graph, then there is not necessarily an internally disjoint $(u,v)$-path. You can't fix one of the path's, the example given was $|V(G)| = 4$ with the path forming a $Z$ pattern, which could not give another internally disjoint path.
\\\\
\noindent
\textbf{Exercise 4.2.6 }\\
\noindent
\textit{Use results from this section to prove that a simple graph $G$ is $2$-connected if and only if $G$ can be obtained from $C_{3}$ by a sequence of edge additions and edge subdivisions.}\\
(For edge addition the endpoints must be distinct).

To prove this we use the ear decomposition theorem.

\textbf{Theorem}: $G$ is 2-connected $\iff$ it admits an ear-decomposition.

Ear-decomposition: $G = C_{k} \cup P_{1} \cup \cdots \cup P_{k}$.

We start with $C_{3}$, and subdivide edges until we obtain $C_{k}$. Add an edge at $C_{k} \cap P_{1}$ and subdivide until you get $C_{k} \cup P_{1}$.

$C_{3} + $edge addition + subdivision \(\Rightarrow\) 2-connected.

We prove this inductively upon the number of edges.

\textit{Base Case} $C_{3}$: This is trivially 2-connected.

\textit{Induction step}: Let $G$ be 2-connected and consider $G$ + edge addition. Adding edges does not decrease connectivity. Consider $G$ + subdivision.
\\\\
\noindent
\textbf{Exercise 4.2.7 }\\
\noindent
\textit{Let $xy$ be an edge in a digraph $G$. Prove that $\kappa(G-xy) \ge \kappa(G) - 1$.}\\
missing
\\\\








%%% Local Variables:
%%% mode: latex
%%% TeX-engine: luatex
%%% TeX-command-extra-options: "-shell-escape"
%%% TeX-master: "main"
%%% End:
