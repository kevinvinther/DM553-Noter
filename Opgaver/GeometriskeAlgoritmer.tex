\chapter{Geometriske Algoritmer}

\section{Kapitel 1}%
\label{sec:99}

Exercises (1.1, 1.2,) 1.3, 1.4.a, 1.7, 1.8, 1.9.

Exercises 1.1 and 1.2 are about proving that the intersection of all convex sets containing P equals the smallest-perimeter polygon with vertices from P containing P. A better sequence of small proofs is the one below, which we will carry out instead of following the outline of Exercises 1.1 and 1.2. Prove that

the intersection of a collection of convex sets is again convex.
any smallest-perimeter polygon containing P has vertices only from P.
any smallest-perimeter polygon containing P is contained in any convex set containing P.
any smallest-perimeter polygon containing P is convex.
any smallest-perimeter polygon with vertices from P equals the intersection of all convex sets containing P (making the polygon unique).

Many of the arguments are most easily carried out as proofs by contradiction.

In 1.8.a, you may assume that the two convex hulls are separated by a vertical line.

\noindent
\textbf{1.1, 1.2}

Prove that:

\begin{itemize}
	\item The intersection of a collection of convex sets is again convex
	      \begin{itemize}
		      \item Given: Convex sets. Transform: Intersection. Result: Convex Set.
		      \item \textbf{Intuition}: If every set in a collection is convex, and we take two points in the intersection of all these sets, then the line segment between those two points must be inside every indivudal convex set. Therefore, it has to be in the intersection, too.
		      \item \textbf{Proof}: Assume the intersection isn't convex. In that case, you'd be able to find two points in the intersection whose line $\overline{pq}$ would go outside the convex set. For this to be true, at least one point would have to be \textit{outside} of the original convex sets. This is a contradiction, as taking the intersection of sets introduces no new elements of the set.
	      \end{itemize}
	\item Any smallest-perimeter polygon containing $P$ has vertices only from $P$.
	      \begin{itemize}
		      \item \textbf{Proof}. Imagine building the smallest polygon. But, while building, instead of using only vertices from $P$, we include a new vertex outside of $P$, which we assume would make the polygon's perimeter smaller. For it to become smaller, however, the point not in $P$ would have to be in $P$. This is a contradiction, as we assumed the point to not be in $P$.
	      \end{itemize}
	\item Any smallest-perimeter polygon containing $P$ is contained in any convex set containing $P$.
	      \begin{itemize}
		      \item This is about proving that, if we have $P$ which is smallest-perimeter, then it must fit in any set containing $P$.
		      \item \textbf{Proof}.
		            Assume that our smallest-perimeter polygon $P$ wasn't contained in the convex set, which we call $C$. There are two possibilites now:
		            \begin{enumerate}
			            \item We can adjust $P$ such that it fits inside $C$, since $C$ contains $P$.
			                  \begin{itemize}
				                  \item \textbf{Contradiction}: We did not start with the smallest set.
			                  \end{itemize}
			            \item The set $C$ does not contain all the points of $P$.
			                  \begin{itemize}
				                  \item \textbf{Contradiction}: This is impossible as it is stated that $C$ contains $P$.
			                  \end{itemize}
		            \end{enumerate}

	      \end{itemize}
	\item Any smallest perimeter polygon containing $P$ is convex.
	      \begin{itemize}
		      \item \textbf{Intuition}: If the polygon isn't convex, then it's possible to make it smaller. Recall that a convex polygon is one that doesn't ``cave in'' anywhere.

		      \item \textbf{Proof}:
		            Assume that the smallest-perimeter polygon isn't convex. Then, we have a concave polygon. At least one point in this concave polygon has a interior angle which is greater than 180 degrees. Assuming that we can list the points in an order, where the lines segments endpoints are ordered in such a way that the line segment $\overline{qr}$ will come after $\overline{pq}$. Then we can remove the line which has an interior angle of above 180 degrees, along with the next line, i.e. if we call the point $p_{i}$, then we remove $\overline{p_{i}p_{i+1}}$ and $\overline{p_{i+1}p_{i+2}}$ and replace it with the line segment $\overline{p_{i}p_{i+2}}$. We can algorithmically look at the same point again, and see if it has a interior angle of above 180 degrees, and, if so, we continue until the interior angle is below 180 degrees. However, this would shorten the perimeter, as there are no longer any reflex angles, which would keep the perimeter longer. This is a contradiction as we started with the shortest perimeter polygon.
	      \end{itemize}
	\item Any smallest-perimeter polygon with vertices from $P$ equals the intersection of all convex sets containing $P$ (making the polygon unique).
	      \begin{itemize}
		      \item \textbf{Proof that the smallest-perimeter polygon equals the intersection of all convex sets containing $P$}: We have earlier shown this to be the case, albeit we didn't specify the set $P$ to be the result of the intersection of multiple convex sets. However this is the exact same proof.
		      \item \textbf{Proof that the polygon is unique}: The resulting polygon is one which fulfills the following conditions:
		            \begin{enumerate}
			            \item It is smallest-perimeter
			            \item It is convex
		            \end{enumerate}
		            There is only one polygon which fulfills these (i.e., the convex hull) therefore making it unique.
	      \end{itemize}
\end{itemize}

\noindent
\textbf{1.3}\\\\
\noindent
Let $E$ be an unsorted set of $n$ segments that are the edges of a convex polygon. Describe an $O(n \log n)$ algorithm that computes from $E$ a list containing all vertices of the polygon, sorted in clockwise order.

\begin{verbatim}
Algorithm FIND_POLYGON_VERTICES(E)
Input: An unsorted set E of n segments that form the edges of a convex polygon.
Output: A list containing the vertices of the polygon, sorted in clockwise order.

1. Create an empty dictionary `adjacency` to store the adjacency list for each
vertex.
2. for each edge (p, q) in E:
     - Add q to the adjacency list of p: adjacency[p].append(q)
     - Add p to the adjacency list of q: adjacency[q].append(p)

3. Pick an arbitrary vertex `start_vertex` (e.g., any key from the adjacency
dictionary).

4. Initialize an empty list `clockwise_vertices` to store the sorted vertices.
5. Set `current_vertex` to `start_vertex` and `previous_vertex` to None.

6. while there are still unexplored edges:
     a. Append `current_vertex` to `clockwise_vertices`.
     b. Find the next vertex:
        - Look at the neighbors of `current_vertex` in the adjacency list.
        - Exclude `previous_vertex` from consideration (to avoid moving backward).
        - If `previous_vertex` is None (starting case), just pick any adjacent
						vertex.
     c. Update `previous_vertex` to `current_vertex`.
     d. Update `current_vertex` to the next vertex.

7. Continue until `current_vertex` equals `start_vertex` (closing the loop).

8. Return `clockwise_vertices`.
\end{verbatim}

(Is something wrong here, is this not $O(n)$?)
\\\\
\noindent
\textbf{1.4.a}\\
\noindent
For the convex hull algorithm we have to be able to test whether a point $r$ lies left or right of the directed line through two points $p$ and $q$. Let $p = (p_{x}, p_{y}), q = (q_{x}, q_{y})$ and $r = (r_{x}, r_{y})$

Show that the sign of the determinant

\begin{equation}
	D = \left| \begin{array}{ccc}
		1 & p_x & p_y \\
		1 & q_x & q_y \\
		1 & r_x & r_y \\
	\end{array} \right|
\end{equation}

determines whether $r$ lies left or right of this line.

\noindent
\textbf{Answer}:
\noindent

The determinant of a $3 \times 3$ matrix is defined by:
\begin{equation}
	\begin{vmatrix}
		a & b & c \\
		d & e & f \\
		g & h & i
	\end{vmatrix}
	= aei + bfg + cdh - ceg - bdi - afh
\end{equation}


Thus we can find the determinant of the matrix defined earlier as the following:

\begin{equation}
	D = \left| \begin{array}{ccc}
		1 & p_x & p_y \\
		1 & q_x & q_y \\
		1 & r_x & r_y \\
	\end{array} \right| = 1 \cdot q_{x} \cdot r_{y} + p_{x} \cdot q_{y} \cdot 1 + p_{y} \cdot 1 \cdot r_{x} - p_{y} \cdot q_{x} \cdot 1 - p_{x} \cdot 1 \cdot r_{y} - 1 \cdot q_{y} \cdot r_{x}
\end{equation}

Of course we can simplify this a lot:
\begin{equation}
	q_{x} \cdot r_{y} + p_{x} \cdot q_{y} + p_{y} \cdot r_{x} - p_{y} \cdot q_{x} - p_{x} \cdot r_{y} - q_{y} \cdot r_{x}
\end{equation}

If we try plotting the number's of three points, where $r$ lies to the \textit{right}, we get the following:

$$p = (1,1)$$
$$q = (2,2)$$
$$r = (1,2)$$

We get $1$ which is a positive number, indicating that positive numbers means that the point lays on the left of the line.
On the other hand, if we change $r$ to be $(2, -2)$ we get $-4$ which is a negative number, indicating that it is indeed on the right side of the line. Now, if $r$ is a colinear point, e.g. with the coordinates $(1.5, 1.5)$ we get the result $0$ which we now know to be indicative of a colinear point (i.e., a point on the line).

We summarize:
\begin{itemize}
	\item $D > 0$: $r$ is \textbf{left} of the line.
	\item $D < 0$: $r$ is \textbf{right} of the line.
	\item $D = 0$: $r$ is \textbf{on} the line.
\end{itemize}

We get these results, as the determinant gives us values based on how the transformation either grows or shrinks the area of the triangle which we create. If the point lies to the right of the line, the area is shrunk, while if it's on the left of the line, the area has grown. While, if it's on the line, we're dealing with a straight line, which doesn't shrink nor grow.

\noindent
\textbf{1.7}\\
\noindent
Consider the following alternative approach to computing the convex hull of a set of points in the plane: We start with the rightmost point. This is the first point $p_{1}$ of the convex hull. Now imagine that we start with a vertical line and rotate it clockwise until it hits another point, $p_{2}$. This is the second point in the convex hull. We continue rotating the line but this time around $p_{2}$ until we hit a point $p_{3}$. In this way we continue until we reach $p_{1}$ again.

\begin{enumerate}
	\item[1.] Give pseudocode for this algorithm.
\end{enumerate}


\begin{algorithmic}
	\STATE Let $p$ be the rightmost point
	\WHILE{the current point is not the first point}
	\STATE The next point $q$ is the point to which all points lie to the right of $\overrightarrow{pq}$
	\STATE $\text{Next[p]} = q$
	\STATE $p = q$ for the next iteration
	\ENDWHILE
\end{algorithmic}

\begin{enumerate}
	\item[2.] Which degenerate cases can occur and how can we deal with them?
\end{enumerate}
\textbf{Colinear Points}\\
\noindent
\textit{Fix}: Modify 2.1 to be: "The next point $q$ is the point to which all points lie to the right of $\overrightarrow{pq}$, and if applicable in the middle of $\overline{pq}$."
I.e. we treat colinear points as being to the right\\
\noindent
\textbf{Two or more points share the same $x$-coordinate:}\\
\noindent
\textit{Fix}: Sort lexigraphically


\begin{enumerate}
	\item[3.] Prove that the algorithm correctly computes the convex hull.
\end{enumerate}

We prove by induction.

\textbf{Induction Hypothesis}: The convex hull in iteration $i$ is correct.

\textbf{Base Case $i = 1$}. Trivial case, the first point is constant, being the rightmost point.

\textbf{Inductive Step}: Assume we have a hull $\{p_{1}, \ldots, p_{i}\}$. Any line segment $\overline{p_{i-1}p_i}$ is part of the hull, since we iteratively go over every point, and check that all other points are to the right. If any one point is to the left, we choose that as the new point, meaning that it is impossible to have the wrong point.

\begin{enumerate}
	\item[4.] Prove that the algorithm can be implemented to run in time $O(n \cdot h)$, where $h$ is the complexity of the convex hull.
\end{enumerate}

We iteratively choose the new best point, looking over all $n$ points. We do this for the number of vertices in the convex hull, as we always find the next point. Therefore, for each point in the convex hull $h$, we use $n$ calculations, giving us $O(n \cdot h)$. In the worst case, the convex hull uses all points, giving us $n = h, O(n^2)$. This also means that the algorithm is \textbf{output-sensitive}.

\begin{enumerate}
	\item[5.] What problems might occur when we deal with inexact floating point arithmetic?
\end{enumerate}

We might end up with a point being calculated as being the next (i.e., all points are to the right) even though it is not, because of floating point arithmetic. We also have no chance of it getting fixed, as when we are done with the current point, we don't look back to fixing it.
\\\\
\noindent
\textbf{1.8}\\
\noindent


\noindent
\textbf{1.9}\\
\noindent
Given a set $S$ we find min, max and median. We can do this in $O(n)$ time, as each operation takes $O(n)$ time.

Now that we have these values, we traverse $S$. For each value $x \in S$, we compare $x$ with the median \textit{med}. Then, depending on the result:
\begin{itemize}
	\item If $x \le $ \textit{med}: Assign the coordinate $(x,\log x)$
	\item If $x > $ \textit{med}: Assign the coordinate $(\max - x + \min + 1, - \log (\max - x + \min))$.
\end{itemize}
First off: The coordinate associated with a value greater than \textit{med} always has a lower $x$-value than the median, and a lower $y$-value, as it is negative, which any number greater than \textit{med} cannot be.

This forms a kind of triangle (almost, if it wasn't for the logarithm, which is necessary as to not create any colinear points). This triangle will itself be the hull, and there will be no points within the triangle.

We run the \textsc{ConvexHull} algorithm on the coordinates, and in return get the edges sorted in order. If we assume we use the algorithm found in the book, we will get a correctly sorted output, in the form of edges, since that algorithm starts on the leftmost point, which we have put to be 1. Note that the last point will have the coordinates $(\min, -\log(\min))$, which will be computed after the first point.

When we have the result, we traverse the result (one last time), and we extract the x-coordinates, as well as checking if $x > $ \textit{med}, and if so, we do the constant time operation $x + max - min - 1$ to get the original $x$ value.

The running time is $O(n)$ (max, min) + $O(n)$ (median) + $O(n)$ (coordinate-creation) + $\Omega (n \log n)$ (\textsc{ConvexHull}) + $O(n)$ (x-coordinate extraction).

\section{Kapitel 2}
\noindent
\textbf{2.1}\\
\noindent
\textit{Let $S$ be a set of $n$ disjoint line segments whose upper endpoints lie on the line $y = 1$ and whose lower endpoints lie on the line $y = 0$. These segments partition the horizontal strip $[-\infty : \infty] \times [0 : 1]$ into $n+1$ regions. Give an $O(n \log n)$ time algorithm to build a binary search tree on the segments in $S$ such that the region containing a query points can be determined in $O(\log n)$ time. Also describe the query algorithm in detail.}


\begin{algorithm}
	\label{alg:balancedbinarytree}
	\caption{Balanced Median Insertion Algorithm}
	\begin{algorithmic}[1]
		\REQUIRE A list of line segments
		\STATE Initialize an empty list of tuples $\textit{tuples}$
		\FOR{each line segment $l_i$ in the list of line segments}
		\STATE Extract the $x$-coordinate from $l_i$
		\STATE Add tuple $(x, l_i)$ to $\textit{tuples}$
		\ENDFOR
		\STATE Sort $\textit{tuples}$ by the $x$-coordinate
		\STATE Call \textsc{Median} on the sorted list of tuples
	\end{algorithmic}
\end{algorithm}

\begin{algorithm}
	\label{alg:median}
	\caption{Median Function}
	\begin{algorithmic}[1]
		\REQUIRE A sorted list of tuples $L$
		\STATE Find the median of $L$
		\STATE Remove the median from $L$
		\STATE Split $L$ into two sublists: $L_{\textit{left}}$ and $L_{\textit{right}}$
		\STATE Insert the median into a balanced search tree
		\IF{length of $L$ is not zero}
		\STATE Call \textsc{Median} on $L_{\textit{left}}$
		\STATE Call \textsc{Median} on $L_{\textit{right}}$
		\ENDIF
	\end{algorithmic}
\end{algorithm}

In algorithm \ref{alg:balancedbinarytree} and \ref{alg:median}, you can see an algorithm which inserts the points in a balanced binary search tree in time $O(n \log n)$.
\begin{enumerate}
	\item Extraction: $O(n)$
	\item Sorting: $O(n \log n)$
	\item Median-finding: $O(n)$ \textbf{however}, $n$ shrinks by $2$, thus $O(n) + O(n/2) + O(n/2) + O(n/4) \cdots O(1) = O(n \log n)$. \textbf{Comment: The median can be found in constant time, as the input is sorted. Thus this is $O(n)$ rather than $O(n \log n)$.}
	\item Inserting into the tree: $O(\log n)$.
	\item In total: $O(n \log n)$.
\end{enumerate}

We will in words describe the algorithm used in order to determine which points is closest to the region: We will take an $x$-value and traverse the binary search tree, and return the closest $x$-value. As the binary search tree is balanced, it's height is $\log n$, the time to find a point is $O(\log n)$. We just return the second part of the tuple, i.e., the reference to the line segment.
\\\\
\noindent
\textbf{2.2}\\
\noindent
\textit{The intersection detection problem for a set $S$ of $n$ line segments is to determine whether there exists a pair of segments in $S$ that intersect. Give a plane sweep algorithm that solves the intersection detection problem in $O(n \log n)$ time.}

We reuse the algorithm found on page 25, \textsc{FindIntersections(S)}. However, we add a line to this specific algorithm, \textit{return False}. Moreover, we modify line 2 in the function \textsc{FindNewEvent} on page 27 to be \textit{return True}.
\\\\
\noindent
\textbf{2.3}\\
\noindent
\textit{Change the code of algorithm \textsc{FindIntersections} (and of the procedures that it calls) such that the working storage is $O(n)$ instead of $O(n+k)$.}

By page 29: ``There is a relatively simple way to do this: only store inteersection points of pairs of segments that are currently adjacent on the sweep line. [...] The modifications required in the algorithm is that the intersection point of two segments must be deleted when they stop being adjacent.''
\\\\
\noindent
\textbf{2.4}\\
\noindent
\textit{Let $S$ be a set of $n$ line segments in the plane that may (partly) overlap each other. For example, $S$ could contain the segments $(0,0)(1,0)$ and $(-1,0)(2,0)$. We want to compute all intersections in $S$. More precisely, we want to compute each proper intersection of two segments in $S$ (that is, each intersection of two non-parallel segments) and for each endpoint of a segment all segments containing the point. Adapt algorithm \textsc{FindIntersections} to this end.}

In essence, this question asks to make the algorithm not consider if a line segment lies inside another line segment. Using the same algorithm as presented in the book, when reporting an intersection we check whether the slopes are identical. If they are, we ignore them and do not report the intersection.
\\\\
\noindent
\textbf{2.14}\\
\noindent
\textit{Let $S$ be a set of $n$ disjoint line segments in the plane, and let $p$ be a point not on any of the line segments of $S$. We wish to determine all line segments of $S$ that $p$ can see, that is, all line segments of $S$ that contain some point $q$ so that the open segment $\overline{pq}$ doesn't intersect any line segment of $S$. Give an $O(n \log n)$ time algorithm for this problem that uses a rotating half-line with its endpoint at $p$.}

We set up a sweepline algorithm. We define the events and status as following:
\begin{itemize}
	\item \textit{Events}: Endpoints in order of angle from $p$.
	\item \textit{Status}: Segments intersecting the sweepline, in order closest to furthest.
\end{itemize}

We initialize the status going through all $n$ line segments, and if they intersect, then we insert into a tree $T$. We initialize the event by inserting according to actions. If we are at the start of a segment, we insert it into $T$, while if we are at the end, we delete it from $T$.

To determine in which order we should insert segments which intersect the sweepline, we use distance from $p$, which we relate to``closeness''.

\noindent
\textbf{Degeneracies/Special Cases}:
\begin{itemize}
	\item Multiple end points intersecting with the sweep line at the same angle. \textit{In this case we take the closest point}.
	\item If a segments endpoint and startpoint are on the same angle from $p$. \textit{We first define whether or not we actually see them, as they in mathematical terms have no width. The described algorithm should work with this special case already.}
\end{itemize}

\textbf{Comment}: In the project don't spend time implementing too hard data structures (such as red black trees). We don't require a balanced search tree.
\\\\
\noindent
\textbf{Ekstraopgave 1}\\
\noindent
\textit{Assume that we have n disjoint horizontal line segments and m disjoint vertical line segments. Give an $O((m+n)\log(m+n))$ algorithm for finding the number of intersection points among the horizontal and vertical lines. Note that intersections points should not be reported, merely counted.}

\textbf{Comment: What I Don't Want To See In An Assignment (Based On Reality):}
\begin{itemize}
	\item Think about Kim as a User. Wants to check quickly which ones work and which ones don't.
	\item Report should give a manual about input format, how to compile it, how to run it, give an example.
	\item Flexibility is important. If you decide to use floating points, and Kim writes ``1'', and it doesn't work, that's a problem.
	\item Output should be standardized in the format, but input should be flexible.
	\item No requirements to the user (e.g. sort the input yourself).
	\item Be succinct.
	\item Input of points should be something like ``42 17'' rather than ``(42, 17)'' or ``P(42, 17)'' etc. A line could be ``42 17 5 9'' and understood as x y x y.
	\item Graphical output should be delivered along with textual output.
	\item Fine to use many packages, but definitely not required.
	\item Graphical output can be pdf, png, etc, but should not be something like latex code without running pdflatex.
	\item Very important for it to be easy with text examples for Kim (such as a flexible input file).
	\item No interactions in the program (e.g. 'please write points: ').
	\item Options are OK, for example to use interactions.
\end{itemize}

\section{Kapitel 3}
\noindent
\textbf{3.1}\\
\noindent
\textit{Prove that any polygon admits a triangulation, even if it has holes. Can you say anything about the number of triangles in the triangulation?}\\
Following the proof laid out in the book, we continue getting triangles like normally. The holes will come in, when you cannot draw a line from $a$ to $b$, and thus you have to put a line on a vertex in the hole. Thus the proof works here as well. However, the induction part on getting the number of triangles does not work here. This is because the holes will introduce
\\\\
\noindent
\textbf{3.2}\\
\noindent
\textit{A rectilinear polygon is a simple polygon of which all edges are horizontal or vertical. Let P be a rectilinear polygon with n vertices. Give an example to show that $n/4$ cameras are sometimes necessary to guard it.}
\begin{figure}[!ht]
	\centering
	\resizebox{5cm}{!}{%
		\begin{circuitikz}
			\tikzstyle{every node}=[font=\LARGE]



			\draw [short] (4,11.25) -- (11.25,11.25);
			\draw [short] (11.25,11.25) -- (11.25,6.25);
			\draw [short] (11.25,6.25) -- (9.75,6.25);
			\draw [short] (9.75,6.25) -- (9.75,9.75);
			\draw [short] (9.75,9.75) -- (8.5,9.75);
			\draw [short] (8.5,9.75) -- (8.5,6.25);
			\draw [short] (8.5,6.25) -- (7,6.25);
			\draw [short] (7,6.25) -- (7,9.75);
			\draw [short] (7,9.75) -- (5.5,9.75);
			\draw [short] (5.5,9.75) -- (5.5,6.25);
			\draw [short] (5.5,6.25) -- (4,6.25);
			\draw [short] (4,6.25) -- (4,11.25);
			\draw  (10.5,6.75) ellipse (0.25cm and 0.25cm);
			\draw  (7.75,10.5) ellipse (0.25cm and 0.25cm);
			\draw  (4.75,6.75) ellipse (0.25cm and 0.25cm);
		\end{circuitikz}
	}%

	\label{fig:my_label}
\end{figure}

Here the circles represent cameras.
\\\\
\noindent
\textbf{3.3}\\
\noindent
\textit{Prove or disprove: The dual graph of the triangulation of a monotone polygon is always a chain, that is, any node in this graph has degree at most two.}\\
\begin{figure}[!ht]
	\centering
	\resizebox{5cm}{!}{%
		\begin{circuitikz}
			\tikzstyle{every node}=[font=\normalsize]


			\draw [short] (2.5,2.75) -- (6.25,2.75);
			\draw [short] (6.25,2.75) -- (6.25,6.5);
			\draw [short] (6.25,6.5) -- (10,6.5);
			\draw [short] (10,6.5) -- (10,10.25);
			\draw [short] (10,10.25) -- (6.25,10.25);
			\draw [short] (6.25,10.25) -- (6.25,14);
			\draw [short] (6.25,14) -- (2.5,14);
			\draw [dashed] (2.5,2.75) -- (4.5,2.75);
			\draw [short] (2.5,2.75) -- (4.5,2.75);
			\draw [short] (2.5,2.75) -- (2.5,6.5);
			\draw [short] (2.5,6.5) -- (-1.25,6.5);
			\draw [short] (-1.25,6.5) -- (-1.25,10.25);
			\draw [short] (-1.25,10.25) -- (2.5,10.25);
			\draw [short] (2.5,10.25) -- (2.5,14);
			\draw [dashed] (-1.25,6.5) -- (2.5,10.25);
			\draw [dashed] (2.5,6.5) -- (2.5,10.25);
			\draw [dashed] (2.5,10.25) -- (6.25,14);
			\draw [dashed] (2.5,6.5) -- (6.25,10.25);
			\draw [dashed] (2.5,10.25) -- (6.25,10.25);
			\draw [dashed] (2.5,6.5) -- (6.25,6.5);
			\draw [dashed] (6.25,6.5) -- (6.25,10.25);
			\draw [dashed] (6.25,6.5) -- (10,10.25);
			\draw [dashed] (2.5,2.75) -- (6.25,6.5);
			\draw  (5.25,4) circle (0.25cm);
			\draw  (3.5,5.5) circle (0.25cm);
			\draw  (8.75,7.75) circle (0.25cm);
			\draw  (7.25,9.25) circle (0.25cm);
			\draw  (5.25,8) circle (0.25cm);
			\draw  (3.75,9.25) circle (0.25cm);
			\draw  (5,11.75) circle (0.25cm);
			\draw  (3.75,13.25) circle (0.25cm);
			\draw  (1.75,8) circle (0.25cm);
			\draw  (0.25,9.25) circle (0.25cm);
			\begin{scope}[rotate around={116.5:(5.25,4)}]
				\draw[domain=5.25:7.5,samples=100,smooth] plot (\x,{1*sin(1*\x r -5.25 r ) +4});
			\end{scope}
			\begin{scope}[rotate around={48.25:(3.5,5.5)}]
				\draw[domain=3.5:6.5,samples=100,smooth] plot (\x,{1*sin(1*\x r -3.5 r ) +5.5});
			\end{scope}
			\begin{scope}[rotate around={18.5:(5.25,8)}]
				\draw[domain=5.25:7.5,samples=100,smooth] plot (\x,{1*sin(1*\x r -5.25 r ) +8});
			\end{scope}
			\begin{scope}[rotate around={-66.75:(7.25,9.25)}]
				\draw[domain=7.25:9.25,samples=100,smooth] plot (\x,{1*sin(1*\x r -7.25 r ) +9.25});
			\end{scope}
			\begin{scope}[rotate around={113.25:(5.25,8)}]
				\draw[domain=5.25:7.25,samples=100,smooth] plot (\x,{1*sin(1*\x r -5.25 r ) +8});
			\end{scope}
			\begin{scope}[rotate around={56.25:(3.75,9.25)}]
				\draw[domain=3.75:6.5,samples=100,smooth] plot (\x,{1*sin(1*\x r -3.75 r ) +9.25});
			\end{scope}
			\begin{scope}[rotate around={106:(5,11.75)}]
				\draw[domain=5:6.75,samples=100,smooth] plot (\x,{1*sin(1*\x r -5 r ) +11.75});
			\end{scope}
			\begin{scope}[rotate around={18.5:(1.75,8)}]
				\draw[domain=1.75:4,samples=100,smooth] plot (\x,{1*sin(1*\x r -1.75 r ) +8});
			\end{scope}
			\begin{scope}[rotate around={119.75:(2,8)}]
				\draw[domain=2:4,samples=100,smooth] plot (\x,{1*sin(1*\x r -2 r ) +8});
			\end{scope}
		\end{circuitikz}
	}%

	\label{fig:my_label}
\end{figure}
\\\\
\noindent
\textbf{3.6}\\
\noindent
\textit{Give an algorithm that computes in O(n log n) time a diagonal that splits a simple polygon with n vertices into two simple polygons each with at most $\lfloor 2n/3 + 2 \rfloor$ vertices. Hint: Use the dual graph of a triangulation.}
\\\\
\noindent
\textbf{3.11}\\
\noindent
\textit{Give an efficient algorithm to determine whether a polygon P with n vertices is monotone with respect to some line, not necessarily a horizontal or vertical one.}\\
We make a sweep line algorithm with the degree that we want, and if it at any point shows non-connectedness, we return False.
\\\\
\noindent
\textbf{3.13}\\
\noindent
\textit{The stabbing number of a triangulated simple polygon P is the maximum
	number of diagonals intersected by any line segment interior to P. Give
	an algorithm that computes a triangulation of a convex polygon that has
	stabbing number O(log n).}
\\\\
\noindent
\textbf{Kim's Opgave}\\
\noindent
\textit{Will algorithm MAKEMONOTONE divide a simple polygon up into the smallest number of y-monotone polygons possible? Prove that it does or find an example where it does not. }


\section{Chapter 4}%
\label{sec:label}

In this chapter we studied the casting problem for molds of one piece. A
sphere cannot be manufactured in this manner, but it can be manufactured
if we use a two-piece mold. Give an example of an object that cannot be
manufactured with a two-piece mold, but that can be manufactured with
a three-piece mold.


\section{Chapter 5}%
\label{sec:label}
\noindent
\textbf{5.1}\\
\noindent
\textit{In the proof of the query time of the kd-tree we found the following recurrence:}

\begin{equation}
	Q(n) = \begin{cases}
		O(1),      & \text{ if } n = 1, \\
		2+2Q(n/4), & \text{ if } n > 1
	\end{cases}
\end{equation}

\textit{	Prove that this reccurence solves to $Q(n) = O(\sqrt{n})$. Also show that $\Omega (\sqrt{n})$ is a lower bound for querying in a kd-tree by defining a set of n points and a query rectangle appropriately.}

You can use the master theorem, which gets this result.

\begin{equation*}
	Q(n) = 2+2Q(\frac{n}{4}) = 2 + 2 \cdot (2 + 2Q(\frac{n}{16})) = 2 + 2^{2} + 2^{2} \cdot Q(\frac{n}{16}) = 2 + ^{2} + 2^{3} + 2^{3} \cdot Q(\frac{n}{64})
\end{equation*}

In the end this becomes

\begin{equation*}
	2 + 2^{2} + 2^{3} + \cdots + 2^{\log_{4}n} + 2^{\log_{4}n} \cdot Q(1)
\end{equation*}

If we look at each (led?), the further from the end you go, the more it halves. Therefore, The sum up till $2^{\log_{4}n} \le 2^{\log_{4}n} \le 3 \cdot 2^{\log_{4}n} = n^{\log_{4}2}$.

We get the last equality by putting $\log_{4}$ on each side such that $\log_{4}n \cdot \log_{4}2 = \log_{4}2 \cdot \log_{4}n$. The number you need in order to get $\log_{4} x = 2$ is $\frac{1}{2}$, therefore $n^{\log_{4}2} = n^{\frac{1}{2}} = \sqrt{n}$.

The recursion formula $Q(n)$ is not only an upper-bound but also a lower-bound.
\\\\
\noindent
\textbf{5.3}\\
\noindent
\textit{In Section 5.2 it was indicated that kd-trees can also be used to store sets of points in higher-dimensional space. Let P be a set of n points in d-dimensional space. In parts a. and b. you may consider d to be a constant.}

\begin{enumerate}
	\item[a) ] Describe an algorithm to construct a d-dimensional kd-tree for the points in P. Prove that the tree uses linear storage and that your algorithm takes $O(n \log n)$ time.
\end{enumerate}
We use the same algorithm as for 2 dimensions, but instead of half, we do $\mod d$.

We will have $d$ lists (copies), which we will need to sort by $x, y, z, \ldots, d$. For each of the $d$ lists we will check ``is this higher or lower than this element?''

The time for this algorithm can be found by this equation:
\begin{equation*}
	O(d \cdot n \log n) + (T(n) = O(d \cdot n) + 2 \cdot T(\frac{n}{2}) \in O(d \cdot n \log n))
\end{equation*}



\begin{enumerate}
	\item[b) ] Describe the query algorithm for performing a d-dimensional range query. Prove that the query time is bounded by $O(n^{1 - 1/d} + k)$.
\end{enumerate}


The time to search can be described by the formula:

\begin{equation*}
	Q(n) = c + 2^{d-1} \cdot Q \left( \frac{n}{2^{d}} \right)
\end{equation*}

\begin{equation*}
	(2^{d-1})^{\log_{2^{d}}n} = n^{\log_{2^{d}} 2^{d-1}}
\end{equation*}

Let $\log_{2^{d}} 2^{d-1} = x$. Then $2^{d\cdot x} = (2^{d})^{x} = 2^{d-1}$.
$d \cdot x = d-{1}$
$x = \frac{d-1}{d} = 1 - \frac{1}{d}$
$n^{1- \frac{1}{d}}$.


\begin{enumerate}
	\item[c) ] Show that the dependence on d in the amount of storage is linear, that is, show that the amount of storage is O(dn) if we do not consider d to be a constant. Give the dependence on d of the construction time and the query time as well.
\end{enumerate}

\\\\
\noindent
\textbf{5.5 (ab)}\\
\noindent
\textit{Algorithm SEARCH K-D TREE can also be used when querying with other ranges than rectangles. For example, a query is answered correctly if the range is a triangle.}

\begin{enumerate}
	\item Show that the query time for range queries with triangles is linear in the worst case, even if no answers are reported at all. Hint: Choose all points to be stored in the kd-tree on the line y = x.
\end{enumerate}
It's impossible to get an output-sensitive algorithm with this. Instead we need $O(n)$ in the case where we want the uppermost point where the points are on the plane with $x=y$.
\begin{enumerate}
	\item Suppose that a data structure is needed that can answer triangular range queries, but only for triangles whose edges are horizontal, vertical, or have slope +1 or −1. Develop a linear size data structure that answers such range queries in O(n3/4 + k) time, where k is the number of points reported. Hint: Choose 4 coordinate axes in the plane and use a 4-dimensional kd-tree.
\end{enumerate}
\\\\
\noindent
\textbf{5.8}\\
\noindent
\textit{Theorem 5.8 showed that a range tree on a set of n points in the plane requires O(n log n) storage. One could bring down the storage requirements by storing associated structures only with a subset of the nodes in the main tree.
\begin{enumerate}
	\item Suppose that only the nodes with depth 0, 2, 4, 6, . . . have an associated structure. Show how the query algorithm can be adapted to answer queries correctly.
\end{enumerate}

\begin{enumerate}
	\item Analyze the storage requirements and query time of such a data struc- ture.
\end{enumerate}
\begin{enumerate}
	\item Suppose that only the nodes with depth 0, 1j log n , 2j log n , . . . have an associated structure, where j ⩾ 2 is a constant. Analyze the storage requirements and query time of this data structure. Express the bounds in n and j.
\end{enumerate}
\\\\
\noindent
\textbf{5.10}\\
\noindent
\textit{In some applications one is interested only in the number of points that lie in a range rather than in reporting all of them. Such queries are often referred to as range counting queries. In this case one would like to avoid having an additive term of O(k) in the query time.}
\begin{enumerate}
	\item Describe how a 1-dimensional range tree can be adapted such that a range counting query can be performed in O(log n) time. Prove the query time bound.
	\item Using the solution to the 1-dimensional problem, describe how d- dimensional range counting queries can be answered in O(logd n) time. Prove the query time.
	\item Describe how fractional cascading can be used to improve the running time with a factor of O(log n) for 2- and higher-dimensional range counting queries.
\end{enumerate}
\\\\
\noindent
\textbf{5.13}\\
\noindent
\textit{In many application one wants to do range searching among objects other than points.}
\begin{enumerate}
	\item Let S be a set of n axis-parallel rectangles in the plane. We want to be able to report all rectangles in S that are completely contained in a query rectangle [x : x ] × [y : y ]. Describe a data structure for this problem that uses O(n log3 n) storage and has O(log4 n + k) query time, where k is the number of reported answers. Hint: Transform the problem to an orthogonal range searching problem in some higher- dimensional space.
	\item Let P consist of a set of n polygons in the plane. Again describe a data structure that uses O(n log3 n) storage and has O(log4 n + k) query time to report all polygons completely contained in the query rectangle, where k is the number of reported answers.
	\item Improve the query time of your solutions (both for a. and b.) to O(log3 n + k).
\end{enumerate}

\\\\
\noindent
\textbf{Kim's Opgave}\\
\noindent
\textit{Devise a method for building a red-black tree in linear time from a sorted sequence of n different keys. The tree should be external (also called leaf-oriented), i.e., all keys should be placed in the leaves and internal nodes hold copies of the keys directing searches to the correct locations. One potential difficulty in this problem is that not all trees have a form such that a red-black tree can be created by coloring nodes red or black.}

%%% Local Variables:
%%% mode: latex
%%% TeX-engine: luatex
%%% TeX-command-extra-options: "-shell-escape"
%%% TeX-master: "main"
%%% End:
