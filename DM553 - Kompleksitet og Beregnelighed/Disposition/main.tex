\documentclass{beamer}

\title{DM553 Emner}
\subtitle{Til mundtlig præsentation af hele pensum.}

\author{Kevin Vinther}
\institute{IMADA}
\date{6. Maj - 14 Juni (?)}


% Theme setting
\usetheme{default}
\usecolortheme{dove}

\usepackage{blindtext}
\usepackage{hyperref}
\usepackage[T1]{fontenc} %thanks's daleif
\usepackage[utf8]{inputenc}
\usepackage{amsthm}

% Sæt sproget til dansk
\usepackage{polyglossia}
\setdefaultlanguage{danish}
\setotherlanguages{english}



% Fonts and layout
\usefonttheme{professionalfonts}
\setbeamerfont{frametitle}{series=\bfseries}
\setbeamertemplate{navigation symbols}{}
\setbeamertemplate{footline}[frame number]

\usepackage{tikz} % Til grafer
\usetikzlibrary{automata} % Til automatgrafer
\usetikzlibrary{positioning} % Positionering af knuder
\usetikzlibrary{arrows.meta, arrows} % arrows til at ændre kanter
\tikzset{
	node distance=2.5cm, % minimumsdistancen mellem to knuderj
	every state/.style={
			semithick,
			fill=gray!10
		},
	initial text={}, % Ingen tekst, bare kanter
	double distance=2pt, % Udseende på accept states
	every edge/.style={ % Egenskaber for hver transition
			draw,
			>=stealth', % bold arrowheads
			auto,
			semithick
		}
}
% Tak til: https://www3.nd.edu/~dchiang/teaching/theory/2018/www/tikz_tutorial.pdf

\begin{document}

%%% Local variables:
%%% Mode: latex
%%% Tex-engine: xetex
%%% Tex-command-extra-options: "-shell-escape"
%%% Tex-master: "main"
%%% End:



\section*{Pumping Lemmas For Regular And Context-Free Languages}

\begin{itemize}
	\item Regulære Sprog
	\item Nonregulæritet
	\item $\{0^{n}1^{n} \mid n \ge 0\}$
	      \begin{itemize}
		      \item Pumpelemmaet for regulære sprog
		            \begin{itemize}
			            \item $xy^{i}z$
			            \item $y \ne \varepsilon$
			            \item $|xy| \le p$
		            \end{itemize}
		      \item Bevis
		            \begin{itemize}
			            \item $p = $antal tilstande
			            \item Hvis $y = \varepsilon$ så er sætningen trivielt sand, da $\varepsilon^{i} = \varepsilon \forall i$
			            \item En tilstand må gentages
			                  \begin{itemize}
				                  \item Fordi vi tager den første gentagne tilstand kan vi sige at $|xy| \le p$
			                  \end{itemize}
		            \end{itemize}
	      \end{itemize}
	\item Kontekstfrie Sprog
	\item Nonkontekstfrie Sprog
	\item $\{a^{n}b^{n}c^{n} \mid n \ge 0\}$
	      \begin{itemize}
		      \item Pumpelemmaet for kontekstfrie sprog
		            \begin{itemize}
			            \item $uv^{i}xy^{i}z$
			            \item $y \ne \varepsilon$
			            \item $|vxy| \le p$
		            \end{itemize}
		      \item Bevis
		            \begin{itemize}
			            \item $p = 2^{|V|+1}$
			            \item For en streng $|w| \ge p$ vil parsetræet have $\ge 2^{|V|+1}$ blade.
			            \item Højden af træet er mindst $|V|+2$ fordi en ekstra til at gå fra variabel til terminal.
			            \item Én af dise $|V|+2$ variabler må være gentagen.
			            \item Betingelser:
			                  \begin{itemize}
				                  \item 1. gælder da den gentagne variabel kan gentages igen og igen, og man kan også bare gå videre til at udlede $x$ fremfor $v$ og $y$.
				                  \item 2. gælder da sætningen er trivielt sand ellers
				                  \item $|vxy| \le p$ gælder, da vi ellers skulle have mere end $2^{|V|+1}$ blade.
			                  \end{itemize}
		            \end{itemize}
	      \end{itemize}
\end{itemize}

\section*{Pushdown Automata And Context-Free Languages}

\begin{itemize}
	\item Introduktion
	      \begin{itemize}
		      \item Forklar hvad kontekstfrie sprog er, CFG og PDA
	      \end{itemize}
	\item Kontekstfrie Sprog
	      \begin{itemize}
		      \item CFG - 4 Tuple ($V, \Sigma, R, S$)
		      \item Chomsky Normal Form, $2|w|-1$
		      \item Pushdown Automat - som en NFA (skal være nondeterministisk!) men med en stak, man kan nemt se hvordan $a^{n}b^{n}$ kan afgøres her.
	      \end{itemize}
	\item Ækvivalens mellem PDA og CFG
	\item Fra CFG til PDA:
	      \begin{itemize}
		      \item Start med pushing af start-symbolet
		      \item Venstremest afledning
		      \item Accep tinput ved at tjekke terminalerne
	      \end{itemize}
	\item Fra PDA til CFG:
	      \begin{itemize}
		      \item Skridt 1:
		      \item Forklar først om den begrænset PDA
		      \item 1. Én accept tilstand
		      \item 2. Der skal enten pushes eller poppes
		      \item 3. Stakken tømmes altid før accept
		      \item Vis at man sagtens kan gøre dette

		      \item Skridt 2:
		      \item Hvis den går fra $\mathdollar$ i stakken til $\mathdollar$ i stakken på et læst $w$, vil den gøre det samme med niveau $\ell$.
		      \item Skridt 3:
		      \item \(\forall p, q \in Q\) vil der være en variabel $A_{pq}$ som vil generere alle strenge $w \in \Sigma^{*}$ hvor stakken ikke går ned
		      \item Der er to scenarier: enten pusher den et symbol der først poppes til sidst, in that case: $aA_{rs}b$
		      \item Ellers kommer den ned på et andet tidpsunkt, in that case: $A_{pr}A_{sq}$.
		      \item Grammatikken er så som følger: $V = \{A_{pq} \mid p, q \in Q\}$
		      \item $S = A_{q_{0}q_{accept}}$
		      \item Regler: $\forall p, q, r, s \in Q, \forall t \in \Gamma, \forall a, b \in \Sigma_{\varepsilon}$ hvor den læser et a, pusher et t laver alt muligt, så læser et b og popper det t laver vi reglen $A_{pq} \rightarrow aA_{rs}b$ til $R$
		      \item $\forall p,q,r \in Q$ lad $A_{pq} \rightarrow A_{pr}A_{rq}$
		      \item $\forall p \in Q $ lad $A_{pp} \rightarrow \varepsilon$.
	      \end{itemize}
\end{itemize}

\section*{Turing Maskiner}
\begin{itemize}
	\item Kort introduktion til hvad en Turingmaskine er: som en DFA med ubegrænset bånd
	\item En 7-tuple: $(\Sigma, \Gamma, Q, q_{0}, q_{accept}, q_{reject}, \delta)$
	\item Grundlæggende om hvordan mskinen læser, skriver og bevæger åndet
	\item \textit{Konfigurationer}
	      \begin{itemize}
		      \item Hvad er en konfiguration
		      \item Hvad vil det sige at en konfiguration giver en andnen? $C_{1} \rightarrow C_2$
		            \begin{itemize}
			            \item At den kan gå fra $C_{1}$ til $C_{2}$ på ét skridt
		            \end{itemize}
		      \item Specielle konfigurationer: start, accept, afvisning
	      \end{itemize}
	\item \textit{Nondeterministiske Turingmaskiner}
	      \begin{itemize}
		      \item Forskelle
		            \begin{itemize}
			            \item Den kan gætte sig frem, og kan derfor have $B = |Q| \cdot |\Gamma| \cdot 3$ overføringer, fremfor kun én.
			            \item God til for eksempel at finde ud af primtal, da den kan gætte på tal der skal kunne ganges i.
		            \end{itemize}
		      \item Konvertering til deterministisk
		            \begin{itemize}
			            \item Lad hver overføring være $B$ eller $0$, og repræsenter valget som en streng i base $B$.
			            \item $283$ vil sige at det første skridt har taget anden overføring, andet skridt har taget ottende overføring og tredje skridt har taget tredje overføring.
			            \item Den deterministiske Turingmaskine har så 3 bånd: et bånd med input, et bånd der ændrer i forhold til hvad NDTM'en ville gøre, og et bånd til hvor vi er med strengen.
			            \item Vi bruger breadth first search, i en shortlex metode.
			            \item $B + B^2 + B^{3} + B^{4} \cdots B^{r}$: \textit{Eksponentielt}
		            \end{itemize}
	      \end{itemize}
	\item \textit{Multibånds Turingmaskiner}
	      \begin{itemize}
		      \item Mere end ét bånd, specifikt $k$ som ikke kan ændre sig i køretiden
		      \item Hvert bånd har hvert sit hoved, og derfor er overføringsfunktionen også ændrende i hovederne på hvert bånd
		      \item Konvertering er relativt simpel.
		      \item Et bånd der simulerer 3: \texttt{\#aaaaaå\#aaaåaaa\#aåaaa}
		      \item Nyt symbol for hver allerede eksisterende symbol, til at vise at det er der hovedet befinder sig
		      \item Hvis den vil til højre for båndet skal man lave et rightshift
		      \item Vi implementere ved at bruge tilstande som hukommelse: for hver tilstand $q_{i}$ har vi to tilstande i enkeltbåndsversionen:
		            \begin{itemize}
			            \item $q^{i}(a_{1}, a_{2}, \ldots, a_{k})$ hvor $a_{i} \in \Gamma \cup \{-\}$
			                  \begin{itemize}
				                  \item Når vi er i tilstanden $q^{i}_{(a_{1}, \ldots a_{i}, -, \ldots, -)}$ har vi set hvad der er under hovederne på de første $i$ bånd.
			                  \end{itemize}
			            \item $p^{i}(\delta_{1}, \delta_{2}, \ldots, \delta_{k}, b_{1}, b_{2}, \ldots, b_{k}, \gamma_{1}, \gamma_{2}, \ldots, \gamma_{k})$ hvor $\delta_{i} \in \Gamma \cup \{-\}$, $b_{i} \in \Gamma$, $\gamma_{i} \in \{R,L,S\}$
			                  \begin{itemize}
				                  \item I tilstand $p^{i}(b_{1}, \ldots, b_{i}, -, \ldots, -, b_{1}, b_{2}, \ldots, b_{k}, \gamma_{1}, \ldots, \gamma_{k})$ har vi ændret de første $i$ bånd, og flyttet hovedet.
			                  \end{itemize}
			            \item Mange tilstande, men polynomielt mange.
		            \end{itemize}
	      \end{itemize}
\end{itemize}

\newpage

\section*{Afgørlighed}

\begin{itemize}
	\item Hvad er afgørlighed?
	\item Hvad er uafgørlighed?
	\item Eksempler på afgørlige problemer, i.e., problemer hvor der findes en afgører:
	      \begin{itemize}
		      \item $A_{DFA}$ -> Trivielt
		      \item $A_{NFA}$ -> Konvertering, derefter kør DFA
		      \item $A_{REX}$ -> Konvertering, derefter kør NFA
		      \item $A_{CFG}$ -> konvertering til CNF, kør alle afledening $2|w|-1$
		      \item $E_{CFG}$ -> Markér terminaler
	      \end{itemize}
	\item Uafgørlige:
	      \begin{itemize}
		      \item $A_{TM}$, givet en TM $M$, accepterer den $w$?
		      \item Bemærk: Den er genkendelig, bare brug $U$.
		      \item Antag $H$ er en Turingmaskine der afgører $A_{TGM}$
		      \item Der bliver lavet en ny, $D$ som bruger $H$, men tager svaret, og gør det omvendte. I.e., hvis den accepterer, så afvis og omvendt, og den tager en Turingmaskine som input både til strengen der bliver accepteret, og maskinen.
		      \item Hvad sker der så hvis $D$ kører på sig selv? Så ville den acceptere, hvis og kun hvis den afviser og omvendt. Dette er et modstrid, og er umuligt.
	      \end{itemize}
	\item Standseproblemet:
	      \begin{itemize}
		      \item Standseproblemet siger ``Givet en Turingmaskine og en streng, standser maskinen på denne streng?'' Hvis ja, så accepter, ellers afvis.
		      \item Vi antager at $R$ afgører standseproblemet, og bruger den i en ny Turingmaskine, $M_{R}$ som først tager Turingmaskinen, og outputter en ny Turingmaskine som løkker på alt input den ikke ville acceptere.
		      \item Vi giver denne nye Turingmaskine, sammen med inputtet $w$ til $R$, som accepterer $\iff$ w \(\in L(m)\), og afviser $\iff w \notin L(m)$.
	      \end{itemize}
\end{itemize}

\newpage

\section*{NP-Komplethedsbeviser - Eksempler}

\begin{itemize}
	\item $P$ = Klassen af problemer der kan løses af en deterministisk maskine i polynomiel tid
	\item $NP$ = Klassen af problemer hvis løsning kan verificeres af en deterministisk maskine i polynomiel tid
	\item $NP$-Komplet: Klassen af problemer som er i NP, men hvor alle andre problemer også kan reduceres til dette (gjort muligt af Cook-Levin sætningen).
	\item Polynomiel Reduktion:
	      \begin{itemize}
		      \item Funktion der går fra A til B:  $f : \Sigma^{*} \rightarrow \Sigma^{*}$ således at $x \in A \iff f(x) \in B$
		      \item Det er polynomielt, i.e., der findes en konstant hvorledes $f(x)$ kan beregnes i tid $O(|x|^{k})$
		      \item Vi skriver $A \le_{p} B$ hvis $A$ reduceres til $B$.
		      \item Dermed hvis $A \le_{P} B$ og $B \in P$ så er $A \in P$, da reduceringen kun tager polynomiel tid, og vi kan køre $B$ i polynomiel tid.
	      \end{itemize}
	\item Hvordan man beviser NP-Komplet
	      \begin{itemize}
		      \item Problemet skal være i $NP$
		      \item Et andet NP-Komplet problem skal kunne reducere til det
	      \end{itemize}
	\item Eksempel: SAT $\le_{p}$ 3-SAT
	      \begin{itemize}
		      \item Alle klausuler skal have størrelse $= 3$
		      \item Reducering fra $|C| \ge 4, |C| = 2, |C| = 1$
	      \end{itemize}
	\item Eksempel 3-SAT $\le_{p}$ CLIQUE
	      \begin{itemize}
		      \item Klausuler må ikke have kanter imellem sig, $X$ og $\overline{X}$ må ikke have kanter mellem sig. Hvis der eksisterer en klikke, så giv literals værdier så de evaluerer sig til sand
	      \end{itemize}
	\item Eksempel CLIQUE $\le_{P}$ Independent-Set
	      \begin{itemize}
		      \item Hvis man kan finde et independent set, tager man bare komplementet af knuderne for at få klikken
	      \end{itemize}
	\item Eksempel Independent Set $\le_{P}$ vertex Cover
	      \begin{itemize}
		      \item $X$ er Independent $\iff$ $V \setminus X$ er et Vertex Cover
	      \end{itemize}
\end{itemize}

\section*{Cook-Levin}

\begin{itemize}

	\item P er de problemer der kan løses deterministisk i polynomiel tid
	\item NP er de problemer hvis løsning kan verificeres i polynomiel tid
	\item Cook-Levin sætningen er basis for NP-komplethed, som altså siger at alle NP problemer kan reduceres til dette.
	\item Det vil sige, at hvis $SAT \in P$, så $P = NP$
	\item En tableau for $N$ er en gren af en nondeterministisk maskine på en streng $w$.
	\item Hvis $N$ er en NDTM der afgører $A$ i tid $n^{k}$ minus en konstant vil tableauen være af størrelse $n^{k}$ i begge sider, altså både rækker og kolonner.
	\item Hele første kolonne er fyldt med \# for at indikere start, samme med den sidste for at indikere slut.
	\item Hver række er en konfiguration.
	\item Første række er startkonfigurationen.
	\item En \textit{accepterende} tableu er en tableu hvor mindst én af rækkende svarer til en accepterende konfiguration af $N$ på $w$.
	\item Hvis en række er accepterende, lader vi de resterende rækker efter denne også være accepterende.
	\item Vores mål er at lave Satisfiability formlen der er sand hvis og kun hvis der er en accepterende tableau for $N$ på $w$: $\phi = \phi_{cell} \wedge \phi_{start} \wedge \phi_{accept} \wedge \phi_{bevæg}$
	\item $X_{i,j,a} = 1$ hvis på række $i$, kolonne $j$ er der symbolet $a$. Der må kun være ét symbol i hver ``celle'', så vi får:
	      \begin{equation*}
		      \phi_{cell} = \bigwedge_{i,j \in [n^{k}]} \left( \left(   \bigvee_{s\in C} x_{i,j,s} \right) \wedge \left( \bigwedge_{s,t \in C, s \ne t} (\overline{x_{i,j,s}} \vee \overline{x_{i,j,t}}) \right)\right)
	      \end{equation*}
	\item Vi vil gerne at startrækken indeholder input.
	      \begin{equation*}
		      \phi_{start} = x_{1,1,\#} \wedge x_{1,2,q_{0}} \wedge \left( \bigwedge_{j=1}^{n} x_{1,2+j,w_{j}}  \right) \wedge \left( \bigwedge_{j=n+3}^{t-1} x_{1,j,\sqcup} \right) \wedge x_{1,t,\#}
	      \end{equation*}
	\item Et eller andet sted skal der bare være en accepttilstand:
	      \begin{equation*}
		      \bigvee_{1 \le i,j \le t} x_{i,j,q_{accept}}
	      \end{equation*}

	\item Move er den sværeste. Vi har et \(3 \times 2\) vindue som vi skal ``køre rundt'' på hele tableuen, for at sikre at alle konfigurationer er lovlige i forhold til overføringsfunktionen.
	\item Vi kalder et vindue lovligt hvis de 3 nederste celler kan resultere fra de 3 øverste celler i et skridt

	      \[
		      \varphi_{move} = \bigwedge_{1 \leq i < t \atop 1 \leq j < t} (\text{(i, j) - window is legal})
	      \]

	      \[
		      \bigvee_{(a_1, a_2, a_3, a_4, a_5, a_6) \text{ is a legal window}} (x_{i, j-1, a_1} \land x_{i, j, a_2} \land x_{i, j+1, a_3} \land x_{i+1, j-1, a_4} \land x_{i+1, j, a_5} \land x_{i+1, j+1, a_6})
	      \]
	\item Der er højest $|C|^{6}$ lovlige vinduer.
	\item Dermed, hvis \(\phi\) er satisfiable, så accepterer $N$ $w$ hvis $w \in A$.
	\item Vis nu at reduktionen er polynomiel:
	      \begin{itemize}
		      \item Antallet af variabler er $n^{2k} \cdot |c| \in O(n^{2k})$
		      \item $|\phi_{start}| \in O(n^{k})$
		      \item $|\phi_{accept}| \in O(n^{2k})$
		      \item $|\phi_{cell}| \in O(n^{2k})$
		      \item $|\phi_{move}| \in O(n^{2k})$ da antallet af lovlige vinduer afhænger af $N$'s overføringstabel.
		      \item $O(\log n)$ til at håndtere indeks
		      \item I alt $|\phi| \in O(n^{2k} \log n)$ som er polynomielt i $n = |w|$.
	      \end{itemize}
\end{itemize}
\newpage

\section*{Approksimationsalgoritmer}
\begin{itemize}
	\item Introduktion
	      \begin{itemize}
		      \item Hvad er approksimationsalgoritmer?
		      \item Anderledes end afgørseslproblemer: I stedet for at være givet et $k$ forsøger den at finde noget nær-optimalt.
	      \end{itemize}
	\item Løsninger
	      \begin{itemize}
		      \item Vi kalder vores løsning $C$, og den optimale $C^{*}$
		      \item Approksimationsforholdet \(\rho(n)\) for et problem kan udregnes ved \[ \max \{ \frac{C}{C^{*}}, \frac{C^{*}}{C} \} \le \rho(n)\]
		      \item Dette afhænger af om det er et maksimerings eller minimeringsproblem.
		      \item Maksimering: $C \le C^{*}$, Minimering: $C \ge C^{*}$
	      \end{itemize}
	\item Eksempel 1: Vertex Cover
	      \begin{itemize}
		      \item 2-approksimationsalgoritme.
		      \item Simpel algoritme: Vælg tilfældigt en kant og fjern alle kanter incident til knuderne i kanten.
		      \item Bliv ved indtil der ikke er flere kanter
		      \item Kanterne giver en matching med $2k$ knuder, hvor $k$ er antal kanter
		      \item $|X^{*}|$ err det optimale vertex cover, og $|X^{*}| \ge k$, fordi der mindst må bruges $k$ knuder
		      \item Dermed, da vi har $2k$ er $\frac{|X|}{|X^{*}|} \le 2$, så det er en 2-approksimationsalgoritme
		      \item Algoritmen kører i polynomiel tid i antallet af kanter.
	      \end{itemize}
	\item Eksempel 2: TSP med $\Delta$-ulighed
	      \begin{itemize}
		      \item \(\Delta\)-uligheden siger at $c_{ij} \le c_{ik} + c_{kj}$ $\forall i,j,k \in V$, altså er det aldrig \textit{hurtigere} end at gå direkte fra $a$ til $b$ end at gå fra $a$ til $c$ til $b$.
		      \item Lav et træ, og så en hamiltoniansk kreds hvor $\frac{c(H)}{c(H^{*})} \le 2$, hvor $H^{*}$ er den optimale tur.
		      \item $c(T^{*}) \le c(H^{*})$: Hvis vi har en hamiltoniansk kreds, og fjerner en kant får vi et spanning træ. Dermed $c(H^{*}) \ge c(T') \ge c(T^{*})$
		      \item Vi laver en walk i Træet, og fjerner duplicates, så vi får en kreds.
		      \item $c(H) \le c(W)$ ud fra trekantsuligheden, hvor $w$ er walken.
		      \item Da walken er $2c(T^{*})$ da vi både går frem og tilbage, og $c(H) \le c(W)$ får vi $c(H) \le c(W) = 2c(T^{*}) \le 2c(H^{*})$
		      \item Dermed $\frac{c(H)}{c(H^{*})} \le 2$
	      \end{itemize}
	\item TSP  uden Trekantsulighed er ikke mulig som approksimationsalgoritme.
\end{itemize}

\section*{Informationsteoretiske Nedre Grænser}

\section*{Modstanderargumenter - Teknikker og Eksempler}

\section*{Algoritmer med faste parametre, parameteriseret kompleksitet og ekaskte eksponentielle algoritmer}


\newpage

\end{document}
%%% Local Variables:
%%% mode: latex
%%% TeX-engine: xetex
%%% TeX-command-extra-options: "-shell-escape"
%%% End:
