\chapter{Afgørlighed}

\section{Afgørlighed}%
\label{sec:afgørlighed}

\begin{note}[Kilder]
	Video 11
\end{note}



\section{Uafgørlighed}%
\label{sec:label}

\begin{note}[Kilder]
	Video 12
\end{note}

Vores mål til at starte med, er at bevise at $A_{TM}$ er uafgørligt. FØr vi gør det, kigger vi på teori omkring uendeligheder.

\begin{definition}[Tællelig mængde]
	En mængde $S$ er tællelig hvis:
	\begin{enumerate}
		\item $S$ er endelig, eller
		\item $\exists f : S \rightarrow \mathbb{N}$ hvor $f$ er en-til-en og onto.
	\end{enumerate}
\end{definition}

\begin{theorem}
	Mængden $B$ af uendelige binære strenge er overtællelig.
\end{theorem}

\begin{proof}
	Antag at $b_{1}, b_{2}, b_{3}, \ldots, b_{k}, b_{k+1}, \ldots$ er en liste af alle uendelige binære strenge.
	Vi definerer $b^{*}$ til at være en binær streng, som altid vil være anderledes end alle andre (og derfor ikke tillader one-to-one og onto). Vi definerer $b^{*}(i)$ til at være den $i$'e plads i $b^{*}$:
	\begin{equation}
		b^{*}(i) = \begin{cases}
			1 & \text{ hvis } b_{i}(i) = 0 \\
			0 & \text{ hvis } b_{i}(i) = 1 \\
		\end{cases}
	\end{equation}
	Ud fra denne definition vil $b^{*}$ \textbf{ikke} kunne være en del af $b_{1}, b_{2}, \ldots$, da den er forskellig fra alle strenge på en eller anden måde. Antag $b^{*} = b_{j}$, så $b^{*}(j) = 1 - b_{j}(j) \ne b_{j}(j)$. Altså $b^{*}$ på position $j$, men $b^{*}(j)$ er anderledes fra $b_{j}$.
\end{proof}

Hvert sprog over et alfabet $\Sigma$ er en delmængde af $P(\Sigma^{*})$ som er mængden af alle delmængder af $\Sigma^{*}$.  Vi ved at $\Sigma^{*}$ er tællelig, fordi vi kan ordne dem leksikografisk $w_{1}, w_{2}, \ldots$. $L$ over alfabetetet $\Sigma$ korresponderer one-to-one til en unik endelig binær streng $b_{L}$ hvor:
\begin{equation}
	b_{L}(i) = \begin{cases}
		1 & \iff w_{i} \in L    \\
		0 & \iff w_{i} \notin L
	\end{cases}
\end{equation}
Det vil altså sige at der er lige så mange sprog, som der er binære strenge, hvilket er overtælleligt.

\begin{corollary}
	Mængden af alle sprog over et ikke-trivielt (\( |\Sigma| \ge 1\)) alfabet \(\Sigma\) er overtælleligt.
\end{corollary}


\textbf{MANGLER: Bevis for at der er tælleligt mange turingmaskiner. omkring 15:00-22:00 i video 12}.

\begin{theorem}
	$A_{TM}$ er uafgørligt.
\end{theorem}

\begin{proof}
	Antag at $H$ er en Turingmaskine som afgører $A_{TM}$.

	Vi kan definere $H$ til at være følgende:
	\begin{equation}
		H = \begin{cases}
			\text{\textit{accept}} & \iff \langle M, w\rangle \in A_{TM}     \\
			\text{\textit{afvis}}  & \iff \langle M, w \rangle \notin A_{TM}
		\end{cases}
	\end{equation}

	Vi bruger $H$ som funktion i en ny Turingmaskine, $D$. $D$ tager kun et input, en kodning af en turingmaskine, i.e. $\langle M \rangle$. Denne kodning fodrer $D$ til $H$ som begge input, altså både som kodningen af turingmaskinen, og strengen den accepterer. I sin essens, spørger $D$ turingmaskinen $H$ om en Turingmaskine accepterer konfigurationen på en anden turingmaskine. Resultatet af dette omvendes så, så hvis $H$ accepterer, så \textit{afviser} $D$, og omvendt.

	Et problem kommer frem her, for hvad hvis vi kører $D$ på input $\langle D \rangle$? Så vil følgende ske:
	\begin{equation}
		\label{eq:wrongatm}
		D(\langle D \rangle ) = \begin{cases}
			\text{\textit{accept}} & \text{ hvis } \langle D \rangle \notin L(D) \\
			\text{\textit{afvis}}  & \text{ hvis } \langle D \rangle \in L(D)    \\
		\end{cases}
	\end{equation}
	Altså ud fra \eqref{eq:wrongatm} får vi at $\langle D \rangle \in L(D) \iff \langle D \rangle \notin L(D)$. Dermed kan $H$ ikke eksistere.
\end{proof}

Givet en matrix af Turingmaskiner, hvor de yderste rækker og kolonner er alle de tælleligt uendelige kodninger til Turingmaskiner. Antag at $\langle m_{j} \rangle$. Den første kolonne er Turingmaskiner, og den første række er Turingmaskiner, men som sprog.

Derfor, ved en celle i matrixen \(\langle m_{i} \rangle \langle m_{j} \rangle\)  hvor $m_{i}$ er fra første kolonne, og $m_{j}$ fra første række:
\begin{equation*}
	\langle m_{i} \rangle \langle m_{j} \rangle = \begin{cases}
		1 & \text{ hvis } \langle m_{j} \rangle \in L(m_{i}) \\
		0 & \text{ ellers}
	\end{cases}
\end{equation*}

Vi antager at $H$ eksisterer. Dermed må $D$ også eksistere, da den bare swapper \textit{accept} og \textit{afvis} tilstande. $D = \langle m_{i} \rangle $ for en $m_{i}$ i listen af alle Turingmaskiner. Men $D$ er uenig med $m_{i}$ på input $\langle m_{i} \rangle $ $\langle m_{i} \rangle \in L(m_{i}) \iff \langle m_{i} \rangle \notin  L(D) = \langle m_{i} \rangle $. Dermed er $D$ ikke i listen og så eksisterer den ikke. Så $A_{TM}$ er \textbf{ikke} aførlig.


\begin{theorem}
	\label{teo:lnotlrec}
	$L$ er Turing-afgørligt $\iff$ $L$ og $\overline{L}$ er Turing-genkendelige.
\end{theorem}

\begin{proof}
	\(\Rightarrow\) Lad $m^{*}$ afgøre $L$. $m^{*}$stopper altid og $L(m^{*}) = L$ så $m^{*}$ genkender $L$ og $\overline{m^{*}}$ genkender $\overline{L}$ ved at tage outputtet fra $m^{*}$ og gøre det omvendt, i.e. afvise hvis den accepterer, og acceptere hvis den afviser.\\
	\(\Leftarrow\) Lad $m_{L}, m_{\overline{L}}$ genkender hhv. $L$ og $\overline{L}$. Vi kan ikke lave en Turingmaskine der blot kører $m_{L}$ og tjekker om den accepterer eller ikke accepterer, da den ikke nødvendigvis er en afgører. Dermed, fremfor at lave en sådan Turingmaskine, laver vi en Turingmaskine $M$ der kører $m_{L}$ og $m_{\overline{L}}$ på samme tid på samme input $w$. Hvis $m_{L}$ accepterer, så accepterer $M$. Hvis $m_{\overline{L}}$ accepterer, så \textit{afviser} $M$. Omvendt, hvis $m_{L}$ afviser, så afviser $M$, og hvis $m_{\overline{L}}$ afviser, så \textit{accepterer} $M$.
\end{proof}

\begin{theorem}
	For hvert sprog $L$ over det universelle alfabet, gælder præcis en af følgende:
	\begin{enumerate}
		\item $L$ og $\overline{L}$ er afgørlige
		\item Ingen af $L$, $\overline{L}$ er genkendelige
		\item $L$ er genkendeligt, men $\overline{L}$ er ugenkendeligt, eller omvendt.
	\end{enumerate}

\end{theorem}

\begin{proof}
	\[
		\begin{array}{c|ccc}
			\overline{L} \setminus L & A & G & IG \\
			\hline
			A                        & * & - & -  \\
			G                        & - & - & *  \\
			IG                       & - & * & *  \\
		\end{array}
	\]

	\begin{align*}
		A  & = \text{afgørlig}                      \\
		G  & = \text{genkendelig men ikke afgørlig} \\
		IG & = \text{ikke genkendelig}              \\
		*  & = \text{mulig}                         \\
		-  & = \text{umulig}
	\end{align*}
\end{proof}

\begin{corollary}
	$\overline{A_{TM}}$ er ugenkendelig.
\end{corollary}
\begin{proof}
	$\overline{A_{TM}} = \{ \langle w' \rangle \mid \text{ 1. Ingen præfiks af } \langle w' \rangle \text{ koder en Turingmaskine eller}$
	$\text{2. For hver } \langle m \rangle \text{ hvor } \langle w' \rangle = \langle m \rangle \langle w \rangle \text{ har vi at } w \notin L(m)\}$

	Vi ved at $\overline{A_{TM}}$ er ugenkendelig, da, hvis den var genkendelig, så ville $A_{TM}$ være afgørlig (ud fra Sætning~\ref{teo:lnotlrec}).
\end{proof}

\section{Uafgørlige Problemer}%
\label{sec:label}

\begin{note}[Kilder]
	Video 13
\end{note}

Vi starter med at definere standseproblemet, hvor vi hovedsageligt kommer til at bruge det engelske navn: \textit{Halting Problem}.
\begin{equation*}
	HALT = \{ \langle m \rangle \langle w \rangle \mid m \text{ er en Turingmaskine og } m \text{ stopper på }w\}
\end{equation*}

\begin{theorem}
	\label{teo:halt}
	$HALT$ er uafgørligt.
\end{theorem}

\begin{proof}
	Antag at turingmaskinen $R$ afgører $HALT$. Vi viser nu at dette vil mene at $A_{TM}$ er afgørligt.

	Udover denne, så konstruerer vi Turingmaskinen $A$ som tager \(\langle m \rangle \) som input, og giver \(\langle m' \rangle \) som output. Hvis \(\langle m \rangle \) er en korrekt kodning af en Turingmaskine, er $L(m') = L(m)$ men hvor $L(m')$ løkker på alle stringe der ikke er en del af $L(m')$. Omvendt, hvis $\langle m \rangle$ ikke er en korrekt kodning, så $\langle m' \rangle = \langle m \rangle$.

	Vi konstruerer nu $m_{R}$ til at afgøre $A_{TM}$. Først giver vi inputtet \(\langle m \rangle \) til $A$ og \(\langle w \rangle \) til $R$, og så outputter vi \(\langle m' \rangle \) fra $A$ og giver videre til input i $R$. Hvis $R$ accepterer, så accepterer $m_{R}$ og omvendt. Dermed, hvis $w \in L(m')$ så accepterer vi, og hvis den løkker (som den kun kan gøre, hvis $w \notin L(m')$), så afviser vi. Dermed er $M_{R}$ en afgører for $A_{TM}$.
\end{proof}

Vi kigger nu på tomhedsproblemet:
\begin{equation}
	\label{eq:etm}
	E_{TM} = \{\langle m \rangle \mid m \text{ er en Turingmaskine og } L(M) = \emptyset\}
\end{equation}
\begin{theorem}
	$E_{TM}$ er uafgørligt.
\end{theorem}
\begin{proof}
	Antag at $S$ er en Turingmaskine som afgører $E_{TM}$. Vi bruger $S$ til at konstruere en ny Turingmaskine som vil afgøre $A_{TM}$.

	Derudover bruger vi Turingmaskinen $B$ som tager input \(\langle M, w \rangle \) og outputter $\langle M_{w} \rangle$. Den konstruerer $\langle M_{w} \rangle $ ved at først tjekke om \(\langle M \rangle \) er en korrekt kodning af en Turingmaskine. Hvis ikke, så er $\langle M_{w} \rangle $ (outputtet) en Turingmaskine, hvor $L(M_{w}) = \emptyset$. Hvis \(\langle m \rangle \) er en Turingmaskine, så er $M_{w}$ en turingmaskine med følgende sprog:
	\begin{equation}
		\label{eq:emptiness}
		L(M_{w}) = \begin{cases}
			\{w\}     & \text{ hvis } w \in L(m)    \\
			\emptyset & \text{ hvis } w \notin L(m)
		\end{cases}
	\end{equation}
	Altså vil $M_{w}$ kun acceptere $w$, hvis $w \in L(m)$ og ellers accepterer den det tomme sprog. Dermed vil $M_{w}$ blot være den følgende algoritme:
	\begin{enumerate}
		\item Tjek om input $x = w$
		\item Hvis $x \ne w$, så \textit{afvis}.
		\item Ellers simulér $M$ på $w$ og accepter hvis og kun hvis $M$ accepterer $w$.
	\end{enumerate}

	Vi kan nu konstruere $S^{*}$ til at afgøre $A_{TM}$. Først tager vi $\langle M , w\rangle $ som input til $B$, som så producerer $\langle M_{w} \rangle $. Vi fodrer så $\langle M_{w} \rangle $ til $S$, som accepterer hvis og kun hvis $w \notin L(M)$, og afviser ellers. $S^{*}$ tager så dette output, og bytter om på det, så en accept bliver afvist, og en afvisning bliver accepteret. Dermed har vi en afgører for $A_{TM}$.
\end{proof}

Vi kommer til at bruge mange maskiner med et sprog lignende den i \eqref{eq:emptiness} i fremtidige beviser om uafgørlighed. Det vigtige at forstå ved Turingmaskinen $M_{w}$ er at den \textit{ikke skal køres}. Vi bruger en anden Turingmaskine ved $M_{w}$, nemlig $R$ til at analysere den, og finde ud af noget om den, i dette tilfælde om dens sprog er tomt eller ej. Dette er et mønster vi kommer til at se mere af, og konkluderer i Rice's sætning.

Vi kan faktisk genkende $\overline{E_{TM}}$. Vi kan genkende $\overline{E_{TM}}$ på følgende måde:
\begin{enumerate}
	\item Tjek om \(\langle w \rangle \) koder en Turingmaskine, hvis ikke, \textit{accepter}  \(\langle w \rangle \)
	\item Lad $M$ være en Turingmaskine kodet af $\langle w \rangle $
	\item Simulér $M$ på strenge over $\Sigma^{*}$ i leksikografisk ordning parallelt:
	      \begin{enumerate}
		      \item For $i = 1, 2, \ldots$
		      \item Simuler $m$ på $i$ skridt på strengene $w_{1}, w_{2}, \ldots, w_{i}$ ifølge leksikografisk ordning.
		      \item Stop når en streng accepteres.
	      \end{enumerate}
\end{enumerate}

Bemærk at $\overline{E_{TM}}$, gennem den algoritme vi har konstrueret, \textbf{ikke} er afgørlig.

\begin{corollary}
	$E_{TM}$ er Turing-ugenkendeligt.
\end{corollary}

Corollarien kommer fra Sætning~\ref{teo:lnotlrec}.

Vi definerer nu to specielle Turingmaskiner, som vi kommer til at bruge i fremtidige beviser:
\begin{itemize}
	\item $M_{\emptyset}$: Går direkte til sin afvisningstilstand, uanset input. $L(M_{\emptyset}) = \emptyset$
	\item $M_{\Sigma^{*}}$: Går direkte til sin accepttilstand, uanset input. $L(M_{\Sigma^{*}}) = \Sigma^{*}$.
\end{itemize}

Vi vil nu kigge på lighedsproblemet.

\begin{equation*}
	EQ_{TM} = \{\langle m_{1}, m_{2} \rangle \mid m_{1} \text{ og } m_{2} \text{ er Turingmaskiner, hvor } L(m_{1}) = L(m_{2})\}
\end{equation*}

\begin{theorem}
	$EQ_{TM}$ er uafgørlig.
\end{theorem}

\begin{proof}
	Vi beviser ved at \textit{reducere} fra $E_{TM}$ \eqref{eq:etm} til $EQ_{TM}$.
	Antag at $m_{EQ}$ afgører $EQ_{TM}$, så kan vi lave en Turingmaskien $m^{*}$ som givet $M_{\emptyset}$ og en Turingmaskine af eget valg, accepterer hvis og kun hvis $M_{\emptyset}$ og Turingmaskinen af eget valg er ens. Dette er umuligt, da $m^{*}$ afgører $E_{TM}$.
\end{proof}

\section{Mapping Reducerbarhed}%
\label{sec:label}

Nu stiller vi spørgsmålet: Er det muligt at formalisere idéen om at \textit{reducere} fra et problem til et andet? Svaret er ja!

\begin{definition}
	$f : \Sigma^{*} \rightarrow \Sigma^{*}$ er \textit{beregnelig} hvis der eksisterer en Turingmaskine $M_{f}$ som starter med $w$ og slutter med $f(w)$ på sit bånd, i.e. $q_{0}w \stackrel{*}{\Rightarrow} q_{acc}f(w)$.
\end{definition}

Bemærk, at når en funktion er beregnelig, så stopper den \textbf{altid}. Et eksempel på en beregnelig funktion er den Turingmaskine $A$ fra beviset i Sætning~\ref{teo:halt}. Denne tager input en streng \(\langle w \rangle \) og:
\begin{enumerate}
	\item Tjekker om \(\langle w \rangle \) er en Turingmaskine beskrivelse
	      \begin{itemize}
		      \item Hvis ja: så ouputter $A$  \(\langle m' \rangle \)hvor \(\langle m' \rangle =  \langle w \rangle \), men hvor den løkker på alle strenge den \textit{ikke} accepterer.
		      \item Hvis nej: så outputter $A$ strengen \(\langle w \rangle \).
	      \end{itemize}
\end{enumerate}

Så $A$ udregner funktionen $f : \langle w \rangle  \longrightarrow \langle w' \rangle $ hvor $\langle w' \rangle $ er outputtet fra $A$.

\begin{definition}
	Lad $A,B$ være sprog. Vi siger at $A$ er \textit{mapping reducerbart} til $B$ hvis der eksisterer en beregnelig funktion $f : \Sigma^{*} \longrightarrow \Sigma^{*}$ hvor $w \in A \iff f(w) \in B$. Vi skriver $A \le_{m} B$ og kalder $f$ en \textit{reduktion} af $A$ til $B$.
\end{definition}

Et eksempel på denne er vores tidligere brug af $EQ_{TM}$ til at afgøre $E_{TM}$. Der havde vi \(\langle m \rangle \stackrel{f}{\longrightarrow} \langle m \rangle \langle m_{\emptyset} \rangle \) som er en mapping reduktion fra $E_{TM}$ til $EQ_{TM}$.

\begin{theorem}
	Hvis $A \le_{m} B$, og $B$ er afgørligt, så er $A$ afgørligt.
\end{theorem}
\begin{proof}
	Vi konstruerer en Turingmaskine $M_{A}$ der tager input $w$ og bruger to funktioner: $M_{f}$ og $M_{B}$. $M_{f}$ tager $w$ og beregner det til $f(w)$ på en sådan måde, at hvis $f \in A$ så er $f(w) \in B$. Dernæst tager vi $f(w)$ til $M_{B}$ som afgører hvorvidt $f(w)$ er i $B$ eller ej (og i forlængelse heraf, om $w$ er i $A$ eller ej).
\end{proof}

\begin{corollary}
	Hvis $A \le_{m} B$ og $A$ er uafgørligt så er $B$ er uafgørligt.
\end{corollary}
\begin{theorem}
	$A \le_{m} B$ og $B$ er genkendeligt så er $A$ genkendeligt.
\end{theorem}

\begin{corollary}
	$A \le_{m} B$ og $A$ er \textbf{ikke} genkendeligt så er $B$ \textbf{ikke} genkendeligt.
\end{corollary}

\begin{claim}
	$A_{TM} \le  \overline{E_{TM}}$
\end{claim}

\begin{proof}
	Givet \(\langle m , w \rangle \) konstruerer vi $\langle m' \rangle $ således at $m'$ er en Turingmaskine og:
	\begin{equation*}
		L(m') = \begin{cases}
			\emptyset  & \text{ hvis } m \text{ ikke er en TM eller } m \text{ er en TM, men } w \notin L(m) \\
			\Sigma^{*} & \text{ ellers } (w \in L(m))
		\end{cases}
	\end{equation*}
	Nu har vi at \(\langle m, w \rangle \in A_{TM} \iff \langle m' \rangle \in \overline{E_{TM}}\).
\end{proof}

\begin{remark}
	\label{remark:albiffnalnb}
	$A \le_{M} B \iff \overline{A} \le_{M} \overline{B}$
\end{remark}
\begin{proof}
	Lad $f$ være en funktion hvor $w \in A \iff f(w) \in B$ så $w \in \overline{A} \iff w \notin A \iff f(w) \notin B \iff f(w) \in \overline{B}$.
\end{proof}
\begin{theorem}
	Hverken $EQ_{TM}$ eller $\overline{EQ_{TM}}$ er genkendelige.
\end{theorem}
\begin{proof}
	Vi ved at $\overline{A_{TM}}$ er ugenkendeligt, så det er nok at vise at $\overline{A_{TM}} \le_{m} EQ_{TM}$ og $\overline{A_{TM}} \le_{M} \overline{EQ_{TM}}$. Fra bemærkelse~\ref{remark:albiffnalnb} er dette det samme som at vise at $A_{TM} \le_{M} \overline{EQ_{TM}}$ og $A_{TM} \le_{M} EQ_{TM}$.

	Vi viser nu reduktionerne:
	$A_{TM} \le_{M} \overline{EQ_{TM}}$: Givet \(\langle M , w \rangle \) konstruer $\hat{M}_{w}$ hvor:
	\begin{equation*}
		L(\hat{M}_{w}) = \begin{cases}
			\emptyset  & \text{ hvis } w \notin L(M) \text{ eller } \langle M \rangle \text{ ikke er en TM} \\
			\Sigma^{*} & \text{ hvis } w \in L(M)
		\end{cases}
	\end{equation*}
	Så gælder det at \(\langle M, w \rangle \in A_{TM} \iff \langle \hat{M}_{w} \rangle \langle M_{\emptyset} \rangle \in \overline{EQ_{TM}} \)
	Ved $A_{TM} \le_{M} EQ_{TM}$, konstruer $\hat{M}_{W}$ som før. Så gælder det at $\langle M , w \rangle \in A_{TM} \iff \langle \hat{M}_{w} \rangle \langle M_{\Sigma^{*}} \rangle \in EQ_{TM}$.
\end{proof}

\section{Flere uafgørlige problemer}%
\label{sec:label}

\begin{note}[Kilder]
	Video 14
\end{note}

Givet sproget $H_{\varepsilon} = \{ \langle m \rangle  \mid m \text{ er en Turingmaskine og } \varepsilon \in L(M)\}$, følger følgende sætning:

\begin{theorem}
	$H_{\varepsilon}$ er uafgørligt.
\end{theorem}

Bemærk først, at sproget $\{w \mid w = \varepsilon\}$ er et afgørligt sprog, en DFA hvis en acceptstate er startsstaten har præcis dette sprog. En Chomsky Grammatik hvis startsymbol peger til den tomme streng har også dette sprog.

\begin{proof}
	Vi giver en mappingreduktion fra $A_{TM}$ til $H_{\varepsilon}$. $\langle m , w\rangle \stackrel{f}{\longrightarrow} \langle m_{w} \rangle $. Husk at en mappingreduktion tager et input, i dette tilfælde fra $A_{TM}$ til $H_{\varepsilon}$. Altså er $\langle m , w \rangle \in A_{TM} \iff \langle m_{W} \rangle \in H_{\varepsilon}$.

	Vi lader sproget af $m_{w}$ være følgende:
	\begin{equation*}
		L(m_{w}) = \begin{cases}
			\emptyset  & \text{ hvis } w \notin L(m) \\
			\Sigma^{*} & \text{ hvis } w \in L(m)    \\
		\end{cases}
	\end{equation*}
	Bemærk her at $\Sigma^{*}$ indeholder $\varepsilon$.
\end{proof}

Vi kigger nu på om sproget der bestemmer om en Turingmaskines sprog er regulært.

\begin{equation}
	Regular_{TM} = \{ \langle m \rangle \mid m \text{ er en Turingmaskine og } L(m) \text{ er regulært} \}
\end{equation}

\begin{theorem}
	$Regular_{TM}$ er uafgørligt.
\end{theorem}

\begin{proof}
	Vi giver en mappignreduktion fra $A_{TM}$ til $Regular_{TM}$ ved at vise hvordan man konstruerer en Turingmaskine $m^{w}$\footnote{Jeg følger Jørgens notation (sådan da), og han bruger $m^{w}$ fordi han blev træt af $m_{w}$.} hvor:
	\begin{equation*}
		L(m^{w}) = \begin{cases}
			L' = \{a^{n}b^{n} \mid n \ge 0\} & \text{ hvis } w \notin L(m) \\
			\Sigma^{*}                       & \text{ hvis } w \in L(m)
		\end{cases}
	\end{equation*}
	Vi kan tydeligt her se at $L(m^{w})$ er regulært hvis og kun hvis $\langle M, w \rangle \in A_{TM}$. Vi skal blot vise nu at en Turingmaskine kan konstruere $m^{w}$ fra $\langle M , w \rangle $.
\end{proof}

\section{Rice's Sætning}%
\label{sec:label}

En egenskab $P$ vedrører sproget for en Turing-maskine $M$, hvis $P$ handler om sproget for $M$ (det vil sige, om L(M)). Eksempler på sådanne egenskaber:
\begin{itemize}
	\item $L(M)$ er regulært
	\item $L(M) = \emptyset$
	\item $L(M)$ indeholder strenge $x,y$ hvor $|x| = |y|$
	\item $L(M)$ indeholder en streng $x$ hvor $|x| = 22$
	\item $\forall i = 1, 2, \ldots$ indeholder $L(M)$ $x_{i}$ hvor $|x_{i}| = i$
\end{itemize}

Følgende egenskab er \textbf{ikke}  om sproget af en Turingmaskine:
\begin{itemize}
	\item Der eksisterer en streng i $\Sigma^{*}$ hvor $M$ starter på $w$ og besøger alle dens tilstande undtagen en af $q_{acc}$, $q_{rej}$.
\end{itemize}

\begin{definition}
	En egenskab $P$ om sproget af en Turingmaskine er \textit{ikke-trivielt} hvis
	\begin{itemize}
		\item Der eksisterer en Turingmaskine $M_{1}$ hvor $L(M_{1})$ har egenskaben $P$
		\item Der eksisterer en Turingmaskine $M_{2}$ hvor $L(M_{2})$ \textbf{ikke} har egenskaben $P$.
	\end{itemize}
\end{definition}

\begin{theorem}[Rice's Sætning]
	\label{teo:rice}
	Hver non-triviel egenskab $P$ om sproget af en Turingmaskine er uafgørligt.
\end{theorem}

\begin{proof}
	Vi antager at det tomme sprog $L(M) = \emptyset$ ikke har egenskaben $P$, ellers kan man kigge på $\overline{P}$.
	Antag at Turingmaskinen $M_{P}$ kan afgøre, for en given kodning $\langle M \rangle $ af en Turingmaskine, om $L(M)$ har en egenskab $P$. Lad $M_{1}$ være en Turingmaskine hvor $L(M_{1})$ har egenskaben $P$. Vi viser nu hvordan man konstruerer en Turingmaskine som, givet \(\langle M, w \rangle \) konstruerer en Turingmaskine $M(w)$ hvor:
	\begin{equation*}
		L(M(w)) = \begin{cases}
			\emptyset & \text{ hvis } w \notin L(M) \text{ eller } m \text{ ikke er en TM} \\
			L(M_{1})  & \text{ hvis } w \in L(M)
		\end{cases}
	\end{equation*}

	Måden vi konstruerer $M(w)$ på, er ved at bruge to funktioner, en for $U'$ og en for $M_{1}$. Først tager vi \(\langle M, w \rangle \) og inputter til $U'$, som er en modificeret version af den universelle Turingmaskine. Hvis \(\langle M \rangle \) ikke er en Turingmaskine løkker den uendeligt. Hvis \(\langle M \rangle \) er en Turingmaskine kører den som normalt, og hvis den accepterer, så starter den $M_{1}$ som så kører som normalt. Det vil sige, at $L(M(w))$ har egenskaben $P$ $\iff$ $\langle M , w \rangle \in A_{TM}$. Så $\langle M, w \rangle \longrightarrow M(w)$ er en mapping reduktion.

	Altså betyder det at for alle sprog der siger at en Turingmaskines sprog skal have en non-triviel egenskab er uafgørlige. Vi kan vise det ved at have en Turingmaskine $B$ som udregner $M(w)$ og så fodrer den til $M_{P}$. Det vil sige, at \(\langle M , w \rangle \) accepteres hvis og kun hvis $w \in L(M_{1})$.
\end{proof}

Vi går nu videre til et andet problem:
\begin{align*}
	L_{ALL} & = \{ \langle M \rangle \mid M \text{ er en Turingmaskine og når den starter}       \\
	        & \text{på den tomme streng vil } M \text{ besøge alle dens tilstande undtagen en}\}
\end{align*}

\begin{theorem}
	$L_{ALL}$ er uafgørligt.
\end{theorem}

\begin{proof}
	Ide er at lave en mapping reduktion fra $A_{TM}$ til $L_{ALL}$. Givet en Turingmaskine $R$ med $r$ tilstande, kan vi ændre dens kode $\langle R \rangle$ til koden $\langle R^{*} \rangle $ af Turingmaskinen $R^{*}$ som følger:
	\begin{itemize}
		\item Tilføj en ny tilstand $q^{*}$ som ikke er i $R$'s mængde af tilstande
		\item Tilføj et nyt symbol $\hat{a}$ som ikke er i $R$'s alfabet
		\item Modificer $\delta$ således at:
		      \begin{itemize}
			      \item Hver overføring $\delta(q', \beta) = q_{accept}$ ændres til $\delta(q', \beta) = (q^{*}, \hat{a},S)$
			      \item $\delta(q^{*}, \hat{a}) = (q, \hat{a}, S)$
			      \item \(\delta(q_{i}, \hat{a}) = (q_{i+1}, \hat{a}, S)\) hvis $i < r -2$
			      \item $\delta(q_{r-2}, \hat{a}) = q_{accept}$
		      \end{itemize}
	\end{itemize}

	Så besøger $R^{*}$ alle tilstande undtagen en hvis og kun hvis den accepterer sit input.

	Givert $\langle M , w \rangle $ kan vi konstruere en Turingmaskien $U_{M,w}$ som har alle koder af $M$ og $w$ i sin egen kode. Når $U_{M,w}$ bliver startet på den tomme streng:

	\begin{enumerate}
		\item Først printer den \(\langle w \rangle \) på bånd 1 og \(\langle m \rangle \) på bånd 2.
		\item Så simulerer den $m$ på $w$, som den universelle Turingmaskine ville gøre det
		\item Accepter hvis og kun hvis $m$ accepterer $w$.
	\end{enumerate}

	Nu har vi $\langle U^{*}_{M,W} \rangle \in L_{ALL} \iff \langle M, w \rangle \in A_{TM}$ og $\langle M , w \rangle \longrightarrow \langle U^{*}_{M,w} \rangle $ er en mappingreduktion. Dermed er $L_{ALL}$ uafgørlig.

\end{proof}




%%% Local Variables:
%%% mode: latex
%%% TeX-engine: xetex
%%% TeX-command-extra-options: "-shell-escape"
%%% TeX-master: "main"
%%% End:
