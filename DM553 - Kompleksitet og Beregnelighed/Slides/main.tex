%% Hey gutter! Jeg håber I er færdige nu, ellers venter I bare med at svare til i er :heart:
%%
%% I forhold til oplæsning er følgende min plan:
%% Jeg præsenterer alle emnerne, så I ikke behøver lave andet end at læse op. Hvis nogen har lyst til at præsentere må de selvfølgelig også det, men jeg har ingen andre kurser, så det er absolut oplagt for mig at gøre det. Hvis I gerne vil, men ikke har tid til at lave slides, kan I bare bruge mine. Jeg skal bare lige give jer adgang til mine noter på github. Jeg skal nok få lavet nogle gode kahoots til dem alle den her gang ;)
%% 17. Maj:
%%    Spørgsmål 1
%% Pensum:
%% * (Sipser 1.1-1.3)
%% * Sipser 1.4: \textbf{Non-regular Languages}
%% * Sipser 2.3: \textbf{Non-context-free Languages}
%% * Weekly Note 1
%% * Video 1
%% * Video 2
%%
%% 19. Maj:
%%    Spørgsmål 2
%% Pensum:
%% * Sipser 2.1-2.2: \textbf{Pushdown Automata and Context-free Languages}
%% * Weekly Note 2
%% * Video 3-6
%%
%% 21. Maj:
%%    Spørgsmål 3
%% Pensum:
%% * Sipser 3: \textbf{Turing Machines}
%% * Weekly Note 3
%% * Weekly Note 4
%% * Video 7-10
%%
%% 23. Maj:
%%    Spørgsmål 4
%% Pensum:
%% * Sipser 4: \textbf{Afgørlighed} (Undtagen Sætning 4.17)
%% * Siper 5.1 pp. 215-220 + 5.3: \textbf{Reducérbarhed}
%% * Weekly Note 5
%% * Weekly Note 6
%% * Video 11-14
%%
%% 25. Maj:
%%    Spørgsmål 5
%% Pensum:
%% * Siper 7.1-7.4: \textbf{Tidskompleksitet inkl. P, NP, NP-komplethed}
%% * CLRS 34: \textbf{NP-Komplethed Beviser} (minus sider 1070-1078)
%% * Weekly Note 7
%% * Weekly Note 8
%% * Weekly Note 9 (ish)
%% * Video 15-17
%%
%% 27. Maj:
%%    Spørgsmål 6
%% Pensum:
%% * Sipser 7.4: \textbf{Bevis at CNF-SAT er NP-Komplet}
%% * Weekly Note 9
%% * Video 18 (Bonus \href{https://www.youtube.com/watch?v=6Az1gtDRaAU}{video})
%%
%% 29. Maj:
%%    Spørgsmål 7
%% Pensum:
%% * CLRS 35: \textbf{Approximationsalgoritmer}
%% * Weekly Note 10 (igen)
%% * Video 19-21
%%
%% 31. Maj:
%%    Spørgsmål 8
%% Pensum:
%% * Baase 2.4: \textbf{Informationsteoretiske nedre grænser for sortering ved sammenligninger}
%% * Jørgens noter: \textbf{Noter på nedre grænser}
%% * CLRS 9.3: \textbf{Median Problem}
%% * Weekly Note 11
%% * Video 22-24
%%
%% 2. Juni:
%%    Spørgsmål 9
%% Pensum:
%% * Baase 3: \textbf{Adversary nedre grænse argumenter}
%% * Baase 3.5: \textbf{Adversary Median Problem}
%% * Jørgens noter: \textbf{Noter på nedre grænser}
%% * Weekly Note 12
%% * Video 24-25
%%
%% 4. Juni:
%%    Spørgsmål 10
%% Pensum:
%% * Cygan et al pp. 3-8, 12-14, 17--22, 51-55: \textbf{Algoritmer med faste parametre}.
%% * F.V. Fomin, D. Kratsch pp. 1-6: \textbf{Eksakte Algoritmer}.
%% * Weekly Note 13
%% * Video 26
%%
%% 5-12 juni: (8 dage inkl.)
%% Se på de svage led, øvepræsentationer, muligvis opgaver

\documentclass{beamer}

\title{DM553 Emner}
\subtitle{Til mundtlig præsentation af hele pensum.}

\author{Kevin Vinther}
\institute{IMADA}
\date{6. Maj - 14 Juni (?)}


% Theme setting
\usetheme{default}
\usecolortheme{dove}

\usepackage{blindtext}
\usepackage{hyperref}
\usepackage[T1]{fontenc} %thanks's daleif
\usepackage[utf8]{inputenc}
\usepackage{amsthm}

% Sæt sproget til dansk
\usepackage{polyglossia}
\setdefaultlanguage{danish}
\setotherlanguages{english}



% Fonts and layout
\usefonttheme{professionalfonts}
\setbeamerfont{frametitle}{series=\bfseries}
\setbeamertemplate{navigation symbols}{}
\setbeamertemplate{footline}[frame number]

\usepackage{tikz} % Til grafer
\usetikzlibrary{automata} % Til automatgrafer
\usetikzlibrary{positioning} % Positionering af knuder
\usetikzlibrary{arrows.meta, arrows} % arrows til at ændre kanter
\tikzset{
	node distance=2.5cm, % minimumsdistancen mellem to knuderj
	every state/.style={
			semithick,
			fill=gray!10
		},
	initial text={}, % Ingen tekst, bare kanter
	double distance=2pt, % Udseende på accept states
	every edge/.style={ % Egenskaber for hver transition
			draw,
			>=stealth', % bold arrowheads
			auto,
			semithick
		}
}
% Tak til: https://www3.nd.edu/~dchiang/teaching/theory/2018/www/tikz_tutorial.pdf

\begin{document}

%%% Local variables:
%%% Mode: latex
%%% Tex-engine: xetex
%%% Tex-command-extra-options: "-shell-escape"
%%% Tex-master: "main"
%%% End:


\begin{frame}[allowframebreaks]
	\titlepage
\end{frame}


\begin{frame}[allowframebreaks]
	\frametitle{Indholdsfortegnelse}
	\tableofcontents
\end{frame}


\section{Pumpelemmaet}%
\label{sec:Pumpelemma}

\begin{frame}
  \frametitle{Pensum}
  \begin{itemize}
    \item (Sipser 1.1-1.3)
    \item Sipser 1.4: \textbf{Non-regular Languages}
    \item Sipser 2.3: \textbf{Non-context-free Languages}
    \item Weekly Note 1
    \item Video 1
    \item Video 2
  \end{itemize}
\end{frame}



%%% local variables:
%%% mode: latex
%%% tex-engine: xetex
%%% tex-command-extra-options: "-shell-escape"
%%% tex-master: "main"
%%% end:

\section{Pushdown Automater og Kontekstfrie Sprog}%
\label{sec:pdacfg}

\begin{frame}
  \frametitle{Pensum}
  \begin{itemize}
    \item Sipser 2.1-2.2: \textbf{Pushdown Automata and Context-free Languages}
    \item Weekly Note 2
    \item Video 3-6
  \end{itemize}
\end{frame}



%%% local variables:
%%% mode: latex
%%% tex-engine: xetex
%%% tex-command-extra-options: "-shell-escape"
%%% tex-master: "main"
%%% end:

\section{Turingmaskiner}%
\label{sec:turingmachines}

\begin{frame}
  \frametitle{Pensum}
  \begin{itemize}
    \item Sipser 3: \textbf{Turing Machines}
    \item Weekly Note 3
    \item Weekly Note 4
    \item Video 7-10
  \end{itemize}
\end{frame}



%%% local variables:
%%% mode: latex
%%% tex-engine: xetex
%%% tex-command-extra-options: "-shell-escape"
%%% tex-master: "main"
%%% end:

\chapter{Afgørlighed}

\section{Uafgørlighed}%
\label{sec:label}

\begin{note}[Kilder]
	Video 12
\end{note}

Vores mål til at starte med, er at bevise at $A_{TM}$ er uafgørligt. FØr vi gør det, kigger vi på teori omkring uendeligheder.

\begin{definition}[Tællelig mængde]
	En mængde $S$ er tællelig hvis:
	\begin{enumerate}
		\item $S$ er endelig, eller
		\item $\exists f : S \rightarrow \mathbb{N}$ hvor $f$ er en-til-en og onto.
	\end{enumerate}
\end{definition}

\begin{theorem}
	Mængden $B$ af uendelige binære strenge er overtællelig.
\end{theorem}

\begin{proof}
	Antag at $b_{1}, b_{2}, b_{3}, \ldots, b_{k}, b_{k+1}, \ldots$ er en liste af alle uendelige binære strenge.
	Vi definerer $b^{*}$ til at være en binær streng, som altid vil være anderledes end alle andre (og derfor ikke tillader one-to-one og onto). Vi definerer $b^{*}(i)$ til at være den $i$'e plads i $b^{*}$:
	\begin{equation}
		b^{*}(i) = \begin{cases}
			1 & \text{ hvis } b_{i}(i) = 0 \\
			0 & \text{ hvis } b_{i}(i) = 1 \\
		\end{cases}
	\end{equation}
	Ud fra denne definition vil $b^{*}$ \textbf{ikke} kunne være en del af $b_{1}, b_{2}, \ldots$, da den er forskellig fra alle strenge på en eller anden måde. Antag $b^{*} = b_{j}$, så $b^{*}(j) = 1 - b_{j}(j) \ne b_{j}(j)$. Altså $b^{*}$ på position $j$, men $b^{*}(j)$ er anderledes fra $b_{j}$.
\end{proof}

Hvert sprog over et alfabet $\Sigma$ er en delmængde af $P(\Sigma^{*})$ som er mængden af alle delmængder af $\Sigma^{*}$.  Vi ved at $\Sigma^{*}$ er tællelig, fordi vi kan ordne dem leksikografisk $w_{1}, w_{2}, \ldots$. $L$ over alfabetetet $\Sigma$ korresponderer one-to-one til en unik endelig binær streng $b_{L}$ hvor:
\begin{equation}
	b_{L}(i) = \begin{cases}
		1 & \iff w_{i} \in L    \\
		0 & \iff w_{i} \notin L
	\end{cases}
\end{equation}
Det vil altså sige at der er lige så mange sprog, som der er binære strenge, hvilket er overtælleligt.

\begin{corollary}
	Mængden af alle sprog over et ikke-trivielt (\( |\Sigma| \ge 1\)) alfabet \(\Sigma\) er overtælleligt.
\end{corollary}


\textbf{MANGLER: Bevis for at der er tælleligt mange turingmaskiner. omkring 15:00-22:00 i video 12}.

\begin{theorem}
	$A_{TM}$ er uafgørligt.
\end{theorem}

\begin{proof}
	Antag at $H$ er en Turingmaskine som afgører $A_{TM}$.

	Vi kan definere $H$ til at være følgende:
	\begin{equation}
		H = \begin{cases}
			\text{\textit{accept}} & \iff \langle M, w\rangle \in A_{TM}     \\
			\text{\textit{afvis}}  & \iff \langle M, w \rangle \notin A_{TM}
		\end{cases}
	\end{equation}

	Vi bruger $H$ som funktion i en ny Turingmaskine, $D$. $D$ tager kun et input, en kodning af en turingmaskine, i.e. $\langle M \rangle$. Denne kodning fodrer $D$ til $H$ som begge input, altså både som kodningen af turingmaskinen, og strengen den accepterer. I sin essens, spørger $D$ turingmaskinen $H$ om en Turingmaskine accepterer konfigurationen på en anden turingmaskine. Resultatet af dette omvendes så, så hvis $H$ accepterer, så \textit{afviser} $D$, og omvendt.

	Et problem kommer frem her, for hvad hvis vi kører $D$ på input $\langle D \rangle$? Så vil følgende ske:
	\begin{equation}
		\label{eq:wrongatm}
		D(\langle D \rangle ) = \begin{cases}
			\text{\textit{accept}} & \text{ hvis } \langle D \rangle \notin L(D) \\
			\text{\textit{afvis}}  & \text{ hvis } \langle D \rangle \in L(D)    \\
		\end{cases}
	\end{equation}
	Altså ud fra \eqref{eq:wrongatm} får vi at $\langle D \rangle \in L(D) \iff \langle D \rangle \notin L(D)$. Dermed kan $H$ ikke eksistere.
\end{proof}

Givet en matrix af Turingmaskiner, hvor de yderste rækker og kolonner er alle de tælleligt uendelige kodninger til Turingmaskiner. Antag at $\langle m_{j} \rangle$. Den første kolonne er Turingmaskiner, og den første række er Turingmaskiner, men som sprog.

Derfor, ved en celle i matrixen \(\langle m_{i} \rangle \langle m_{j} \rangle\)  hvor $m_{i}$ er fra første kolonne, og $m_{j}$ fra første række:
\begin{equation*}
	\langle m_{i} \rangle \langle m_{j} \rangle = \begin{cases}
		1 & \text{ hvis } \langle m_{j} \rangle \in L(m_{i}) \\
		0 & \text{ ellers}
	\end{cases}
\end{equation*}

Vi antager at $H$ eksisterer. Dermed må $D$ også eksistere, da den bare swapper \textit{accept} og \textit{afvis} tilstande. $D = \langle m_{i} \rangle $ for en $m_{i}$ i listen af alle Turingmaskiner. Men $D$ er uenig med $m_{i}$ på input $\langle m_{i} \rangle $ $\langle m_{i} \rangle \in L(m_{i}) \iff \langle m_{i} \rangle \notin  L(D) = \langle m_{i} \rangle $. Dermed er $D$ ikke i listen og så eksisterer den ikke. Så $A_{TM}$ er \textbf{ikke} aførlig.


\begin{theorem}
	\label{teo:lnotlrec}
	$L$ er Turing-afgørligt $\iff$ $L$ og $\overline{L}$ er Turing-genkendelige.
\end{theorem}

\begin{proof}
	\(\Rightarrow\) Lad $m^{*}$ afgøre $L$. $m^{*}$stopper altid og $L(m^{*}) = L$ så $m^{*}$ genkender $L$ og $\overline{m^{*}}$ genkender $\overline{L}$ ved at tage outputtet fra $m^{*}$ og gøre det omvendt, i.e. afvise hvis den accepterer, og acceptere hvis den afviser.\\
	\(\Leftarrow\) Lad $m_{L}, m_{\overline{L}}$ genkender hhv. $L$ og $\overline{L}$. Vi kan ikke lave en Turingmaskine der blot kører $m_{L}$ og tjekker om den accepterer eller ikke accepterer, da den ikke nødvendigvis er en afgører. Dermed, fremfor at lave en sådan Turingmaskine, laver vi en Turingmaskine $M$ der kører $m_{L}$ og $m_{\overline{L}}$ på samme tid på samme input $w$. Hvis $m_{L}$ accepterer, så accepterer $M$. Hvis $m_{\overline{L}}$ accepterer, så \textit{afviser} $M$. Omvendt, hvis $m_{L}$ afviser, så afviser $M$, og hvis $m_{\overline{L}}$ afviser, så \textit{accepterer} $M$.
\end{proof}

\begin{theorem}
	For hvert sprog $L$ over det universelle alfabet, gælder præcis en af følgende:
	\begin{enumerate}
		\item $L$ og $\overline{L}$ er afgørlige
		\item Ingen af $L$, $\overline{L}$ er genkendelige
		\item $L$ er genkendeligt, men $\overline{L}$ er ugenkendeligt, eller omvendt.
	\end{enumerate}

\end{theorem}

\begin{proof}
	\[
		\begin{array}{c|ccc}
			\overline{L} \setminus L & A & G & IG \\
			\hline
			A                        & * & - & -  \\
			G                        & - & - & *  \\
			IG                       & - & * & *  \\
		\end{array}
	\]

	\begin{align*}
		A  & = \text{afgørlig}                      \\
		G  & = \text{genkendelig men ikke afgørlig} \\
		IG & = \text{ikke genkendelig}              \\
		*  & = \text{mulig}                         \\
		-  & = \text{umulig}
	\end{align*}
\end{proof}

\begin{corollary}
	$\overline{A_{TM}}$ er ugenkendelig.
\end{corollary}
\begin{proof}
	$\overline{A_{TM}} = \{ \langle w' \rangle \mid \text{ 1. Ingen præfiks af } \langle w' \rangle \text{ koder en Turingmaskine eller}$
	$\text{2. For hver } \langle m \rangle \text{ hvor } \langle w' \rangle = \langle m \rangle \langle w \rangle \text{ har vi at } w \notin L(m)\}$

	Vi ved at $\overline{A_{TM}}$ er ugenkendelig, da, hvis den var genkendelig, så ville $A_{TM}$ være afgørlig (ud fra Sætning~\ref{teo:lnotlrec}).
\end{proof}

\section{Uafgørlige Problemer}%
\label{sec:label}

\begin{note}[Kilder]
	Video 13
\end{note}

Vi starter med at definere standseproblemet, hvor vi hovedsageligt kommer til at bruge det engelske navn: \textit{Halting Problem}.
\begin{equation*}
	HALT = \{ \langle m \rangle \langle w \rangle \mid m \text{ er en Turingmaskine og } m \text{ stopper på }w\}
\end{equation*}

\begin{theorem}
	\label{teo:halt}
	$HALT$ er uafgørligt.
\end{theorem}

\begin{proof}
	Antag at turingmaskinen $R$ afgører $HALT$. Vi viser nu at dette vil mene at $A_{TM}$ er afgørligt.

	Udover denne, så konstruerer vi Turingmaskinen $A$ som tager \(\langle m \rangle \) som input, og giver \(\langle m' \rangle \) som output. Hvis \(\langle m \rangle \) er en korrekt kodning af en Turingmaskine, er $L(m') = L(m)$ men hvor $L(m')$ løkker på alle stringe der ikke er en del af $L(m')$. Omvendt, hvis $\langle m \rangle$ ikke er en korrekt kodning, så $\langle m' \rangle = \langle m \rangle$.

	Vi konstruerer nu $m_{R}$ til at afgøre $A_{TM}$. Først giver vi inputtet \(\langle m \rangle \) til $A$ og \(\langle w \rangle \) til $R$, og så outputter vi \(\langle m' \rangle \) fra $A$ og giver videre til input i $R$. Hvis $R$ accepterer, så accepterer $m_{R}$ og omvendt. Dermed, hvis $w \in L(m')$ så accepterer vi, og hvis den løkker (som den kun kan gøre, hvis $w \notin L(m')$), så afviser vi. Dermed er $M_{R}$ en afgører for $A_{TM}$.
\end{proof}

Vi kigger nu på tomhedsproblemet:
\begin{equation}
	\label{eq:etm}
	E_{TM} = \{\langle m \rangle \mid m \text{ er en Turingmaskine og } L(M) = \emptyset\}
\end{equation}
\begin{theorem}
	$E_{TM}$ er uafgørligt.
\end{theorem}
\begin{proof}
	Antag at $S$ er en Turingmaskine som afgører $E_{TM}$. Vi bruger $S$ til at konstruere en ny Turingmaskine som vil afgøre $A_{TM}$.

	Derudover bruger vi Turingmaskinen $B$ som tager input \(\langle M, w \rangle \) og outputter $\langle M_{w} \rangle$. Den konstruerer $\langle M_{w} \rangle $ ved at først tjekke om \(\langle M \rangle \) er en korrekt kodning af en Turingmaskine. Hvis ikke, så er $\langle M_{w} \rangle $ (outputtet) en Turingmaskine, hvor $L(M_{w}) = \emptyset$. Hvis \(\langle m \rangle \) er en Turingmaskine, så er $M_{w}$ en turingmaskine med følgende sprog:
	\begin{equation}
		\label{eq:emptiness}
		L(M_{w}) = \begin{cases}
			\{w\}     & \text{ hvis } w \in L(m)    \\
			\emptyset & \text{ hvis } w \notin L(m)
		\end{cases}
	\end{equation}
	Altså vil $M_{w}$ kun acceptere $w$, hvis $w \in L(m)$ og ellers accepterer den det tomme sprog. Dermed vil $M_{w}$ blot være den følgende algoritme:
	\begin{enumerate}
		\item Tjek om input $x = w$
		\item Hvis $x \ne w$, så \textit{afvis}.
		\item Ellers simulér $M$ på $w$ og accepter hvis og kun hvis $M$ accepterer $w$.
	\end{enumerate}

	Vi kan nu konstruere $S^{*}$ til at afgøre $A_{TM}$. Først tager vi $\langle M , w\rangle $ som input til $B$, som så producerer $\langle M_{w} \rangle $. Vi fodrer så $\langle M_{w} \rangle $ til $S$, som accepterer hvis og kun hvis $w \notin L(M)$, og afviser ellers. $S^{*}$ tager så dette output, og bytter om på det, så en accept bliver afvist, og en afvisning bliver accepteret. Dermed har vi en afgører for $A_{TM}$.
\end{proof}

Vi kommer til at bruge mange maskiner med et sprog lignende den i \eqref{eq:emptiness} i fremtidige beviser om uafgørlighed. Det vigtige at forstå ved Turingmaskinen $M_{w}$ er at den \textit{ikke skal køres}. Vi bruger en anden Turingmaskine ved $M_{w}$, nemlig $R$ til at analysere den, og finde ud af noget om den, i dette tilfælde om dens sprog er tomt eller ej. Dette er et mønster vi kommer til at se mere af, og konkluderer i Rice's sætning.

Vi kan faktisk genkende $\overline{E_{TM}}$. Vi kan genkende $\overline{E_{TM}}$ på følgende måde:
\begin{enumerate}
	\item Tjek om \(\langle w \rangle \) koder en Turingmaskine, hvis ikke, \textit{accepter}  \(\langle w \rangle \)
	\item Lad $M$ være en Turingmaskine kodet af $\langle w \rangle $
	\item Simulér $M$ på strenge over $\Sigma^{*}$ i leksikografisk ordning parallelt:
	      \begin{enumerate}
		      \item For $i = 1, 2, \ldots$
		      \item Simuler $m$ på $i$ skridt på strengene $w_{1}, w_{2}, \ldots, w_{i}$ ifølge leksikografisk ordning.
		      \item Stop når en streng accepteres.
	      \end{enumerate}
\end{enumerate}

Bemærk at $\overline{E_{TM}}$, gennem den algoritme vi har konstrueret, \textbf{ikke} er afgørlig.

\begin{corollary}
	$E_{TM}$ er Turing-ugenkendeligt.
\end{corollary}

Corollarien kommer fra Sætning~\ref{teo:lnotlrec}.

Vi definerer nu to specielle Turingmaskiner, som vi kommer til at bruge i fremtidige beviser:
\begin{itemize}
	\item $M_{\emptyset}$: Går direkte til sin afvisningstilstand, uanset input. $L(M_{\emptyset}) = \emptyset$
	\item $M_{\Sigma^{*}}$: Går direkte til sin accepttilstand, uanset input. $L(M_{\Sigma^{*}}) = \Sigma^{*}$.
\end{itemize}

Vi vil nu kigge på lighedsproblemet.

\begin{equation*}
	EQ_{TM} = \{\langle m_{1}, m_{2} \rangle \mid m_{1} \text{ og } m_{2} \text{ er Turingmaskiner, hvor } L(m_{1}) = L(m_{2})\}
\end{equation*}

\begin{theorem}
	$EQ_{TM}$ er uafgørlig.
\end{theorem}

\begin{proof}
	Vi beviser ved at \textit{reducere} fra $E_{TM}$ \eqref{eq:etm} til $EQ_{TM}$.
	Antag at $m_{EQ}$ afgører $EQ_{TM}$, så kan vi lave en Turingmaskien $m^{*}$ som givet $M_{\emptyset}$ og en Turingmaskine af eget valg, accepterer hvis og kun hvis $M_{\emptyset}$ og Turingmaskinen af eget valg er ens. Dette er umuligt, da $m^{*}$ afgører $E_{TM}$.
\end{proof}

\section{Mapping Reducerbarhed}%
\label{sec:label}

Nu stiller vi spørgsmålet: Er det muligt at formalisere idéen om at \textit{reducere} fra et problem til et andet? Svaret er ja!

\begin{definition}
	$f : \Sigma^{*} \rightarrow \Sigma^{*}$ er \textit{beregnelig} hvis der eksisterer en Turingmaskine $M_{f}$ som starter med $w$ og slutter med $f(w)$ på sit bånd, i.e. $q_{0}w \stackrel{*}{\Rightarrow} q_{acc}f(w)$.
\end{definition}

Bemærk, at når en funktion er beregnelig, så stopper den \textbf{altid}. Et eksempel på en beregnelig funktion er den Turingmaskine $A$ fra beviset i Sætning~\ref{teo:halt}. Denne tager input en streng \(\langle w \rangle \) og:
\begin{enumerate}
	\item Tjekker om \(\langle w \rangle \) er en Turingmaskine beskrivelse
	      \begin{itemize}
		      \item Hvis ja: så ouputter $A$  \(\langle m' \rangle \)hvor \(\langle m' \rangle =  \langle w \rangle \), men hvor den løkker på alle strenge den \textit{ikke} accepterer.
		      \item Hvis nej: så outputter $A$ strengen \(\langle w \rangle \).
	      \end{itemize}
\end{enumerate}

Så $A$ udregner funktionen $f : \langle w \rangle  \longrightarrow \langle w' \rangle $ hvor $\langle w' \rangle $ er outputtet fra $A$.

\begin{definition}
	Lad $A,B$ være sprog. Vi siger at $A$ er \textit{mapping reducerbart} til $B$ hvis der eksisterer en beregnelig funktion $f : \Sigma^{*} \longrightarrow \Sigma^{*}$ hvor $w \in A \iff f(w) \in B$. Vi skriver $A \le_{m} B$ og kalder $f$ en \textit{reduktion} af $A$ til $B$.
\end{definition}

Et eksempel på denne er vores tidligere brug af $EQ_{TM}$ til at afgøre $E_{TM}$. Der havde vi \(\langle m \rangle \stackrel{f}{\longrightarrow} \langle m \rangle \langle m_{\emptyset} \rangle \) som er en mapping reduktion fra $E_{TM}$ til $EQ_{TM}$.

\begin{theorem}
	Hvis $A \le_{m} B$, og $B$ er afgørligt, så er $A$ afgørligt.
\end{theorem}
\begin{proof}
	Vi konstruerer en Turingmaskine $M_{A}$ der tager input $w$ og bruger to funktioner: $M_{f}$ og $M_{B}$. $M_{f}$ tager $w$ og beregner det til $f(w)$ på en sådan måde, at hvis $f \in A$ så er $f(w) \in B$. Dernæst tager vi $f(w)$ til $M_{B}$ som afgører hvorvidt $f(w)$ er i $B$ eller ej (og i forlængelse heraf, om $w$ er i $A$ eller ej).
\end{proof}

\begin{corollary}
	Hvis $A \le_{m} B$ og $A$ er uafgørligt så er $B$ er uafgørligt.
\end{corollary}
\begin{theorem}
	$A \le_{m} B$ og $B$ er genkendeligt så er $A$ genkendeligt.
\end{theorem}

\begin{corollary}
	$A \le_{m} B$ og $A$ er \textbf{ikke} genkendeligt så er $B$ \textbf{ikke} genkendeligt.
\end{corollary}

\begin{claim}
	$A_{TM} \le  \overline{E_{TM}}$
\end{claim}

\begin{proof}
	Givet \(\langle m , w \rangle \) konstruerer vi $\langle m' \rangle $ således at $m'$ er en Turingmaskine og:
	\begin{equation*}
		L(m') = \begin{cases}
			\emptyset  & \text{ hvis } m \text{ ikke er en TM eller } m \text{ er en TM, men } w \notin L(m) \\
			\Sigma^{*} & \text{ ellers } (w \in L(m))
		\end{cases}
	\end{equation*}
	Nu har vi at \(\langle m, w \rangle \in A_{TM} \iff \langle m' \rangle \in \overline{E_{TM}}\).
\end{proof}

\begin{remark}
	\label{remark:albiffnalnb}
	$A \le_{M} B \iff \overline{A} \le_{M} \overline{B}$
\end{remark}
\begin{proof}
	Lad $f$ være en funktion hvor $w \in A \iff f(w) \in B$ så $w \in \overline{A} \iff w \notin A \iff f(w) \notin B \iff f(w) \in \overline{B}$.
\end{proof}
\begin{theorem}
	Hverken $EQ_{TM}$ eller $\overline{EQ_{TM}}$ er genkendelige.
\end{theorem}
\begin{proof}
	Vi ved at $\overline{A_{TM}}$ er ugenkendeligt, så det er nok at vise at $\overline{A_{TM}} \le_{m} EQ_{TM}$ og $\overline{A_{TM}} \le_{M} \overline{EQ_{TM}}$. Fra bemærkelse~\ref{remark:albiffnalnb} er dette det samme som at vise at $A_{TM} \le_{M} \overline{EQ_{TM}}$ og $A_{TM} \le_{M} EQ_{TM}$.

	Vi viser nu reduktionerne:
	$A_{TM} \le_{M} \overline{EQ_{TM}}$: Givet \(\langle M , w \rangle \) konstruer $\hat{M}_{w}$ hvor:
	\begin{equation*}
		L(\hat{M}_{w}) = \begin{cases}
			\emptyset  & \text{ hvis } w \notin L(M) \text{ eller } \langle M \rangle \text{ ikke er en TM} \\
			\Sigma^{*} & \text{ hvis } w \in L(M)
		\end{cases}
	\end{equation*}
	Så gælder det at \(\langle M, w \rangle \in A_{TM} \iff \langle \hat{M}_{w} \rangle \langle M_{\emptyset} \rangle \in \overline{EQ_{TM}} \)
	Ved $A_{TM} \le_{M} EQ_{TM}$, konstruer $\hat{M}_{W}$ som før. Så gælder det at $\langle M , w \rangle \in A_{TM} \iff \langle \hat{M}_{w} \rangle \langle M_{\Sigma^{*}} \rangle \in EQ_{TM}$.
\end{proof}

\section{Flere uafgørlige problemer}%
\label{sec:label}

\begin{note}[Kilder]
	Video 14
\end{note}

Givet sproget $H_{\varepsilon} = \{ \langle m \rangle  \mid m \text{ er en Turingmaskine og } \varepsilon \in L(M)\}$, følger følgende sætning:

\begin{theorem}
	$H_{\varepsilon}$ er uafgørligt.
\end{theorem}

Bemærk først, at sproget $\{w \mid w = \varepsilon\}$ er et afgørligt sprog, en DFA hvis en acceptstate er startsstaten har præcis dette sprog. En Chomsky Grammatik hvis startsymbol peger til den tomme streng har også dette sprog.

\begin{proof}
	Vi giver en mappingreduktion fra $A_{TM}$ til $H_{\varepsilon}$. $\langle m , w\rangle \stackrel{f}{\longrightarrow} \langle m_{w} \rangle $. Husk at en mappingreduktion tager et input, i dette tilfælde fra $A_{TM}$ til $H_{\varepsilon}$. Altså er $\langle m , w \rangle \in A_{TM} \iff \langle m_{W} \rangle \in H_{\varepsilon}$.

	Vi lader sproget af $m_{w}$ være følgende:
	\begin{equation*}
		L(m_{w}) = \begin{cases}
			\emptyset  & \text{ hvis } w \notin L(m) \\
			\Sigma^{*} & \text{ hvis } w \in L(m)    \\
		\end{cases}
	\end{equation*}
	Bemærk her at $\Sigma^{*}$ indeholder $\varepsilon$.
\end{proof}

Vi kigger nu på om sproget der bestemmer om en Turingmaskines sprog er regulært.

\begin{equation}
	Regular_{TM} = \{ \langle m \rangle \mid m \text{ er en Turingmaskine og } L(m) \text{ er regulært} \}
\end{equation}

\begin{theorem}
	$Regular_{TM}$ er uafgørligt.
\end{theorem}

\begin{proof}
	Vi giver en mappignreduktion fra $A_{TM}$ til $Regular_{TM}$ ved at vise hvordan man konstruerer en Turingmaskine $m^{w}$\footnote{Jeg følger Jørgens notation (sådan da), og han bruger $m^{w}$ fordi han blev træt af $m_{w}$.} hvor:
	\begin{equation*}
		L(m^{w}) = \begin{cases}
			L' = \{a^{n}b^{n} \mid n \ge 0\} & \text{ hvis } w \notin L(m) \\
			\Sigma^{*}                       & \text{ hvis } w \in L(m)
		\end{cases}
	\end{equation*}
	Vi kan tydeligt her se at $L(m^{w})$ er regulært hvis og kun hvis $\langle M, w \rangle \in A_{TM}$. Vi skal blot vise nu at en Turingmaskine kan konstruere $m^{w}$ fra $\langle M , w \rangle $.
\end{proof}

\section{Rice's Sætning}%
\label{sec:label}

En egenskab $P$ vedrører sproget for en Turing-maskine $M$, hvis $P$ handler om sproget for $M$ (det vil sige, om L(M)). Eksempler på sådanne egenskaber:
\begin{itemize}
	\item $L(M)$ er regulært
	\item $L(M) = \emptyset$
	\item $L(M)$ indeholder strenge $x,y$ hvor $|x| = |y|$
	\item $L(M)$ indeholder en streng $x$ hvor $|x| = 22$
	\item $\forall i = 1, 2, \ldots$ indeholder $L(M)$ $x_{i}$ hvor $|x_{i}| = i$
\end{itemize}

Følgende egenskab er \textbf{ikke}  om sproget af en Turingmaskine:
\begin{itemize}
	\item Der eksisterer en streng i $\Sigma^{*}$ hvor $M$ starter på $w$ og besøger alle dens tilstande undtagen en af $q_{acc}$, $q_{rej}$.
\end{itemize}

\begin{definition}
	En egenskab $P$ om sproget af en Turingmaskine er \textit{ikke-trivielt} hvis
	\begin{itemize}
		\item Der eksisterer en Turingmaskine $M_{1}$ hvor $L(M_{1})$ har egenskaben $P$
		\item Der eksisterer en Turingmaskine $M_{2}$ hvor $L(M_{2})$ \textbf{ikke} har egenskaben $P$.
	\end{itemize}
\end{definition}

\begin{theorem}[Rice's Sætning]
	\label{teo:rice}
	Hver non-triviel egenskab $P$ om sproget af en Turingmaskine er uafgørligt.
\end{theorem}

\begin{proof}
	Vi antager at det tomme sprog $L(M) = \emptyset$ ikke har egenskaben $P$, ellers kan man kigge på $\overline{P}$.
	Antag at Turingmaskinen $M_{P}$ kan afgøre, for en given kodning $\langle M \rangle $ af en Turingmaskine, om $L(M)$ har en egenskab $P$. Lad $M_{1}$ være en Turingmaskine hvor $L(M_{1})$ har egenskaben $P$. Vi viser nu hvordan man konstruerer en Turingmaskine som, givet \(\langle M, w \rangle \) konstruerer en Turingmaskine $M(w)$ hvor:
	\begin{equation*}
		L(M(w)) = \begin{cases}
			\emptyset & \text{ hvis } w \notin L(M) \text{ eller } m \text{ ikke er en TM} \\
			L(M_{1})  & \text{ hvis } w \in L(M)
		\end{cases}
	\end{equation*}

	Måden vi konstruerer $M(w)$ på, er ved at bruge to funktioner, en for $U'$ og en for $M_{1}$. Først tager vi \(\langle M, w \rangle \) og inputter til $U'$, som er en modificeret version af den universelle Turingmaskine. Hvis \(\langle M \rangle \) ikke er en Turingmaskine løkker den uendeligt. Hvis \(\langle M \rangle \) er en Turingmaskine kører den som normalt, og hvis den accepterer, så starter den $M_{1}$ som så kører som normalt. Det vil sige, at $L(M(w))$ har egenskaben $P$ $\iff$ $\langle M , w \rangle \in A_{TM}$. Så $\langle M, w \rangle \longrightarrow M(w)$ er en mapping reduktion.

	Altså betyder det at for alle sprog der siger at en Turingmaskines sprog skal have en non-triviel egenskab er uafgørlige. Vi kan vise det ved at have en Turingmaskine $B$ som udregner $M(w)$ og så fodrer den til $M_{P}$. Det vil sige, at \(\langle M , w \rangle \) accepteres hvis og kun hvis $w \in L(M_{1})$.
\end{proof}

Vi går nu videre til et andet problem:
\begin{align*}
	L_{ALL} & = \{ \langle M \rangle \mid M \text{ er en Turingmaskine og når den starter}       \\
	        & \text{på den tomme streng vil } M \text{ besøge alle dens tilstande undtagen en}\}
\end{align*}

\begin{theorem}
	$L_{ALL}$ er uafgørligt.
\end{theorem}

\begin{proof}
	Ide er at lave en mapping reduktion fra $A_{TM}$ til $L_{ALL}$. Givet en Turingmaskine $R$ med $r$ tilstande, kan vi ændre dens kode $\langle R \rangle$ til koden $\langle R^{*} \rangle $ af Turingmaskinen $R^{*}$ som følger:
	\begin{itemize}
		\item Tilføj en ny tilstand $q^{*}$ som ikke er i $R$'s mængde af tilstande
		\item Tilføj et nyt symbol $\hat{a}$ som ikke er i $R$'s alfabet
		\item Modificer $\delta$ således at:
		      \begin{itemize}
			      \item Hver overføring $\delta(q', \beta) = q_{accept}$ ændres til $\delta(q', \beta) = (q^{*}, \hat{a},S)$
			      \item $\delta(q^{*}, \hat{a}) = (q, \hat{a}, S)$
			      \item \(\delta(q_{i}, \hat{a}) = (q_{i+1}, \hat{a}, S)\) hvis $i < r -2$
			      \item $\delta(q_{r-2}, \hat{a}) = q_{accept}$
		      \end{itemize}
	\end{itemize}

	Så besøger $R^{*}$ alle tilstande undtagen en hvis og kun hvis den accepterer sit input.

	Givert $\langle M , w \rangle $ kan vi konstruere en Turingmaskien $U_{M,w}$ som har alle koder af $M$ og $w$ i sin egen kode. Når $U_{M,w}$ bliver startet på den tomme streng:

	\begin{enumerate}
		\item Først printer den \(\langle w \rangle \) på bånd 1 og \(\langle m \rangle \) på bånd 2.
		\item Så simulerer den $m$ på $w$, som den universelle Turingmaskine ville gøre det
		\item Accepter hvis og kun hvis $m$ accepterer $w$.
	\end{enumerate}

	Nu har vi $\langle U^{*}_{M,W} \rangle \in L_{ALL} \iff \langle M, w \rangle \in A_{TM}$ og $\langle M , w \rangle \longrightarrow \langle U^{*}_{M,w} \rangle $ er en mappingreduktion. Dermed er $L_{ALL}$ uafgørlig.

\end{proof}




%%% Local Variables:
%%% mode: latex
%%% TeX-engine: xetex
%%% TeX-command-extra-options: "-shell-escape"
%%% TeX-master: "main"
%%% End:

\section{NP-kompletheds Beviser}%
\label{sec:npkomplethed}

\begin{frame}
	\frametitle{Pensum}
	\begin{itemize}
		\item Siper 7.1-7.4: \textbf{Tidskompleksitet inkl. P, NP, NP-komplethed}
		\item CLRS 34: \textbf{NP-Komplethed Beviser} (minus sider 1070-1078)
		\item Weekly Note 7
		\item Weekly Note 8
		\item Weekly Note 9 (ish)
		\item Video 15-17
	\end{itemize}
\end{frame}

\subsection{Måling af kompleksitet}%
\label{subsec:label}

\begin{frame}[allowframebreaks]
	\frametitle{Kompleksitet}
	\begin{itemize}
		\item Trods der eksisterer sprog som vi kan afgøre i teori, kan vi ofte ikke afgøre dem \textit{i praksis}, grundet deres kompleksitet (køretid.)
		\item Vi vil kigge på sproget $A = \{0^{k}1^{k} \mid k \ge 0\}$.
		\item Vi laver en algoritme, og finder dens køretid.
		\item $M_{1} = $''På input streng $w$:
		      \begin{enumerate}
			      \item Scan henover båndet og \textit{afvis} hvis et 0 findes til højre af et 1. (Altså hvis den ser et 1, og så et 0).
			      \item Gentag hvis der stadig er 0 og 1 på båndet:
			            \begin{enumerate}
				            \item Scan henover båndet og afkryds hvert 0 og hvert 1
			            \end{enumerate}

			      \item Hvis der stadig er 0'er tilbage efter 1 er blevet krydset af, eller omvendt, så \textit{afvis}. Ellers \textit{accepter}.
		      \end{enumerate}

		\item Køretiden af en algoritme kan være afhængig af flere parametre.
		\item Hvis algoritmen kører på en graf, kan den for eksempel være afhængig af antallet af knuder og kanter.
		\item I \textit{worst-case køretid} kigger vi på hvad det værst mulige tilfælde er, givet alle inputs af en længde (ofte $n$).
	\end{itemize}
	\begin{definition}
		Lad $M$ være en deterministisk TM som standser på alle input. \textit{Køretiden} eller \textit{tidskompleksiteten} af $M$ er funktionen $f : \mathcal{N} \rightarrow \mathcal{N}$, hvor $f(n)$ er maksimumsantallet af skridt som $M$ tager på et input af længde $n$. Hvis $f(n)$ er køretiden af $M$, så siger vi at $M$ kører i tid $f(n)$ og at $M$ er en $f(n)$-tids Turingmaskine. Vi bruger normalvis $n$ til at repræsentere længden af et input.
	\end{definition}
\end{frame}

\begin{frame}[allowframebreaks]
	\frametitle{Store-O og lille-O notation}
	\begin{itemize}
		\item Den præcise køretid af en algoritme er ofte meget kompleks.
		\item Derfor bruger vi \textit{asymptotisk analyse}.
		\item Ved asymptotisk notation kigger vi ikke på koefficienter og kun ``highest order term'' af udtrykket af køretiden. Så for eksempel bliver $2n+3$ bare til $n$.
		\item Et andet eksempel er $f(n) = 6n^{3} + 2n^{2}+20n+45$. Dette bliver til $f(n) = O(n^{3})$.
		\item Asymptotisk notation er også det der kalders store-$O$ notation.
	\end{itemize}

	\begin{definition}
		Lad $f$ og $g$ være funktioner $f,g : \mathcal{N} \rightarrow \mathcal{R}^{+}$. Vi siger at $f(n) = O(g(n))$ hvis positive heltal $c$ og $n_{o}$ eksisterer således at for hvert heltal $n \ge n_{0}$,
		\begin{equation}
			f(n) \le cg(n)
		\end{equation}
		Når $f(n) = O(g(n))$, siger vi at $g(n)$ er et \textit{upper bound}, eller \textit{øvre grænse} for $f(n)$, eller mere præcist at $g(n)$ er en asymptotisk øvre grænse for $f(n)$.
	\end{definition}
	\begin{itemize}
		\item Intuitivt betyder $f(n) = O(g(n))$ at $f$ er mindre end eller lig med $g$.
		\item Bemærk også her at logaritmens base er undertrykket, så $\log_{2}$ bliver bare til $O(\log)$
		\item Det vil altså sige at $f_{2}(n) = 3n \log_{2} n + 5n \log_{2}\log_{2}n+2 = O(n \log n)$.
		\item Store-$O$ kan også findes i arimetiske udtryk såsom $f(n) = O(n^{2})+O(n)$ bemærk dog at dette blot er lig med $O(n^{2})$ da $n^{2} > n$.
		\item Ved $f(n) = 2^{O(n)}$ er dette et udtryk for $2^{cn}$, for en konstant $c$.
		\item Lille-$o$ notation siger at en funktion er asymptotisk mindre end en anden.
		\item Analogt til store-$O$ og lille-$O$ er hhv. $\le$ og $<$.
	\end{itemize}

	\begin{definition}
		Lad $f$ og $g$ være funktionerne $f, g : \mathcal{N} \rightarrow \mathcal{R}^{+}$. Say that $f(n) = o(g(n))$ hvis
		\begin{equation*}
			\lim_{n \rightarrow \infty} \frac{f(n)}{g(n)} = 0
		\end{equation*}
		I andre ord betyder $f(n) = o(g(n))$ at for alle reelle tal $c > 0$, eksisterer der et tal $n_{0}$ hvor $f(n) < cg(n)$ for alle $n \ge n_{0}$
	\end{definition}

	\begin{itemize}
		\item Følgende er eksempler af ligninger med lille-$o$:
		\item $\sqrt{n} = o(n)$
		\item $n = o(n \log \log n)$
		\item $n \log \log n = o(n \log n)$
		\item $n \log n = o(n^{2})$
		\item $n^{2} = o(n^{3})$
	\end{itemize}
\end{frame}

\begin{frame}[allowframebreaks]
	\frametitle{Analyse af Algoritmer}
	\begin{itemize}
		\item Nu vil vi faktisk til at analysere algoritmen vi konstruerede tidligere.
		\item $M_{1} = $''På input streng $w$:
		      \begin{enumerate}
			      \item Scan henover båndet og \textit{afvis} hvis et 0 findes til højre af et 1. (Altså hvis den ser et 1, og så et 0).
			      \item Gentag hvis der stadig er 0 og 1 på båndet:
			            \begin{enumerate}
				            \item Scan henover båndet og afkryds hvert 0 og hvert 1
			            \end{enumerate}

			      \item Hvis der stadig er 0'er tilbage efter 1 er blevet krydset af, eller omvendt, så \textit{afvis}. Ellers \textit{accepter}.
		      \end{enumerate}

		\item Vi analysere hver af dens fire stadier seperat.
		      \begin{enumerate}
			      \item At scanne tager $n$ skridt, da længden af strengen er $n$. Vi skal også tilbage igen, og dette tager $n$ skridt. I alt $2n$. (Husk at $2n = O(n)$)
			      \item[2-3] Hvert scan her bruger $O(n)$ skridt. Fordi hver scan afkrydser to symboler, er det højets $n/2$ scanninger (fordi vi krydser to af hver gang.) Dermed er tiden taget af stadier 2-3 $(n/2)O(n) = O(n^{2})$ skridt.
			      \item[4] Tiden den bruger på at finde ud af dette er \textit{højest} $O(n)$.
		      \end{enumerate}
		\item Dermed er tiden $O(n) + O(n^{2})+O(n) = O(n^{2})$
	\end{itemize}

	\begin{definition}
		Lad $t : \mathcal{N} \rightarrow \mathcal{R}^{+}$ være en funktion. Vi definerer \textit{tidskompleksitetsklassen} $TIME(t(n))$ til at være samlingen af alle sprog der er afgjort af en Turingmaskine som kører i $O(t(n))$ tid.
	\end{definition}
	\begin{itemize}
		\item Det vil altså sige at algoritmen $A$ vi analyserede før er en del af $TIME(n^{2})$, altså $A \in TIME(n^{2})$.
		\item Jeg har udeladt et bevis for at denne algoritme kan afgøres i $O(n \log n)$ tid, men en vigtigt takeaway fra dette er at alle sprog der kan afgøres i $o(n \log n)$ (bemærk lille-$o$!) er regulære.
		\item Church-turing tesen (som er beregnelighedsteori) siger at alle ``reasonable'' modeller af komputering er ækvivalente.
		\item I kompleksitetsteori bestemmer modelvalget kompleksiteten på et sprog.
		\item For eksempel kan $A$ afgøres i $O(n)$ tid på en 2-bånds Turingmaskine.
	\end{itemize}
\end{frame}

\begin{frame}[allowframebreaks]
	\frametitle{Kompleksitetsforholdet mellem modeller}

	\begin{itemize}
		\item Vi vil nu kigge på hvordan valget af en model kan påvirke køretiden af sprog.
		\item Vi kigger på tre modeller: enkeltbåndsmaskinen, multibåndsmaskionen og den nondeterministiske maskine.
	\end{itemize}

	\begin{theorem}
		Lad $t(n)$ være en funktion hvor $t(n) \ge n$. Så for hver $t(n)$-tids multibånds Turingmaskine, er der en ækvivalent $O(t^{2}(n))$-tids enkelt-bånds Turingmaskine.
	\end{theorem}
	\begin{itemize}
		\item Vi vil vise dette ved at analysere algoritmen vi brugte tidligere til at konvertere fra multibånds til enkeltbånds.
		\item Lad $M$ være en $k$-bånds TM der kører i $t(n)$ tid.
		\item Lad $S$ være en enkeltbånds TM der kører i $O(t^{2}(n))$ tid.
		\item $S$ simulerer $M$ som beskrevet (meget) tidligere.
		\item Til at starte med tager $S$ sit bånd i det format der repræsenterer alle bånd af $M$ og simulerer derefter $M$'s skridt.
		\item For at simulere et skridt scanner $S$ al informationen for at bestemme informationerne under symbolerne ved $M$'s båndhoveder.
		\item $S$ laver så en til scanning over dets bånd til at opdatere båndindholdet og hovedpositioner.
		\item Hvis en af hovederne går til højre mere end der er plads til, rightshifter vi.
		\item For hvert skridt af $M$, laver $S$ to scanninger over den aktive del af dens bånd. Den første scanning får informationen nødvendigt til at bevæge sig, og den anden scanning udfører denne bevægelse.
		\item Hvert bånd i $M$ har længde højest $t(n)$, da det er muligt at hovedet bevæger sig til højre for hvert skridt.
		\item Dermed bruger en scanning af den aktive del af $S$' bånd højest $O(t(n))$ skridt.
		\item For at simulerer hvert af $M$'s skridt, laver $S$ to scanninger og op til $k$  højreshifts, hver bruger $O(t(n))$ tid, så den endelige tid til ét skridt et $O(t(n))$.
		\item Vi sætter nu en grænse på den endelige tid af hele $S$:
		      \begin{enumerate}
			      \item Først tager 4S båndet i den rigtige format. Dette bruger $O(n)$ tid.
			      \item Så simulerer $S$ hvert af de $t(n)$ skridt af $M$, dette bruger $O(t(n))$ tid. Denne del af simuleringen bruger altså $t(n) \times O(t(n)) = O(t^{2}(n))$ trin.
			      \item Dermed er hele simuleringen $O(n) + O(t^{2}(n))$ skridt.
			      \item $O(n) + O(t^{2}(n)) = O(t^{2}(n))$
		      \end{enumerate}

		\item Vi vil nu kigge på nondeterministisk Turingmaskines køretid og en simulerende køretid.
	\end{itemize}
	\begin{definition}
		Lad $N$ være en NDTM som er afgører. \textit{Køretiden} af $N$ er funktionen $f : \mathcal{N} \rightarrow \mathcal{N}$, hvor $f(n)$ er det maksimum antal af skridt som $N$ bruger på en gren af dens komputering, på en længde $n$.
	\end{definition}

	\begin{center}
		\includegraphics[scale=0.3]{figur/figur710.png}
	\end{center}

	\begin{theorem}
		Lad $t(n)$ være en funktion, hvor $t(n) \ge n$. Så for hver $t(n)$ tids NDTM enkeltbånds Turingmaskine, har den ækvivalent $2^{O(t(n))}$-tids (D)TM på et enkelt bånd.
	\end{theorem}

	\begin{itemize}
		\item På en input længde $n$, så har hver gren af $N$'s træ længde højest $t(n)$.
		\item Hver knude i træet har højest $b$ børn, hvor $b$ er det maskimale antal af lovlige valg af $N$'s overføringsfunktion.'s overføringsfunktion.
		\item Dermed er det endelige antal af blade i træet højest $b^{t(n)}$
		\item Simuleringerne går videre ved at udforske træet i BFS.
		\item Antallet af knuder i træet er mindre end det dobbelte af antallet af blade, så vi kan sætte en grænse $O(b^{t(n)})$.
		\item Tiden det tager at gå fra roden til en knude er $O(t(n))$.
		\item Køretiden er altså $O(t(n)b^{t(n)}) = 2^{O(t(n))}$.
	\end{itemize}
\end{frame}

\subsection{Klassen $P$}%
\label{subsec:label}

\begin{frame}[allowframebreaks]
	\frametitle{Klassen $P$}
	\begin{itemize}
		\item Vi så tidligere at en multibånds TM kan konverteres til en enkeltbånds TM der kører i $O(n^{2})$ tid, altså polynomiel.
		\item Vi så også at en NDTM kan konverteres til en TM der kører i $2^{O(n)}$ tid, som altså er eksponentielt.
		\item Vi differentierer ofte algoritmer mellem at være \textit{eksponentielle} og \textit{polynomielle}, hvor vi synes at de polynomielle algoritmer er en del hurtigere end de eksponentielle.
		\item For eksempel polynomiet $n^{3}$ og eksponenten $2^{n}$. Hvis $n = 1000$ så er $n^{3} \approx 1 000 000 000$, hvorimod $2^{n}$ er langt større end antallet af atomer i universet (mellem $10^{78}$ og $10^{82}$).
		\item Vi får ofte eksponentielle algoritmer ved at lave brute-force løsninger til problemer.
	\end{itemize}

	\begin{definition}
		$P$ er klassen af sprog som af afgørlige i polynomiel tid på en deterministisk enkeltbånds Turingmaskine.
		\begin{equation}
			P = \bigcup_k TIME(n^{k})
		\end{equation}
	\end{definition}

	\begin{itemize}
		\item Klassen $P$ er vigtig fordi:
		      \begin{enumerate}
			      \item $P$ er en invariant for alle modeller der er polynomielt ækvivalente til en enkeltbånds TM
			      \item $P$ er nogenlunde det vi tænker på som klassen af problemer vi realistisk set kan løse på en computer
		      \end{enumerate}
	\end{itemize}
\end{frame}

\begin{frame}[allowframebreaks]
	\frametitle{Eksempler på problemer i $P$}
	\begin{itemize}
		\item Når vi repræsenterer problemerne i kodning (\(\langle M \rangle\)), antager vi at denne kodning er rimelig, og kan blive sat om til en intern repræsentation i polynomiel tid.
		\item Vi vil i disse eksempler kigge meget på grafer.
		\item En rimelig kodning af en graf er en liste af dens knuder og kenter.
		\item En anden rimelig kodning (som også oftest er brugt i programmering) er en \textit{adjacency matrix}, hvor den $(i,j)$'e ``entry'' er $1$ hvis der er en kant fra knude $i$ til knude $j$, og 0 ellers.
		\item Det første problem omhandler rettede grafer, nemlig $PATH$ problemet.
	\end{itemize}

	\begin{equation}
		PATH = \{\langle G, s, t \rangle \mid G \text{ er en rettet graf med en rettet vej fra } s \text{ til } t\}
	\end{equation}

	\begin{theorem}
		$PATH \in P$
	\end{theorem}

	\begin{itemize}
		\item Vi beviser ved at konstruere en algoritme der kører i polynomiel tid. Først kigger vi dog på brute-force løsningen, som vi kan konkludere ikke er hurtig nok.
		\item En Brute-force løsning kigger alle mulige veje og tjekker om nogen af dem er rettet fra $s$ til $t$.
		\item En vej kan længst have længde $m$ hvor $m$ er antallet af knuder. ($m$ fordi det aldrig vil være nødvendigt at gentage en knude.)
		\item Antallet af potentielle veje er cirka $m^{m}$, som er eksponentielt baseret på antal knuder i $G$.
		\item Altså er brute-force algoritmen $\notin P$.
		\item Hvis vi i stedet kører en form for BFS får vi en algoritme der kører i polynomiel tid:
		\item $M =$''På input \(\langle G, s, t \rangle \) hvor $G$ er en rettet graf med knuder $s$ og $t$.
		      \begin{enumerate}
			      \item Markér knude $s$
			      \item Gentag følgende indtil der ikke er flere knuder der bliver markeret:
			            \begin{enumerate}
				            \item Scan alle knuder af $G$. Hvis en kant $(a,b)$ er fundet ved at gå fra en markeret knude $a$ til en markeret knude $b$, markér knude $b$.
			            \end{enumerate}
			      \item Hvis $t$ er markeret, \textit{accepter} og ellers \textit{afvis}.
		      \end{enumerate}
		\item Hvis $t$ markeres er der fundet en vej fra $s$ til $t$.
		\item Stadierne 1 og 3 kører kun en enkelt gang.
		\item Stadie 2 kører højest $m$ gange, fordi hver gang tager den en til knude.
		\item Dermed er antallet af stadier brugt højest $m+2$, hvilket er polynomielt i $G$.
		\item Vi kigger nu på problemet om \textit{relative primtal} (også kaldet inbyrdisk primiske).
		\item To tal er indbyrdisk primiske hvis 1 er det største heltal som ligeligt deler dem begge.
		\item For eksempel er 10 og 21 indbyrdisk primiske, selvom ingen af dem er primtal. 10 og 22 er dog ikke indbyrdisk primiske fordi begge er delelige med $2$.
		\item Vi definerer nu problemet $RELPRIME$.
	\end{itemize}

	\begin{equation}
		RELPRIME = \{\langle x, y \rangle \mid x \text{ og } y \text{ er indbyrdisk primiske}\}
	\end{equation}

	\begin{theorem}
		$RELPRIME \in P$
	\end{theorem}

	\begin{itemize}
		\item En måde at løse dette på er ved at gå alle tal igennem, og se om de er delelige med tallene. Dette er dog eksponentielt i sin køretid.
		\item I stedet bruger vi Euklids algoritme, som vi havde om i Diskret Matematik.
		\item Denne kører polynomielt.
		\item Følgende er Euklids algoritme:
		\item $E = $''På input \(\langle x,y\rangle\), hvor $x$ og $y$ er naturlige tal i binær:
		      \begin{enumerate}
			      \item Gentag indtil $y = 0$:
			            \begin{enumerate}
				            \item[2] $x \leftarrow x \mod y$
				            \item[3] Byt $x$ og $y$
			            \end{enumerate}
			      \item[4] Output $x$''
		      \end{enumerate}
		\item Vi bruger $E$  som subrutine i algoritmen $R$:
		\item $R = $''På input \(\langle x, y \rangle\), hvor $x$ og $y$ er naturlige tal i binær:
		      \begin{enumerate}
			      \item Kør $E$ på \(\langle x, y \rangle \)
			      \item Hvis resultatet er $1$, \textit{accepter}, ellers \textit{afvis}.''
		      \end{enumerate}

		\item Vi vil nu vise at $E$ og dermed $R$ er polynomielt.
		\item Stadie 2 af $R$ skærer værdien mindst halvt.
		\item Efter stadie 2, er $x < y$.
		\item Efter stadie 3 er $x > y$, da vi har byttet værdierne.
		\item Når stadie 2 eksekveres er $x > y$.
		\item Hvis $x/2 < y$, så $x \mod y = x - y < x / 2$, og $x$ ryger mindst ned til halvdelen
		\item De originale værdier af $x$ og $y$ byttes hver gang stadie 3 kører.
		\item Dermed er maksimums antallet af gangene stadie 2 og 3 køres den $\min\{ 2 \log_{2} x, 2 \log_{2} y\}$
		\item Disse logaritmer er propertionelle til længderne af repræsentationerne.
		\item Hvert stadie i $E$ bruger polynomiel tid, og da antallet af stadier der køres er $O(n)$ er den endelige køretid polynomiel.
		\item Sidst kigger vi på problemet om at alle kontekstfrie sprog er afgørlige i polynomiel tid.
	\end{itemize}

	\begin{theorem}
		Hvert kontekstfrit sprog $\in P$.
	\end{theorem}

	\begin{itemize}
		\item Algoritmen vi konstruerede tidligere til at bevise at alle kontekstfrie sprog var afgørlige er \textbf{ikke} nok her!
		\item Den algoritme forsøger alle afledninger med $2n-1$ skridt, når dens input er en streng af længde 4n.
		\item Antallet af afledninger med $k$ skridt kan være eksponentielt i $k$.
		\item For at få en algoritme der kører i polynomiel tid bruger vi \textit{dynamic programming}.
		\item Dynamic programming holder på løsningsinformation og deler problemer op i mindre problemer.
		\item Vi gør dette ved at lave en table af alle delproblemer, og putter deres løsninger ind systematisk, som vi finder dem.
		\item I det her tilfælde er delproblemerne om hver variabel i $G$ genererer hver delstreng i $w$.
		\item Algoritmen putter løsningen ind i en tabel af størrelse $n \times n$.
		\item Hvor $i \le j$, er den $(i,j)$'e ``entry'' i tabellen samlingen af variabler der genererer delstringen $w_{i}w_{i+1}\ldots w_{j}$. Hvor $i > j$ er ``entriesne'' ikke brugt.
		\item Algoritmen fylder hver entry for hver delstreng af $w$.
		\item Først fylder den for delstrengene af længde 1, så længde 2, etc.
		\item Den bruger disse entries til de mindre længder til at hjælpe med at bestemme entries i længere længder.
		\item Antag for eksempel at algoritmen allerede har fundet ud af hvilke variabler genererer alle delstrenge op til længde $k$.
		\item For at bestemme om en variabel $A$ genererer to delstrenge af længde $k+1$, splitter algoritmen edn delstreng i to dele på de $k$ mulige måder.
		\item For hvert split, kigger algoritmen på reglen $A \rightarrow BC$, for at bestemmet om $B$ genererer den første del og $C$ den anden del.
		\item Hvis både $B$ og $C$ genererer hhv. første og anden del, så genererer $A$ delstrengen og kan blive tilføjet til tabellen.
		\item Algoritmen starter processen med strenge af længde $1$ ved at kigge efter regler $A \rightarrow b$
		\item Den følgende algoritme implementerer idéen præsenteret tidligere.
		\item Lad $G$ være en Chomsky grammatik der genererer CFL'et $L$.
		\item Antag at $S$ er startvariablen.
	\end{itemize}
	\includegraphics[scale=0.3]{figur/polycfg.png}
	\begin{itemize}
		\item Stadierne $4$ og $5$ kører højest $nv$ gange, hvor $v$ er antallet af variabler.
		\item Disse stadier kører altså $O(n)$ gange
		\item Stadie 6 kører højest $n$ gange.
		\item Hver gang stadie 6 kører, kører stadie 7 højest $n$ gange.
		\item Hver gang stadie 7 kører, kører stadierne 8 og 9 højest $n$ gange.
		\item Hver gang stadie 9 kører, kører stadie 10 $r$ gange, hvor $r$ er antallet af regler
		\item Stadie 3, den indre løkke, kører $O(n^{3})$ gange.
		\item $O(n) + O(n) + \ldots + O(r) + O(n^{3}) = O(n^{3})$
	\end{itemize}
\end{frame}

\subsection{Klassen $NP$}%
\label{subsec:label}

\begin{frame}[allowframebreaks]
	\frametitle{Klassen $NP$}
	\begin{itemize}
		\item Nogen problemer kender vi ikke polynomielle løsninger til.
		\item Et eksempel på et sådan problem er $HAMPATH$ problemet.
	\end{itemize}

	\begin{align*}
		HAMPATH = & \{\langle G, s, t \rangle \mid G \text{ er en rettet graf med en HAMPTH fra } \\
		          & s \text{ til } t\}
	\end{align*}

	\begin{itemize}
		\item En hamiltoniansk vej er en rettet vej som går gennem hver knude én gang.
		\item Trods der ikke eksisterer en polynomiel løsning til Hampath problemet, kan vi verificere løsningen i polynomiel tid.
		\item Altså kan vi overbevises om at en løsning til problemet er en korrekt løsning i polynomiel tid.
	\end{itemize}

	\begin{definition}
		En \textit{verifikator} for et sprog $A$ er en algoritme $V$, hvor
		\begin{equation*}
			A = \{w \mid V \text{ accepterer } \langle w, c \rangle \text{ for en streng }c\}
		\end{equation*}

		Vi regner tiden for en verifikator kun ud fra længden af $w$, så en verifikator i polynomiel tid kører i polynomiel tid i længden af $w$. Et sprog $A$ er verificerbart i polynomiel tid hvis det har en verifikator der kører i polynomiel tid.
	\end{definition}

	\begin{itemize}
		\item $c$ kaldes for et \textit{certifikat} eller \textit{bevis} af medlemskab i $A$.
		\item For verifikatorerer i polynomiel tid, har certifikatet også polynomiel længde i $w$, fordi det er alt verifikatoren kan nå.
		\item For eksempel, ved HAMPATH problemet er et certifikat en vej fra $s$ til $t$.
		\item
	\end{itemize}
	\begin{definition}
		$NP$ er klassen af sprog som har verifikatorer i polynomiel tid.
	\end{definition}
	\begin{itemize}
		\item Denne definition klargører også at $P \subseteq NP$.
		\item Følgende er en NDTM der afgører HAMPATH problemet i polynomiel tid:
		\item $N_{1} = $''På input \(\langle G, s, t \rangle\), hvor $G$ er en rettet graf med knuder $s$ og $t$:
		      \begin{enumerate}
			      \item Skriv en liste af $m$ tal $p_{1}, \ldots, p_{m}$, hvor $m$ er antallet af knuder i $G$. Hver tal i listen er nondeterministisk valgt til at være mellem $1$ og $m$.
			      \item Se om der er nogle ens tal, hvis der er \textit{afvis}
			      \item Tjek om $s = p_{1}$ og $t = p_{m}$. Hvis ikke \textit{afvis}.
			      \item For hvert $i$ mellem $1$ og $m-1$, tjek om $(p_{i}, p_{i+1})$ er en kant i $G$. Hvis ikke, \textit{afvis}. Ellers \textit{accepter}.
		      \end{enumerate}
	\end{itemize}

	\begin{theorem}
		Et sprog $A$ er i $NP$ $\iff$ $A$ er afgjort  af en NDTM i polynomiel tid.
	\end{theorem}

	\begin{itemize}
		\item Vi viser hvordan man konverterer en verifikator i polynomiel tid, til en NDTM i polynomiel tid, og omvendt.
		\item NDTM'en simulerer verifikatoren ved at gætte på certifikatet.
		\item Verifikatoren simulerer NDTM'en ved at bruge den accepterende gren som certifikat.
		\item \(\Rightarrow\) Lad $A \in NP$. Lad $V$ være vertifikatoren i polynomiel tid for $A$. Antag at $V$ er en Turingmaskine der kører i tid $n^{k}$ og konstruer $N$:
		\item $N =$''På input $w$ af længde $n$:
		      \begin{enumerate}
			      \item Nondeterministisk vælg streng $c$ af længde højest $n^{k}$
			      \item Kør $V$ på input $\langle w, c \rangle $
			      \item Hvis $V$ accepterer, \textit{accepter}, ellers \textit{afvis}.
		      \end{enumerate}

		\item \(\Leftarrow\) Antag at $A$ er afgjort af en NDTM i polynomiel tid $N$ og konstruer en verifikator i polynomiel tid $V$ som følger:
		\item $V = $''På input \(\langle w, c \rangle  \)hvor $w$ og $c$ er strenge:
		      \begin{enumerate}
			      \item Simuler $N$ på input $w$, og lad hvert symbol af $c$ være beskrivlsen af det nondeterministisk valg lavet på hvert skridt.
			      \item Hvis denne gren af $N$ accepterer, så \textit{accepter}, ellers \textit{afvis}.
		      \end{enumerate}
	\end{itemize}

	\begin{definition}
		\begin{align*}
			NTIME(t(n)) = \{ & L \mid L \text{ er et sprog afgjort af en } O(t(n)) \\ &\text{ tids NDTM }\}
		\end{align*}
	\end{definition}

	\begin{corollary}
		\begin{equation}
			NP = \bigcup_k NTIME(n^{k})
		\end{equation}
	\end{corollary}
\end{frame}

\begin{frame}[allowframebreaks]
	\frametitle{Eksempler på problemer i $NP$}

	\item \textbf{Klikke-problemet}
	\item En \textbf{klikke} i en urettet graf, er en delgraf hvor hver to knuder er forbundet af en kant.
	\item En $k-$klikke er en klikke der indeholder $k$ knuder.

	\begin{equation}
		CLIQUE = \{\langle G, k \rangle \mid G \text{ er en urettet graf med en }k-\text{klikke}\}
	\end{equation}

	\begin{theorem}
		$CLIQUE \in NP$
	\end{theorem}
	\begin{itemize}
		\item Vi viser to beviser, en for certifikat, og en for NDTM.
		\item Følgende er en verifikator $V$ til clique:
		\item $V = $''På input \(\langle \langle G, k \rangle , c \rangle \)
		      \begin{enumerate}
			      \item Test om $c$ er en delgraf med $k$ knuder i $G$
			      \item Test om $G$ indeholder alle kanter der forbinder knuder i $c$
			      \item Hvis begge er ``ja'', \textit{accepter}, ellers \textit{afvis}
		      \end{enumerate}
		\item Følgende er en NDTM der afgører $CLIQUE$:
		\item $N = $''På input $\langle G , k \rangle $ hvor $G$ er en graf:
		      \begin{enumerate}
			      \item Nondeterministisk vælg en delmængde $c$ af $k$ knuder i $G$
			      \item Test om $G$ indeholder alle kanter der forbinder knuder i $c$
			      \item Hvis ``ja'', \textit{accepter}, ellers \textit{afvis}.
		      \end{enumerate}

		\item Vi vil nu kigge på \textit{SUBSET-SUM} problemet.
		\item Givet en samling af tal $x_{1}, x_{2}, \ldots, x_{k}$ og et tal $t$, vil vi finde ud af om samlingen indeholder en delsamling som summerer op til $t$.
		      \begin{align*}
			      \text{SUBSET-SUM} = \{\langle S , t \rangle \mid S = \{x_{1}, \ldots, x_{k}\}, \\
			      \text{og en} \{y_{1}, \ldots, y_{l}\} \subseteq \{x_{1}, \ldots, x_{k}\}, \text{har vi} \sum{y_{i}}= t\}
		      \end{align*}
		\item Her er mængderne \textit{multisets} (multimængde?), så det er tilladt at der er repetition af elementerne.
	\end{itemize}

	\begin{theorem}
		SUBSET-SUM $\in NP$
	\end{theorem}

	\begin{itemize}
		\item Vi viser igen to beviser, et for certifikatet, og et for en NDTM.
		\item Lad $V$ være en verifikator for Subset-Sum:

		\item $V =$''På input \(\langle \langle S, t \rangle, c \rangle\):
		      \begin{enumerate}
			      \item Test om $c$ er en samling af tal som summer op til $t$.
			      \item Test om $S$ indeholder alle tal i $c$.
			      \item Hvis begge ``ja'' \textit{accepter}, og ellers \textit{afvis}.
		      \end{enumerate}
		\item $N =$''På input \(\langle S, t \rangle\):
		      \begin{enumerate}
			      \item Vælg nondeterministisk en delmængde $c$ af tallene i $S$.
			      \item Test om $c$ er en samling af tal der summerer op til $t$.
			      \item Hvis ja, \textit{accepter}, ellers \textit{afvis}.
		      \end{enumerate}
		\item Bemærk her at trods både \textit{clique} og \textit{subset-sum} er i $NP$ er deres komplementer \textbf{ikke}.
		\item Vi laver en seperat kompleksitetsklasse $coNP$ som nidehodler sprogene som er komplementer af sprog i $NP$.
		\item Vi ved ikke om $coNP = NP$
	\end{itemize}
\end{frame}

\subsection{NP-Komplethed}%
\label{subsec:npcompleteness}

\begin{frame}[allowframebreaks]
	\frametitle{NP-Komplethed}
	\begin{itemize}
		\item I 70'erne opdagede Stephen Cook og Leonid Levin (uafhængigt af hinanden) at der er problemer i $NP$ hvis individuelle kompleksitet relateres til hele klassen.
		\item Altså, hvis man finder en polynomiel løsning til en af disse problemer, finder man en polynomiel løsning til hele klassen, og dermed $P = NP$, og omvendt.
		\item Disse siges at være $NP$-Komplette.
		\item Hurtigt recap af SAT-problemet, som er $NP$-komplet:
		      \begin{itemize}
			      \item Boolske variabler: Værdier 0 eller 1
			      \item Boolske operationer: AND $\land$, OR $\lor$, NOT $\neg$ (eller $\overline{x}$):
			      \item Boolsk formel: Et udtryk med boolske variabler og operationer, e.g. $\phi = (\overline{x} \lor y) \land (x \land or \overline{z})$
			      \item Et boolsk udtryk er \textit{satisfiable} hvis en tildeling af værdier til variablerne gør at $\phi$ evaluerer til $1$.
		      \end{itemize}

		\item Satisfiability problemet tester om en boolsk formel er satisfiable:
		      \begin{equation*}
			      SAT = \{\langle \phi \rangle \mid \phi \text{ er en satisfiable Boolsk formel}\}
		      \end{equation*}
		\item Det betyder altså, at ud fra vores definitioner af NP-Komplet og SAT:
	\end{itemize}
	\begin{theorem}
		$SAT \in P \iff P = NP$
	\end{theorem}
\end{frame}

\begin{frame}[allowframebreaks]
	\frametitle{Reducerbarhed i Polynomiel tid}
	\begin{itemize}
		\item Ligesom vi så tidligere med at vi kunne reducere fra et problem til en anden i beregnelighedsteori, så kan vi gøre det samme i kompleksitetsteori.
		\item Her siger vi at hvis et problem $A$ kan \textit{effektivt} reduceres til problem $B$, kan en effektiv løsning til $B$ bruges til at løse $A$ effektivt.
	\end{itemize}

	\begin{definition}
		En funktion $f : \Sigma^{*} \longrightarrow \Sigma^{*}$ er en polynomieltids beregnelig funktion  hvis en Turingmaskine $M$ der kører i polynomiel tid eksisterer som standser med $f(w)$ på sit bånd, når den er startet på et input $w$.
	\end{definition}

	\begin{definition}
		Sprog $A$ er polynomieltids mapping reducerbart, eller simpelt polynomielttids reducerbart (eller reducerbart i polynomiel tid) til sprog $B$, skrevet $A \le_{P} B$, hvis en beregnelig funktion i polynomiel tid $f : \Sigma^{*} \longrightarrow \Sigma^{*}$ eksisterer, hvor
		\begin{equation*}
			\forall w : w \in A \iff f(w) \in B
		\end{equation*}
		Funktionen $f$ kaldes reduktionen i polynomiel tid af $A$ til $B$.
	\end{definition}

	\begin{theorem}
		Hvis $A \le_{P} B$ og $B \in P$, så $A \in P$
	\end{theorem}

	\begin{itemize}
		\item Vi beviser.
		\item Lad $M$ være en algoritme der kører i polynomiel tid og afgører $B$, og lad $f$ være reduktionen i polynomiel tid fra $A$ til $B$. VI beskriver en algoritme $N$ der kører i polynomiel tid som afgører $A$:
		\item $N =$''På input $w$:
		      \begin{enumerate}
			      \item Udregn $f(w)$
			      \item Kør $M$ på input $f(w)$ og output hvad $M$ outputter.''
		      \end{enumerate}
		\item Fordi $f$ er en reduktion fra $A$ til $B$ så er $w \in A$ når $f(w) \in b$.
		\item Dermed accepterer $M$ $f(w)$ når $w \in A$.
		\item Derudover kører $N$ i polynomiel tid, fordi hver af sine to stadier kører i polynomiel tid.
		\item Bemærk at stadie 2 kører i polynomiel tid fordi to polynomier sammensat er et nyt polynomie.
		\item Vi vil nu kigge på en reduktion fra $3$-SAT til CLIQUE.
		\item Husk at $3$-SAT problemet er et satisfiable problem hvor alle klausuler har tre literals. E.g.:
		      \begin{equation}
			      (x_{1} \lor \overline{x_{2}} \lor \overline{x_{3}}) \land (x_{3} \lor \overline{x_{5}} \lor x_{6})
		      \end{equation}
		\item Lad $\text{3-SAT} = \{\langle \phi  \rangle \mid \phi \text{ er en satisfiable 3-CNF formel}\}$
		\item CNF betyder conjunctive normal form, og er den form set over.
	\end{itemize}

	\begin{theorem}
		3SAT er polynomielt reducerbart til CLIQUE
	\end{theorem}

	\begin{itemize}
		\item Vi vil i reduktionen konvertere formlerne til grafer.
		\item I de konstruerede grafer korresponderer en klikke af en specifik størrelse til en satisfying tildeling over formlen.
		\item Lad $\phi$ være en formel med $k$ klausuler, såsom:
		      \begin{equation}
			      \phi = (a_1 \lor b_1 \lor c_1) \land (a_2 \lor b_2 \lor c_2) \land \cdots \land (a_k \lor b_k \lor c_k).
		      \end{equation}

		\item En reduktion $f$ genererer strengen $\langle G, k \rangle$ hvor $G$ er en urettet graf defineret som følger.
		\item Knuderne i $G$ organiseres i $k$ grupper af de tre knuder som hver kaldes \textit{triples} $t_{1}, \ldots, t_{k}$.
		\item Hver triple svarer til en klausul i $\phi$.
		\item Navngiv hver knude i $G$ med dens korresponderende literal i $\phi$.
		\item Kanterne i $G$  forbinder alle par undtagen to typer:
		      \begin{itemize}
			      \item \textbf{Ingen} kanter mellem knuderne i den samme triple
			      \item \textbf{Ingen} kanter mellem to knuder med uoverensstemmelige navne (e.g. $x_{1}$ og $\overline{x_{1}}$).
		      \end{itemize}

		\item Eksempel ved formlen $\phi = (x_1 \lor \overline{x_1} \lor x_2) \land (\overline{x_1} \lor \overline{x_2} \lor \overline{x_2}) \land (\overline{x_1} \lor x_2 \lor \overline{x_2}).$

		\item Bliver lavet om til:
		      \includegraphics[scale=0.5]{figur/figur733.png}
		\item Husk at en $k$-klikke er en komplet delgraf af størrelse $k$ af en større graf $G$.
		\item Altså hvor alle knuderne er forbundet af kanter.
	\end{itemize}

	\begin{theorem}
		$\phi$  er satisfiable $\iff$ $G$ har en $k$-klikke.
	\end{theorem}

	\begin{itemize}
		\item Antag at $\phi$ har en satisfying tildeling.
		\item \(\Rightarrow\) Ved en sådan tiledeling er der mindst én literal i hver klausul der evalueres til at være sand. (Altså den evalueres til at være sand, Jørgen spurgte mig om dette i DM551 eksamen!)
		\item I hver triple i $G$ vælger vi en knude korresponderende til en sand literal i den tildeling der er satisfying.
		\item Hvis mere end én literal er sand vælger vi arbitrært en af dem.
		\item De knuder vi vælger giver en $k$-klikke.
		\item Antallet af knuder vi har valgt er $k$, fordi vi vælger en fra hver af de $k$ tripler.
		\item Det er en $k$-klikke fordi:
		      \begin{enumerate}
			      \item De kan ikke være fra samme triple, og dermed ikke have en kant mellem sig
			      \item De kan ikke være uoverensstemmelige, da de begge skal have en \textit{sand} sandhedsværdi.
		      \end{enumerate}
		\item \(\Leftarrow\) Antag at $G$ har en $k$-klikk. Ingen af de to knuder kan være i samme triple. Dermed må hver af de $k$ tripler have præcis én af knuderne.
		\item Vi tildeler sandhedsværdier til variablerne i $\phi$ så hver literal  der er en del af klikke knuderne bliver gjort sand.
		\item Dette er altid muligt da der ikke er kanter mellem uoverensstemmelige literale.
		\item Denne tildeling af sandhedsværdier er sand, da der er mindst én sand literal i hver triple (klausul).
		\item Dermed er $\phi$ satisfiable.
	\end{itemize}
\end{frame}

\begin{frame}[allowframebreaks]
	\frametitle{Definitionen af NP-Komplethed}
	\begin{definition}
		Et sprog $B$ er $NP$-komplet hvis det satisfier to betingelser:
		\begin{enumerate}
			\item $B \in NP$
			\item $\forall A \in NP : A \le_{P} B$
		\end{enumerate}
	\end{definition}

	\begin{theorem}
		Hvis $B$ er NP-Komplet og $B \in P$, så er $P = NP$.
	\end{theorem}

	\begin{itemize}
		\item Beviset følger fra definitionen af reducerbarhed i polynomiel tid.
	\end{itemize}

	\begin{theorem}
		Hvis $B$ er NP-Komplet og $B \le_{P} C$ for $C \in NP$ så er $C$ NP-komplet.
	\end{theorem}
	\begin{itemize}
		\item Vi ved allerede at $C \in NP$. Vi skal derfor vise at alle $A \in NP$ er reducerbart i polynomiel tid til $C$.
		\item Fordi $B$ er NP-Komplet, er hvert sprog i $NP$ reducerbart i polynomiel tid til $B$ og $B$ er polynomielt tid reducerbart til $C$.
		\item Hvis $A$ er polynomielt reducerbart til $B$  og $B$ til $C$, så er $A$ til $C$.
	\end{itemize}
\end{frame}

\begin{frame}[allowframebreaks]
	\frametitle{Vertex-Cover problemet}
	\begin{itemize}
		\item Ved en urettet graf $G$, er et \textit{vertex cover} af $G$ en delmængde af knuder hvor hver kant af $G$ rører en af disse knuder.
		      \begin{align*}
			      \text{VERTEX-COVER} & = \{\langle G , k \rangle \mid G \text{ er en urettet graf som har en } \\
			                          & k\text{-knude vertex cover.}\}
		      \end{align*}
	\end{itemize}
	\begin{theorem}
		VERTEX-COVER er NP-Komplet
	\end{theorem}

	\begin{itemize}
		\item Vi viser at 3-SAT er polytids reducerbart til VERTEX-COVER.
		\item Reduktionen konverterer en 3cnf-formel $\phi$ til en graf $G$ og et tal $k$, så $\phi$ er satisfiable når $G$ har et vertex cover med $k$ knuder.
		\item Til at bevise dette bruger vi ``gadgets'', som er små ``dele'', vi vil putte sammen.
		\item Vi kommer til at have to typer af disse ``gadgets'':
		      \begin{itemize}
			      \item Variabel Gadget: En lille graf for hver af variablerne
			      \item Klausul Gadget: En lille graf for hvert klausul
		      \end{itemize}

		\item Variabel gadgetten har to knuder. Givet en literal $v_{i}$, vil der være en knude $v_{i} \in V$  og $\overline{v_{i}} \in G$. Derudover vil $(v_{i}, \overline{v_{i}}) \in E$.
		\item At have denne variabel gadget skal sikre for os at vi kun vælger én af de knuder, fremfor begge.
		\item Givet en klausul $(v_{i} \lor v_{j} \lor v_{k})$ får vi en ``gadget'' hvor $\{v_{i}, v_{j}, v_{k}\} \in G$ og $\{(v_{i}, v_{k}), (v_{j}, v_{k}), (v_{j}, v_{i})\} \in E$. Altså får vi en form for trekant.
		\item Kort terminologi, $G_{V}$ er gadgateten der omhandler variabler og $G_{K}$ er gadgetten med klausuler. Derudover $G_{V} \cup G_{K} = G$
		\item Vi tilføjer nu kanter mellem alle knuder der har samme navn, så knuden $v_{i} \in G_{V}$ og knuden $v_{i} \in G_{K}$ skal have en kant mellem sig $(v_{i}, v_{i}) \in E$.
		\item Hvis en knude blive valgt til vertex-coveret, vil det betyde at sandhedsværdien skal være sand.
		\item Givet en formel $\phi$ der har $m$ variabler og $l$ klausuler, så har grafen $G = (V,E)$ antal knuder $|V| = 2m+3l$.
		\item Lad $k = m+2l$
		\item Følgende eksempel er for $\phi = (x_{1} \lor x_{1} \lor x_{2}) \land (\overline{x_{1}} \lor \overline{x_{2}} \lor \overline{x_{2}}) \land (\overline{x_{1}} \lor x_{2} \lor x_{2})$. For denne er $k = 8 = 2 + 2 \cdot 3$.
		      \includegraphics[scale=0.5]{figur/figur745.png}
		\item Grunden til at $k = m+2l$:
		      \begin{itemize}
			      \item $2l$ = Man skal vælge mindst 2 knuder for at cover hver kant
			      \item $m$ = Vi må kun vælge én af variablerne i variabel gadgetten.
		      \end{itemize}
		\item For at bevise at denne reduktion fungerer skal vi vise at:
		\item $\phi$ er satisfiable $\iff$ $G$ har et vertex cover med $k$ knuder.
		\item Hvis \(\phi\) er satisfiable \(\Rightarrow\) har $G$ et vertex cover med $k$ knuder.
		\item
	\end{itemize}
\end{frame}

%%% Local Variables:
%%% mode: latex
%%% TeX-engine: xetex
%%% TeX-command-extra-options: "-shell-escape"
%%% TeX-master: "main"
%%% End:

\chapter{Cook-Levin Sætning}
\label{chap:cooklevin}



%%% Local Variables:
%%% mode: latex
%%% TeX-engine: xetex
%%% TeX-command-extra-options: "-shell-escape"
%%% TeX-master: "main"
%%% End:

\section{Approximationsalgoritmer}%
\label{sec:Approximationsalgoritmer}

\begin{frame}
  \frametitle{Pensum}
  \begin{itemize}
    \item CLRS 35: \textbf{Approximationsalgoritmer}
    \item Weekly Note 10 (igen)
    \item Video 19-21
  \end{itemize}
\end{frame}



%%% local variables:
%%% mode: latex
%%% tex-engine: xetex
%%% tex-command-extra-options: "-shell-escape"
%%% tex-master: "main"
%%% end:

\section{Nedre Grænser}%
\label{sec:nedregrænser}

\begin{frame}
  \frametitle{Pensum}
  \begin{itemize}
    \item Baase 2.4: \textbf{Informationsteoretiske nedre grænser for sortering ved sammenligninger}
    \item Jørgens noter: \textbf{Noter på nedre grænser}
    \item CLRS 9.3: \textbf{Median Problem}
    \item Weekly Note 11
    \item Video 22-24
  \end{itemize}
\end{frame}



%%% local variables:
%%% mode: latex
%%% tex-engine: xetex
%%% tex-command-extra-options: "-shell-escape"
%%% tex-master: "main"
%%% end:

\section{Adversary Argumenter}%
\label{sec:adversaryargumenter}

\begin{frame}
	\frametitle{Pensum}
	\begin{itemize}
		\item Baase 3: \textbf{Adversary nedre grænse argumenter}
		\item Baase 3.5: \textbf{Adversary Median Problem}
		\item Jørgens noter: \textbf{Noter på nedre grænser}
		\item Weekly Note 12
		\item Video 24-25
	\end{itemize}
\end{frame}

\begin{frame}[allowframebreaks]
  \frametitle{Nedre Grænser for Sammenligningsbaseret Sortering}
  \begin{itemize}
	\item Vi vil gerne bevise at en algoritme mindst vil kunne lave $\Theta(n \log n)$ sammenligninger.
	\item Til at bevise dette bruger vi \textit{decision trees}.
	\item Alle deterministiske sammenligningsbaseret sorteringsalgoritmer $A$  kan associeres med et decision tree.
	\item Et decision tree har en knude som stiller en udtalelse op, e.g. $x_{14} < x_{17}$, og har så to kanter, en for hvis \textit{ja} og en for hvis \textit{nej}.
	\item Disse kanter fører så videre til flere knuder der også har nej/ja kanter, osv. indtil du kommer til bladene.
	\item Det vil også sige at roden af træet er den første sammenligning der bliver lavet.
	\item Hvert blad repræsenterer altså en permutation for hvordan tallene kan se ud.
	\item Det vil sige at alle mulige permutationer er blade, og dermed er der mindst $n!$ blade, for et input af størrelse $n$. Det er også muligt at en permutation kan være i et blad mere end én gang.
	\item Et decision træ er et binært træ, hvilket betyder at for hvert ``niveau'' i træet, bliver antallet af knuder \textbf{højest} fordoblet.
	\item Dermed er der også højest $2^{h}$ blade, hvor $h$ er højden på træet.
	\item Dermed må $2^{h} \ge n!$.
	\item Dermed må $h \ge \log n! \ge n \log n - cn$
	\item Vi får dette fra at $2^{h} \ge n!$, hvor vi kan tage $\log_{2}$ på begge sider og få $\log_{2}(2^{h}) \ge \log_{2}(n!)$ som giver $h \ge \log_{2}(n!)$.
	\item Den anden del, at $\log n! \ge n \log n - cn$ håber jeg ikke vi skal kunne til eksamen, for ChatGPT's bevis bruger både $\pi$ og $e$, og er i øvrigt ultra langt.
	\item Hver vej fra roden til et blad svarer til sammenligninger lavet af $A$ til at sortere et input. Dermed er antallet af sammenligninger lig med længden af vejen.
	\item Dermed bruger $A$ \textbf{mindst} $n \log n - cn$ sammenligninger på et input.
	\item Vi kan også finde en nedre grænse for den gennemsnitlige mængde af sammenligninger via decision trees.
	\item Givet en mængde $P$ som indeholder alle veje fra roden til bladene, kan vi definere $epl$ (externaexternal path length) til at være:
		  \begin{equation*}
			epl = \sum_{p \in P} \text{længde}(p)
		  \end{equation*}
	\item Lemma 2.8 i Baase, siger at $epl$ er minimeret når $T$ er ``almost balanced'', som vil sige at forskellen i distancen mellem rodden og alle blade enten er den samme for alle blade, eller er højest 1.
  \end{itemize}
  \begin{center}
	\includegraphics[scale=0.3]{figur/baaselemma28.png}
  \end{center}
  \begin{itemize}
	\item Beviset givet i Baase siger at hvis et blad $X$ på level $k \le d-2$, så kan vi lave et træ med det samme antal blade, og en mindre $epl$.
	\item  Vi vælger en knude $Y$ på level $d-1$, som ikke er et blad, og fjerner dens børn, og giver så to børn til $X$. Antallet af børn har ikke ændret sig.
	\item EPL-en er blevet nedsat med $2d+k$, fordi vejene til børnene (to børn, hver af længde $d$) af $Y$ og vejen til $X$ (længde $k$) ikke længere er forbundet.
	\item Når vi så tager børnene tilbage til $X$  så får vi en øgning i EPL-en. Da børnene til $X$ vil være på level $k+1$, og der er to, bliver det $2(k+1)$. Da $Y$ forbliver på level $d-1$ er længden $d-1$.
	\item Dermed er den totale øgning i EPL $2(k+1)+(d-1)=2k+2+d-1=2k+d+1$
	\item Vi vil nu gerne finde ændringen, som vi gør ved at fjerne øgningen fra aftagelsen.
	\item Aftagelsen: $2d+k$
	\item Øgningen: $2k+d+1$
	\item $(2d+k)-(2k+d+1) = 2d+k-2k-d-1=d-k-1$
	\item Vi ved at $k \le d-2$, så $d-k-1 > 0 $
	\item Så altså får vi et træ med en lavere EPL.
	\item Lemma 2.9: Minimums EPL-en for et decision træ med $l$ blade er $l \lfloor \log l \rfloor + 2(l-2^{\lfloor \log l\rfloor })$
	\item Hvis $l = 2^{k}$  for en $k$, og alle blade på level $\log l$, så er EPL--en $l \log l$
	\item Hvis $l \ne 2^{k}$ for en $k$, så er dybden af træet $d = \lceil \log l \rceil$ og alle bladene er på level $d-1$ og $d$.
	\item Summen for alle vejlængderne til niveau $d-1$ er $l(d-1)$.
	\item Alle blade på niveau $d$ skal altså tilføje $1$ ekstra til EPL--en.
	\item Antallet af blade på level $d$ er $2(l-2^{d-1})$, da, for hver knude på level $d-1$ som \textbf{ikke} er et blad, er der to blade på level $d$.
	\item Dermed er summen $l(d-1)+2(l-2^{d-1}) = l \lfloor \log l \rfloor + 2(l-2^{\lfloor \log l \rfloor})$.
	\item Grunden til at vi kan sige oprundet $\log l$, er fordi det er det nederste niveau, og $\log l$ vil give os en mellemting. Dermed er nedrundet det niveau der kommer over, e.g. 8 og 9, hvis $\log l = 8.4$. Vi kan kun forvente et heltal hvis $l = 2^{k}$.
	\item Bemærk at hvis $l = 2^{k}$ så vil $2(l-2^{k}) = 2(2^{k}-2^{k}) = 2(0) = 0$
	\item Lemma 2.10: Den gennemsnitlige længe af en vej i et 2-træ med $l$ blade er mindst $\lfloor \log l \rfloor$.
	\item Beviset her er ret nemt:
		  \begin{equation}
\frac{l \lfloor \log l \rfloor + 2(l-2^{\lfloor \log l \rfloor})}{l} + \varepsilon
		  \end{equation}

	\item Hvor \(0 \le \varepsilon < 1\) da $l-2^{\lfloor \log l \rfloor }< \frac{l}{2}$, fordi $l > 2^{\lfloor \log l \rfloor}$
	\item Teorem 2.11: Den gennemsnitlige mængde af sammenligninger lavet af en algoritme til at sortere $n$ elementer er mindst $\lfloor \log n! \rfloor \approx \lfloor n \log n - 1.5n \rfloor$
  \end{itemize}
\end{frame}

\begin{frame}[allowframebreaks]
  \frametitle{Modstander Argumenter for Sortering}
\begin{itemize}
  \item Vores modstander til det her er ``incredibly powerful''.
  \item Vedkommende har en samling af permutationer af listen der skal sorteres, som opdateres efter et nyt valg er lavet i algoritmen.
  \item Dermed kan modstanderen, efter hvert skridt, finde den værst mulige permutation, som er mulig for algoritmen.
  \item Man kan se på en sorteringsalgoritme, som en algoritme der starter med at have alle $n!$ permutationer, og skal eleminere alle undtagen én som så bliver den sorteret permutation.
  \item Vi (Jørgen) introducerer følgende terminologi:
  \item \(\delta_{i}\) = permutationer der stadig er mulige efter den $i$'e sammenligning.
  \item Dermed er \(|\delta_{o}| = n!\)
  \item Hver gang din algoritme stiller spørgsmålet $x < y$? så er modstanderens mål er ``dræbe'' så få permutationer som muligt.
  \item Dermed vil modstanderen gerne svare således at $\frac{|\delta_{i}|}{|\delta_{i-1}|} \ge \frac{1}{2}$, altså at mere end halvdelen af permutationerne er tilbage.
  \item Det er ikke muligt at garantere mere end halvdelen, da det bedst mulige svar, ville fjerne halvdelen.
  \item Dermed skal du bruge \textbf{mindst} $\log_{2}n!$ sammenligninger før $|\delta_{k}|=1$.
  \item Vi viste tidligere at $\log_{2}n! = n \log n - cn$.
  \item De tidligere problemer vi har kigget på har det været nemt for modstanderen at finde et godt modsvar, men her kræver det utroligt meget for modstanderen.
\end{itemize}
\end{frame}
% 21:31 eller iindtil han er blevet færdig med at forklare det'' `nemme argument``''`.

%%% Local Variables:
%%% mode: latex
%%% TeX-engine: xetex
%%% TeX-command-extra-options: "-shell-escape"
%%% TeX-master: "main"
%%% End:

\chapter{Faste parameteralgoritmer, parametriseret kompleksitet og eksakte eksponentielle algoritmer}

\section{Faste Parameteralgoritmer og Parametriseret Kompleksitet}%
\label{sec:label}

\begin{note}[Kilder]
	\href{https://imada.sdu.dk/u/jbj/DM553/FPTDM553.pdf}{Parameterized Algorithms, Cygan: pp. 3-7, 12-14, 17-22, 51-55}\\
	Video 25
\end{note}

Antag at du er ejer af en bar. Du skal sørge for at der ikke kommer nogen slåskampe. Du kender allerede de personer der gerne vil slås, og hvem de gerne vil slås med. Dit mål er derfor at, givet $n$ personer, vil du gerne ekskludere $k$ personer, således at så få personer som muligt kommer op og slås.

Vi kan modellere dette som et prbolem af \textsc{Vertex-Cover} problemet. Her er en knude en person, og en kant mellem to knuder betyder at de to knuder vil slås.

\begin{wrapfigure}{r}{0.5\textwidth}
	\centering
	\begin{tikzpicture}[scale=0.5]
		\begin{scope}[every node/.style={circle,thick,draw}]
			\node (A) at (0,0) {};
			\node (B) at (0,3) {};
			\node (C) at (2.5,4) {};
			\node (D) at (2.5,1) {};
			\node (E) at (2.5,-1) {};
			\node (F) at (5,2) {};
		\end{scope}
		\begin{scope}[>={Stealth[black]}]
			\path [-] (A) edge node {} (B);
			\path [-] (A) edge node {} (D);
			\path [-] (D) edge node {} (C);
			\path [-] (D) edge node {} (E);
			\path [-] (D) edge node {} (F);
			\path [-] (E) edge node {} (F);
		\end{scope}
	\end{tikzpicture}
	\caption{\label{fig:barfightvertexcover} En graf $G = (V,E)$.}
\end{wrapfigure}

I Figur~\ref{fig:barfightvertexcover} ses et eksempel på en normal graf. For at konverte forstå grafen i forhold til barkamps problemet, er hver knude en person, og hver kant mellem to knuder, en slåskamp der venter på at ske. De knuder vi vælger til at være en del af mængden returneret af \textsc{Vertex-Cover} problemet, er de personer vi ikke lader komme ind i baren. Dermed er en instans $(G, k)$ af barkampsproblemet, lig med en instans $(G,k)$ af vertex-cover problemet. For at se hvorfor det er en instans af \textsc{Vertex-Cover} problemet, husk at formålet er at finde en mængde af knuder, som ``cover'' alle kanter. Dermed, hvis vi finder et ``cover'' på størrelse $k$ der cover så mange kanter som muligt, så har vi fundet de $k$ personer, der vil slås med flest andre personer i analogien. Da alle kanter er covered, betyder det dermed også at \textbf{ingen} slåskampe kommer til at foregå.

Vi skal løse barkampsproblemet, men da \textsc{Vertex-Cover} er $\mathcal{NP}$-komplet, kan vi ikke bare gøre det som normalt. Vi antager at der er 1000 personer, som vil komme ind på baren, og at de har booket en plads. Derudover vil vi højest have at 10 personer bliver ekskluderet fra deres plads. Der er nu to umiddelbare metoder: \textit{Den Naive Metode} og \textit{Binomialmetoden}. Ved den naive metode prøver vi alle $2^{1000} \approx 1.07 \cdot 10^{301}$ delmængder og tjekker om en af dem er et cover af størrelse $\le k$. En algoritme der kører den naive metode, vil nok ikke blive færdig før universet har kollapset ind på sig selv. Ved binomialmetoden prøver vi alle $\binom{n}{k}$ delmængder af størrelse $k$, hvilket er en del bedre, cirka $2.63 \cdot 10^{23}$ muligheder. Dog er dette stadig alt for mange, og vil tage år at udregne.

Vores løsning til problemet, er at tage instansen, og give det nogle regler, som vi kalder ``sikkerhedsregler''. Disse regler sikrer et mindre input, som vi så kan arbejde med, enten ved brute force, eller med en klogere løsning. Denne fremgangsmåde kaldes \textit{problemreducering} eller \textit{kernelisering}. Vi vil nu kigge på hvilke regler vi kan opsætte for \textsc{Vertex-Cover} problemet. Givet en graf $G = (V,E)$ og et ikke-negativt heltal $k$:
\begin{itemize}
	\item \textbf{Regel 1}: Hvis $\forall v \in V \mid d(v) = 0$, så fjerner vi $v$ fra grafen.
	\item \textbf{Regel 2}: Hvis $\forall v \in V \mid d(v) \ge k + 1$ så fjerner vi $v$ og reducerer $k$ med én. vi putter $v$ ind, da, hvis vi ikke putter $v$ ind, skal vi putte \textbf{alle} $k+1$ naboer ind, hvilket vi ikke kan givet tallet $k$.
	\item \textbf{Regel 3}. Hvis $\forall v \in V \mid d(v) = 1 \text{ og } w \text{ er den eneste nabo til }v$   så fjerner vi $v$ og $w$ i grafen, og reducerer $k$ med 1. Vi kan se det som at putte $w$ i coveret, da der kan være mange kanter til den, men vi ved der kun er én til $v$.
\end{itemize}

Vi putter disse regler på inputtet $G$ indtil ingen af dem er mulige længere, og lader $G' = (V', E')$ være den resulterende graf og lade $k'$ være det resulterende parameter. Hvis, efter vi har fulgt disse regler, $k' = 0$ og der stadig er kanter i grafen, så kan vi \textit{afvise} $(G, k)$, da vi ved ud fra vores regler at dem vi tilføjer til coveret, ville være i det optimale cover. Hvis vi har kørt alle reglerne igennem, og $k > 0$, så ved vi at $\forall v \in V \mid d_{G'}(v) \le k$ fra regel 2. Da hver kant højest kan være incident til $k'$ knuder, ved vi at $|E'| \le k'^{2}$. Grunden til vi bruger $k'^{2}$ fremfor $k^{2}$, er fordi regel 2 bliver brugt, selv når $k$ bliver reduceret. Ydermere, efter at have brugt alle reglerne, gælder det at $|V(G')| \le k^{2}$. Vi får dette resultat fra følgende udregning:
\begin{equation*}
	|V'| = \sum_{v \in V'} 1 = \frac{1}{2} \sum_{v \in V'} 2 \le \frac{1}{2} \sum_{v \in V'} d(v) = |E'| \le k'^{2} \le k^{2}
\end{equation*}

Delen af udregningen der siger $\frac{1}{2} \sum_{v \in V'}2 \le \frac{1}{2} \sum_{v \in V'} d(v)  $ kommer fra at vi har fjernet alle $v \mid d(v) = 1$ og $v \mid d(v) = 0$ i den originale graf, og dermed må det gælde at $\forall v \in V \mid d(v) \ge 2$.

Når vi har sat disse regler ind og udført dem, kan vi køre en brute-force algoritme på inputtet $(G', k')$ og prøve alle $k'$ delmængder af $V'$ som er højest $\binom{k'^{2}}{k'} \le \binom{k^{2}}{k}$. Hvis $k = 10$ betyder det at vi \textbf{højest} skal prøve $\binom{100}{10} \approx 1.73 \cdot 10^{13}$ mulige delmængder. Bemærk nu, at \textit{reduceringen} eller \textit{kerneliseringen} af inputtet blev gjort i polynomiel tid, og har reduceret tiden markant.

Vi vil også gerne vide hvilke knuder der ender med at være i coveret, og ikke bare om det er muligt. Lad $C_{1}$ være de knuder vi har tilføjet til vertex coveret i regel 2 og 3, og lad $C'$ være coveret der er fundet når $(G', k')$ er en ``ja'' instans. instans. Dermed er $C_{1} \cup C'$ et cover af $G$ hvis størrelse er $\le k$.

\subsection{Køretid på parameteriseret \textsc{Vertex-Cover}}%
\label{subsec:label}

Tiden der bliver brugt i kerneliseringen af inputtet er $O((n+m)k)$. Vi reducerer $k$ højest $k$ gange (mere end $k$ vil afvise instansen), og de $k$ gange der reducerer, kan vi lave $n+m$ arbejde, hvor $n$ er antallet af knuder og $m$ antallet af kanter. Hvis vi ikke afviser instansen, så kommer vi til en instans $(G', k')$ således at $(G,k)$ er en ``ja'' instans $\iff$ $(G', k')$ er en ``ja'' instans.

Når vi har kørt reglerne igennem, og har en ny instans $(G', k')$ skal vi højest brute force igennem $\binom{k^{2}}{k}$ mulige løsninger. Hver løsning tager $O(|V'|+|E')$ tid at kigge igennem, hvilket vi så tidligere var lig med $O(k^{2})$. Dermed kan vi løse $(G, k)$ i følgende tid, hvor $g(k)$ er en funktion af $k$:
\begin{equation*}
	O((n+m)k) + O(\binom{k^{2}}{k}k^{2}) = O(g(k)(n+m)) = O(g(k)\cdot n^{2})
\end{equation*}

\subsection{Definitioner}%
\label{subsec:label}

\begin{definition}[Parameteriseret Problem]
	Et \textit{parameteriseret problem} $Q$ er \textit{Fastsat Parameter Traktabel} (FPT) eller \textit{Parametertraktabel i Faste Parametre} hvis der eksisterer en algoritme $A_{Q}$ der løser $Q$ i tid $O(f(k) \cdot n^{c})$ for en beregnlig funktion $f$ og en konstant $c \in \mathbb{R}_{+}$.
\end{definition}

Vi har tidligere vist at \textsc{Vertex-Cover} er FPT, da vi fandt en løsning der kørte i tid $O(g(k) \cdot n^{2})$, hvor $g(k) = f(k)$ og $2 = c$.

\begin{definition}[Kernelisation, kernel]
	En kernelisationsalgoritme (en kernel) for et parameteriseret problem $Q$ er en algoritme $A_{Q}$, som, givet en instans $(I,k)$ kører i polynomiel tid i $|(I,k)|$ og outputter en ækvivalent instans $(I', k')$, hvor $|I'| + k' \le g(k)$ for hver instans $(I, k)$ af $Q$ og $g$ er en fastsat beregnelig funktion.
\end{definition}

Ved \textsc{Vertex-Cover} problemet, tog vi input $(G, k)$ og producerede en kernel $(G', k')$ som opfylder kravet $|G'| + k' \le 2k^{2} + k$.

Bemærk at hvis et parameteriseret problem $Q$ med parameter $k$ har en kernel af størrelse $O(g(k))$ for en $g$, så kan vi løse $Q$ først ved at finde en kernel og så ved at tjekke alle mulige løsninger for kernelen (brute force).

%% 32:40

%%% Local Variables:
%%% mode: latex
%%% TeX-engine: xetex
%%% TeX-command-extra-options: "-shell-escape"
%%% TeX-master: "main"
%%% End:



\end{document}

%%% Local Variables:
%%% mode: latex
%%% TeX-engine: xetex
%%% TeX-command-extra-options: "-shell-escape"
%%% End:
