\chapter{Reducérbarhed}

I dette kapitel kommer vi til at kigge på problemer der ikke kan løses. Et eksempel på et sådan problem er $A_{TM}$ som vi så i Sektion~\ref{sec:undecidability}. Vi introducerer også i dette kapitel den primære metode hvorpå vi beviser uafgørlighed, nemlig reducérbarhed. En reduktion er en måde hvorpå man kan konvertere ét problem til et andet problem på en måde hvorpå det andet problem kan bruges til at løse det første problem.

Givet problemet at du er i en ny by og du vil gerne finde vej, men det kan du ikke. I stedet får du målet at du gerne vil have et kort, så du kan finde vej med det kort. På den måde reduceres problemet fra at du skal finde vej til at du skal have et kort.

Reducérbarhed involverer to problemer, $A$ og $B$. Hvis $A$ reduceres til $B$, kan vi bruge en løsning til $B$ til at løse $A$. I eksemplet fra før, vil problem $A$ være at finde vej, og $B$ være at finde et kort.

Vi kan bruge dette til at bevise uafgørlighed. Hvis $A$ reduceres til $B$, kan det ikke være sværere at løse $A$ end det kan at løse $B$. Dermed, hvis $A$ er uafgørligt er $B$ også uafgørligt. Dette går også den anden vej, altså hvis $B$ er uafgørligt, så er $A$ også (hvis du ikke kan finde vej, hjælper et kort ikke, hvis du ikke kan finde et kort, kan du ikke finde vej.)

\section{Uafgørlige problemer fra Sprogteori}%
\label{sec:uafgørligsprogteri}

Standseproblemet (halting problem) $$HALT = \{\text{\texttt{<m><w>}} \mid m \text{ er en Turingmaskine og m stopper på }w\}$$
Standsproblemet er et meget vigtigt problem i datalogi, da det viser at vi ikke kan bestemme om en maskine stopper på sit input, eller om det kører i en uendelig løkke.

Vi bruger uafgørligheden fra $A_{TM}$ til at vises at den kan reduceres til $HALT_{TM}$, som vi så fra introduktionen til kapitlet, altså også betyder at $HALT_{TM}$ er uafgørligt. Måden vi gør dette på, er ved et modstridsbevis. Vi antager altså at $HALT_{TM}$ er afgørligt, men viser at det kan reduceres fra $A_{TM}$.

\begin{theorem}
	$HALT_{TM}$ er uafgørligt.
\end{theorem}

\begin{proof}
	Antag at Turingmaskinen $R$ afgører HALT. Vi vil nu vise at dette indebærer at $A_{TM}$ er afgørligt (som vi så ikke passede i sidste kapitel.)

	Vi konstruerer en ny Turingmaskine, $A$. Vi definerer $A$ til at være en maskine der tager en kodning af en maskine som input, og giver en kodning til output. Vi definerer $A$ til at gøre følgende:
	\begin{enumerate}
		\item Hvis \texttt{<m>} ikke er en Turingmaskine, så \texttt{<m'> = <m>}
		\item Ellers er \texttt{<m>} en Turingmaskine hvor det gælder at $L(m') = L(
			      m)$ hvor $m'$ løkker uendeligt på alle strenge der ikke er en del af $L(m')$.
	\end{enumerate}

	Det vil altså sige at $A$ kun stopper på strenge som er \textbf{accepteret}, og løkker uendeligt på strenge der ikke accepteres.

	Lad os nu konsturere en ny Turingmaskine, $R$ som tager en maskine \texttt{<m>} og en streng \texttt{<w>} som input. $R$ fungerer meget simpelt, hvis \texttt{<m>} accepterer \texttt{<w>}, så accepterer $R$ kombinationen, og omvendt, hvis \texttt{<m>} afviser \texttt{<w>}, så afviser $R$ kombinationen. Det vil altså sige, at hvis vi kan konstrurere en sådan maskine, så vil $HALT$ være afgørligt.

	Vi konsturerer nu en ny maskine, $M_{R}$ som tager som input \texttt{<m>} og \texttt{<w>}. \texttt{<m>} bliver fodret til $A$ som giver sit output som input til $R$, og \texttt{<w>} går direkte til input i $R$. $R$ vil så enten give en accept eller afvisning baseret på disse inputs. Men som vi så før, hvis en streng ikke er en del af sproget i $A$ så kører den i en uendelig løkke, hvilket betyder at $R$ ud fra en uendelig løkke skal kunne afvise, hvilket naturligvis ikke er muligt.

\end{proof}

\begin{equation*}
	E_{TM} = \{\text{\texttt{<m>}} \mid m \text{ er en Turingmaskine og } L(M) = \emptyset\}
\end{equation*}

\begin{theorem}
	\label{teo:etmundecidable}
	$E_{TM}$ er uafgørligt.
\end{theorem}

\begin{proof}
Måden vi beviser dette på, er ved at konstruere en ny Turingmaskine, som vi giver som input til en Turingmaskine som afgører $E_{TM}$, og hvis den accepterer, så \textit{afviser} vi, og ellers \textit{accepterer} vi.

	Antag at $S$ er en Turingmaskine som afgører $E_{TM}$. Vi bruger $S$ til at konsturere en afgører for $A_{TM}$ og kommer til et modstrid.

	Til at gøre dette bruger vi en maskine, $B$ som fungerer således: $B$ tager to input, \texttt{<m>} og \texttt{<w>}. Ouputtet fra $B$  er \texttt{<$m_{w}$>}. Den fungerer som følger:
	\begin{enumerate}
		\item Først tjekkes om \texttt{<m>} er en kodning til en Turingmaskine. Hvis ikke, så outputter den $m_{w}$, som er en Turingmaskine hvor $L(m_{w}) = \emptyset$.
		\item Hvis \texttt{<m>} er en Turingmaskine, så output Turingmaskine $m_{w}$ hvor
		      \begin{equation*}
			      L(m_{w}) =\begin{cases}
				      \{w\}     & \text{ hvis } w \in L(M)    \\
				      \emptyset & \text{ hvis } w \notin L(M) \\
			      \end{cases}
		      \end{equation*}
	\end{enumerate}
	Altså vil det sige at hvis det ikke er en Turingmaskine, outputter $m_{w}$ en turingmaskine hvis sprog er det tomme sprog, altså $\emptyset$. Derimod, hvis det er en Turingmaskine, så beholder den $w$ til at være en accepteret streng hvis det er en accepteret streng i \texttt{<m>}, men hvis \textbf{ikke} det er en accepteret streng, så bliver $L(M_{w}) = \emptyset$, altså vil \textbf{ingen} strenge accepteres.

	Maskinen $m_{w}$ fungerer således:
	\begin{enumerate}
		\item Tjek om input $x$ er lig $w$.
		\item Hvis $x \neq w$ så afvis
		\item Ellers, så simulér $m$ på $w$ og accepter hvis og kun hvis $m$ accepterer $w$.
	\end{enumerate}

	Husk nu $S$, som er en Turingmaskine der afgører $E_{TM}$. Hvis $S$ får $m_{w}$ som input, og kan afgøre om $L(m_{w}) = \emptyset$ eller ej. Det betyder altså, at hvis $S$ accepterer, og dermed $L(m_{w}) = \emptyset$, så accepterer $M$ \textbf{ikke} $w$ (da sproget så ville være $w$!) Dermed betyder det også at hvis $M$ accepterer $w$, så er $L(m_{w}) = w$, og $S$ afgører $A_{TM}$.

	Lad os nu ``indkapsulere'' disse maskine i en ny maskine, $S^{*}$. $S^{*}$ tager to inputs, \texttt{<m>} og \texttt{<w>}. Disse går til maskinen $B$, som giver output \texttt{<$m_{w}$>} som går til $S$. $S$ afgører så om \texttt{<$m_{w}$>} er en Turingmaskine og $L(M) = \emptyset$. $S^{*}$ tager så dette input og gør det omvendt, så en accept vil være en afvisning og en afvisning vil være en accept.

	Vi kan nu se at $S^{*}$ afgører $A_{TM}$ som er umuligt pr. Teorem \ref{teo:atmundecidable}.
\end{proof}

Trods at $E_{TM}$ ikke er afgørligt, så er $\overline{E_{TM}}$ genkendeligt. Bemærk at
\begin{align*}
	\overline{E_{TM}} = \{  \text{\texttt{<m>}} \mid & \text{  enten er \texttt{<w>}  ikke en Turingmaskine, eller \texttt{<w> = <m>}} \\
	                                                 & \text{for en Turingmaskine hvor } L(M) \ne \emptyset\}
\end{align*}

Følgende er en beskrivelse af en Turingmaskine der genkender $\overline{E_{TM}}$:
\begin{enumerate}
	\item Tjek om \texttt{<m>} koder en Turingmaskine, hvis ikke så \textit{accepter} \texttt{<w>}
	\item Lad $M$ være en Turingmaskine kodet af \texttt{<w>}
	\item Simulér $M$ på strengene over $\Sigma^{*}$ in lexikografisk order i parallel.
\end{enumerate}

Altså, for $i = 1, 2, \ldots$ simulér $M$ i $i$ skridt på strengene $w_{1}, w_{2} , \ldots, w_{i}$ ifølge den lexikografiske rækkefølge. Stop når en streng accepteres.

Bemærk at hvis $L(M) \neq \emptyset$ så stopper den beskrevne algoritme efter højest $p$ skridt hvor $p = \max (t,q)$ og $m$ accepterer $w$ på $t$ skridt og $w = w_{q}$.

\begin{corollary}
	$E_{TM}$ er \textbf{ikke} Turing-genkendeligt.
\end{corollary}

Vi vil nu kigge på to specielle Turingmaskine, $M_{\emptyset}$ og $M_{\Sigma^{*}}$.

$M_{\emptyset}$ går direkte til afvisstaten uanset hvad inputtet er (inklusiv tomt input!) Så $\delta(q_{0}, a) = q_{reject} \;\; \forall a \in \Gamma$, $L(M_{\emptyset}) = \emptyset$.
$M_{\Sigma^{*}}$ går direkte til dens accept state uanset hvad input er, altså $\delta(q_{i}, a) = q_{accept} \;\; i = 0, 1, \ldots, \; \; \forall a \in \Gamma$.

Vi vil nu gerne vise at følgende Turingmaskine er uafgørlig:

\begin{equation*}
	EQ_{TM} = \{\text{\texttt{<m$_1$><m$_2$>}} \mid m_{1} \text{ og } m_{2} \text{ er Turingmaskiner hvor } L(m_{1}) = L(m_{2})\}
\end{equation*}

\begin{theorem}
	$EQ_{TM}$ er uafgørligt.
\end{theorem}

\begin{proof}
	Vi vil vise dette ved at bruge $E_{TM}$ (fra Teorem \ref{teo:etmundecidable}) ved at reducere til den $EQ_{TM}$.

	Antag at $m_{EQ}$ afgører $EQ_{TM}$, så kan vi lave Turingmaskinen $m^{*}$ der givet et input \texttt{<m>} og \texttt{<m$_{\emptyset}$>} enten accepterer eller afviser baseret på om de to maskiner er lig hinande (altså om $L(m) = L(m_{\emptyset}) = \emptyset$.) Vi ved at dette ikke er afgørligt (fra Teorem \ref{teo:etmundecidable}.)
\end{proof}

Vi har nu set et eksempel på et bevis hvor vi siger at et sprog er uafgørligt, fordi hvis det kan afgøres, vil et andet bevist uafgørligt sprog ikke være afgørligt. Vi vil gerne se om det er muligt at formalisere denne process, så vi ikke ``bare'' laver en hypotetisk Turingmaskine der beviser ved at reducerer. Til dette bruger vi Sipser kapitel 5.3.
\begin{definition}
	$f : \Sigma^{*} \rightarrow \Sigma^{*}$ er \textbf{beregneligt}\footnote{Computable på engelsk} hvis $\exists TM \;M_{f}$ som starter med $w$ og altid stopper med $f(w)$ på sit bånd: $q_{0}w \stackrel{*}{\Rightarrow} q_{acc}f(w)$.
\end{definition}

Et eksempel på dette er $f : \mathbb{N} \rightarrow \mathbb{N}$, $f(x) = x^{2}$.

\begin{definition}
	Lad $A, B$ være sprog. Vi siger at $A$ er reducérbart\footnote{Mapping reducible} til $B$ hvis $\exists$ en beregnbar funktion $f : \Sigma^{*} \rightarrow \Sigma^{*}$ således at $w \in A \iff f(w) \in B$. Vi skriver $A \le _{m}B$ og kalder $f$ en reduktion af $A$ til $B$.
\end{definition}

\begin{theorem}
	Hvis $A \leq _{m}B$ og $B$ er afgørligt, så er $A$ afgørlig.
\end{theorem}

\begin{proof}
	Lad $M$ være en afgører for $B$ og $f$ være reduktionen fra $A$ til $B$. Vi beskriver en afgører $N$ for $A$ som følger:

	$N = $ ``På input $w$
	\begin{enumerate}
		\item Beregn $f(w)$
		\item Kør $M$ på input $f(w)$ og output hvad $M$ outputter''
	\end{enumerate}

	Hermed, hvis $w \in A$, så $f(w) \in B$ fordi $f$ er en reduktion fra $A$ til $B$. Dermed accepterer $M$ $f(w)$ når $w \in A$.
\end{proof}

\begin{corollary}
	Hvis $A \leq _{m}B$ er uafgørligt, så er $B$ også uafgørligt.
\end{corollary}

\begin{theorem}
	$A \leq_{m} B$ og $B$ er genkendeligt, så er $A$ genkendeligt.
\end{theorem}

\begin{proof}
	Beviset her er det samme som i beviset med afgørlighed, undtagen her bruges genkendelighed.
\end{proof}

\begin{corollary}
	$A \leq_m B$ og $A$ er \textbf{ikke} genkendeligt, så er $B$ ikke genkendeligt.
\end{corollary}


Generelt set kan vi følge en opskrift på et uafgørlighedsbevis ved hjælp af reducering:
Lad $K$ være et sprog du ønsker at vise til at være uafgørligt.
\begin{enumerate}

	\item Vælg et passende sprog, $R$ som vi allerede ved er uafgørligt.
	\item Beskriv en reduktion fra $R$ til $K$ ved hjælp af en beregnelig funktion. Lad $A$ være en Turingmaskine der beregner denne reduktion. Altså givet $x$ udregner $A$ værdien $f(x)$ (hvis funktionen er $f$.)
	\item Lad $M_{K}$ være en hypotetisk Turingmkasine der afgører $K$ (vores mål er at vise at denne ikke eksisterer.)
	\item Brug $A$ og $M_{K}$ til at lave en ny Turingmaskine $M^{*}$ som afgører $R$.
	\item Konkludér fra ovenstående et modstrid, og at $K$ dermed er uafgørlig.
\end{enumerate}



\begin{equation*}
	H_{\varepsilon} = \{\text{\texttt{<m>}} \mid m \text{ er en Turingmaskine og } \varepsilon \in L(M)\}
\end{equation*}

\begin{theorem}
	$H_{\varepsilon}$ er uafgørligt.
\end{theorem}

\begin{proof}
	Vi bruger opskriften beskrevet tidligere.
	\begin{enumerate}
		\item Lad $R$ være sproget $HALT_{TM}$ som vi ved er uafgørligt.
		\item Lad $A$ være en Turingmaskine der givet kodningen \texttt{<m>} for en Turingmaskine og \texttt{<w>} for en streng outputter \texttt{<m$_{w}$>}. Denne Turingmaskine kører på et tomt input, og accepterer hvis $M$ accepterer $w$. Den gør dette ved at kopiere $w$ til båndet, og så køre $M$, ved hjælp af den universelle Turingmaskine \texttt{<U>}. Hvis \texttt{<m>} stopper og accepterer \texttt{<w>} så vil \texttt{<m$_{w}$>} også stoppe og acceptere. Hvis ikke, kører den forevigt. Det vil altså sige at \texttt{<m$_{w}$>} stopper på den tomme streng hvis og kun hvis $M$ stopper på $w$.
		\item Antag at $M_{H_{\varepsilon}}$ afgører $H_{\varepsilon}$
		\item Brug $A$ og \texttt{<m$_{H_{\varepsilon}}$>} til at lave en Turingmaskine $M^{HALT}$ der afgør $HALT_{TM}$.
		\item Da $HALT_{TM}$ er uafgørligt er $H_{\varepsilon}$ uafgørligt.
	\end{enumerate}

\end{proof}

\begin{equation*}
	Regular_{TM} = \{\text{\texttt{<m>}} \mid m \text{ er en Turingmaskine og }L(m) \text{ er regulær}\}
\end{equation*}

\begin{theorem}
	$Regular_{TM}$ er uafgørlig.
\end{theorem}

\begin{proof}
	Vi reducerer fra $A_{TM}$. Vi antager at $Regular_{TM}$ er afgørligt af en Turingmaskine, $R$ og bruger dette til at konsturere en Turingmaskine $S$, som afgører $A_{TM}$. Så altså:
	\begin{enumerate}
		\item $R$ afgører $Regular_{TM}$
		\item $A$ afgører $A_{TM}$ og bruger $R$ til at gøre dette.
	\end{enumerate}

	Vores idé er at $S$ tager inputtet \texttt{<m><w>} og ændrer på \texttt{<m>} så den resulterende Turingmaskine genkender et regulært sprog hvis og kun hvis \texttt{<m>} accepterer \texttt{<w>}. Denne ændring af \texttt{<m>} kalder vi \texttt{<m$_{2}$>}.

	Vi konstruerer \texttt{<m$_{2}$>} til at genkende sproget $\{0^{n}1^{n} \mid n \ge 0\}$ (altså et ikke-regulært sprog) hvis \texttt{<m>} ikke accepterer \texttt{<w>}, og genkender det regulære sprog $\Sigma^{*}$ hvis \texttt{<m>} accepterer \texttt{<w>}. Altså:

	\begin{equation*}
		L(\text{\texttt{<m$_{2}$>}}) = \begin{cases}
			\{0^{n}1^{n} \mid n \ge 0\} & \text{ hvis } w \notin L(\text{\texttt{<m>}}) \\
			\Sigma^{*}                  & \text{ hvis } w \in L(\text{\texttt{<m>}})
		\end{cases}
	\end{equation*}

	Hvor vi må huske at $L(m)$ er et regulært sprog, så hvis $w$ er regulært, så er $L(m_{2}) = \Sigma^{*}$.
	Bemærk at $m_{2}$ ikke er konstrueret til faktisk at blive kørt på et input, men for at kunne ``fodre'' dens beskrivelse i en afgører for $Regular_{TM}$, som vi antaget eksisterer.

	Vi lader $R$ være en Turingmaskine der afgører $Regular_{TM}$ og vi konstruerer Turingmaskinen $S$ til at afgøre $A_{TM}$. Så fungerer $S$ på følgende måde:\\\\
	\noindent
	$S =$ ``På input \texttt{<m><w>}, hvor $m$ er en Turingmaskine og $w$ en streng:
	\begin{enumerate}
		\item Konstruér den følgende Turingmaskine $m_{2}$.
		      $m_{2} = $ ``på input $x$
		      \begin{enumerate}
			      \item Hvis $x$ er af formen $0^{n}1^{n}$, \textit{accepter}.
			      \item Hvis $x$ ikke har denne form, så kør $M$ på input $w$ og \textit{accepter} hvis $M$ accepterer $w$.
		      \end{enumerate}
		\item Kør $R$ på input \texttt{<m$_{2}$>}
		\item Hvis $R$ accepterer, så \textit{accepter}, hvis $R$ afviser, så \textit{afvis.}''
	\end{enumerate}


	Hvis vi følger den ``opskrift'' vi har fundet tidligere kan vi bevise som følger:
	\begin{enumerate}
		\item Vælg et passende sprog, $R$, som vi allerede ved er uafgørligt.
		      \begin{itemize}
			      \item Her vælger vi $A_{TM}$. $A_{TM}$ spørger, givet en Turingmaskine $m$ og en streng $w$, accepterer $M$ så $w$?
		      \end{itemize}
		\item Beskriv en reduktion fra $R$ til $K$ ved hjælp af en beregnelig funktion. Lad $A$ være en Turingmaskine der begeregner denne reduktion.
		      \begin{itemize}
			      \item Lad $K = Regular_{TM}$
			      \item Vi konstruerer en beregnelig funktion v.h.a. Tuiringmaskinen $A$, som transformere enhver instans \texttt{<m><w>} af $A_{TM}$ til en Turingmaskine \texttt{<m$_{2}$>} hvor \texttt{<m$_{2}$>}s sprog er regulært hvis og kun hvis $M$ accepterer $w$.
			      \item $A$ fungerer som følger:
			            \begin{itemize}
				            \item Givet \texttt{<m><w>} kører $m_{2}$, og tjekker på input $x$ om det har formen $0^{n}1^{n}$, hvis ja, så accepterer $m_{2}$ $x$.
				            \item Uanset $x$'s form, så simuleres $M$ på $w$. $M_{2}$ er designet til at acceptere alle strenge, hvilket gør dens sprog til $\Sigma^{*}$ (et regulært sprog.)
			            \end{itemize}
		      \end{itemize}
		\item Lad $M_{K}$ være en hypotetisk Turingmaskine der afgører $K$ (vore smål er at vise at denne ikke eksisterer.)
		      \begin{itemize}
			      \item $M_{K}$ er en hypotetisk maskine der afgører $REGULAR_{TM}$, som bestemmer om en givet Turingmaskine genkender et regulært sprog.
		      \end{itemize}

		\item Brug $A$ og $M_{K}$ til at lave en ny Turingmaskine $M^{*}$ der afgører $R$.
		      \begin{itemize}
			      \item $M^{*}$ bruger $A$ til at transformere en instans \texttt{<m><w>} af $A_{TM}$ til $M_{2}$.
			      \item $M^{*}$ kører så $M_{K}$ på $M_{2}$. Hvis $M_{K}$ bestemmer $M_{2}$'s sprog er regulært, så konkluderer $M^{*}$ at $M$ accepterer $w$, og ellers konkluderer den at $M$ \textbf{ikke} accepterer $w$.
			      \item Essentielt afgørerer $M^{*}$ sproget $A_{TM}$ ved brug af $M_{K}$ som er en modstrid, fordi $A_{TM}$ er uafgørligt.
		      \end{itemize}
		\item Konkludér fra ovenstående et modstrid, og at $K$ dermed er uafgørlig.
		      \begin{itemize}
			      \item Eksistensen af $M^{*}$ leder til en modstrid da $A_{TM}$ er afgørlig.
			      \item Siden vores konstruktion af $M^{*}$ var baseret på vores antagelse at $REGULAR_{TM}$ (vores $K$) er afgørlig, så indebærer modstriden at vores antagelse er falsk.
			      \item Derfor er $REGULAR_{TM}$ uafgørlig.
		      \end{itemize}
	\end{enumerate}


\end{proof}

\subsection{Konsekvenser i praktik}%
\label{subsec:uafgørlighedpraktik}

De her noter kommer fra ChatGPT, hvor jeg har spurgt hvordan disse problemer har en effekt på softwareudvikling, og generelt praktisk arbejde. Jeg har selvfølgelig omskrevet noterne til mit eget sprog. Jeg har kun to eksempler her, standseproblemet (Halting problemet) og tomhedsproblemet:


\subsubsection{Standseproblemet}
Standseproblemet siger at vi aldrig kan vide om et program vil stoppe eller ej. To implikationer på dette i praktik er:
\begin{itemize}
  \item I \textbf{debugging} og \textbf{verifikation}. Hvis vi vidste om et program stoppede, ville vi også kunne debugge det en del nemmere, da vi kunne finde ud af præcis hvad fejlen var når programmet stoppede, og finde præcis denne fejl. På samme måde ville problemet om verifikation (er en algoritme korrekt?) kunne løses, da det ville tillade at køre alle muligheder igennem, hvor de ville stoppe på et tidspunkt (hvori problemet nu er at verifikationsmuligheder ikke nødvendigvis stopper. Der findes dog værktøj såsom TLA, der forsøger at fikse dette (som jeg skrev bachelor i, hehe.))
  \item I \textbf{Cybersikkerhed}: Hvis vi kunne finde ud af om et program ville stoppe, ville vi også kunne finde ud af om kode indeholder \textit{malware}. Da vi ikke kan garantere at et program stopper, kan vi heller ikke garantere at vi har fundet eksisterende malware i kode.
\end{itemize}

\subsubsection{Tomhedsproblemet ($L(M) = \emptyset$)}
Her kan vi også primært se det i to felter:
\begin{itemize}
  \item I \textit{compiler design}: Hvis vi ved at et substykke af kode er dødt (altså $L(M) = \emptyset$ hvor $M$ er den døde stykke kode), så vil vi kunne optimere et program langt mere.
  \item I \textit{verifikation} (igen): Hvis dette problem var afgørligt, ville vi vide om der var nogen muligheder (e.g., kan givet et input $x$, ved vi at algoritmen ikke vil acceptere noget $x$?)
\end{itemize}



\subsection{Rice's Teorem}%
\label{subsec:ricetheorem}

En egenskab $P$ vedrører sproget for en Turingmaskine $M$ hvis $P$ handler om sproget $M$. Følgende er eksempler på sådanne egenskaber:
\begin{enumerate}
	\item $L(m)$ er regulært
	\item $L(m) = \emptyset$
	\item $L(m)$ indeholder strengene $x,y$ således at $|x| = |y|$
	\item $L(m)$ indeholder en streng $x$ hvor $|x| = 22$
\end{enumerate}
Følgende egenskab er \textbf{ikke} om sproget af en Turingmaskine: $\exists w \in \Sigma^{*}$ således at når $m$ er startet på $w$ vil den besøge alle states undtagen $q_{acc}$ og $q_{reject}$.

\begin{definition}
	En egenskab $P$ om sproget af en Turingmaskine er \textbf{ikke-trivielt}, hvis
	\begin{itemize}


		\item $\exists$ En Turingmaskine $m_{1}$ således at $L(m_{1})$ har egenskaben $P$
		\item $\exists$ En Turingmaskine $m_{2}$ således at $L(m_{2})$ \textbf{ikke} har egenskaben $P$
	\end{itemize}

\end{definition}

Altså er en egenskab ikke-triviel hvis der eksisterer en Turingmaskine hvis sprog har egenskaben $P$, og der eksisterer en seperat Turingmaskine hvis sprog \textbf{ikke} har egenskaben $P$.

Hvis vi kigger på eksemplerne før kan alle deles op i at der eksisterer en Turingmaskine der har disse egenskaber i deres sprog, og på samme måde findes der en hvor det ikke gælder. For eksempel ved det første, er der en Turingmaskine hvor sproget er regulært, men på samme tid er der også en Turingmaskine hvor sproget \textbf{ikke} er regulært.

Det vil altså sige at \textbf{trivielle egenskaber} er egenskaber som gælder for alle eller for ingen Turingmaskiner. Eksempler på disse:
\begin{itemize}
  \item Egenskaben at en Turingmaskine eksisterer (gælder for alle Turingmaskiner.)
  \item Egenskaben at TM'en har et alfabet (gælder for alle Turingmaskiner.)
  \item Egenskaben at en TM er en TM (gælder for alle Turingmaskiner.)
\end{itemize}

\begin{theorem}[Rice's Theorem]
	Hver ikke-triviel egenskab $P$ om sproget af Turingmaskiner er uafgørlig.
\end{theorem}

Bemærk at sættet af ikke-trivielle egenskaber er uendeligt\footnote{Du kan altid tilføje mere til en ikke triviel egenskab, såsom at gøre længden længere, etc.}. Det vil altså sige at dette teorem siger at der eksisterer et uendeligt antal af sprog der er uafgørlige.

\begin{proof}
Modstridsbevis.

	Vi antager at det tomme sprog $L(M) = \emptyset$ \textbf{ikke} har egenskaben $P$, ellers overvej egenskaben $\overline{P}$. Hvis $P$ er ikke-triviel, så er $\overline{P}$ naturligvis også ikke-triviel, ud fra definitionen af ikke-trivielle egenskaber.


	Antag at Turingmaskinen $M_{P}$ kan afgøre for en givet kodning \texttt{<m>} af en Turingmaskine, om $L(M)$ har egenskaben $P$. Så, givet en Turingmaskine $m_{p}$ der tager som input kodningen af en Turingmaskine, \texttt{<m>}, vil den enten:
	\begin{itemize}
		\item Acceptere $\iff$  $m$ er en Turingmaskine og $L(m)$ har egenskaben $P$
		\item Afvise $\iff$ $m$ \textbf{ikke} er en Turingmaskine eller $L(m)$ ikke har egenskaben $P$.
	\end{itemize}

	Vi vil nu lave en Turingmaskinen der giver os et modstrid. Lad $M_{1}$ være en Turingmaskine hvor $L(M_{1})$ har egenskaben $P$. Husk at  vi antager af $L(M_{\emptyset})$ \textit{ikke} har egenskaben $P$. Vi vil vise hvordan man konstruerer en sådan Turingmaskine, hvilket givet \texttt{<m><w>} konstruerer en Turingmaskine $m(w)$ hvor
	\begin{equation*}
		L(M(w)) = \begin{cases}
			\emptyset & \text{ hvis } w \notin L(m) \text{ eller } m \text{ ikke er en Turingmaskine} \\
			L(M_{1})  & \text{ hvis } w \in L(m)
		\end{cases}
	\end{equation*}

	$L(m(w))$ har egenskaben $P$ $\iff$ \texttt{<m><w>} $\in A_{TM}$. Så \texttt{<m><w>} $\rightarrow$ $m(w)$ er en reduktion.
\end{proof}



%%% Local Variables:
%%% mode: latex
%%% TeX-engine: xetex
%%% TeX-command-extra-options: "-shell-escape"
%%% TeX-master: "main"
%%% End:
