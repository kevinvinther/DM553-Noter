\section{Adversary Argumenter}%
\label{sec:adversaryargumenter}

\begin{frame}
	\frametitle{Pensum}
	\begin{itemize}
		\item Baase 3: \textbf{Adversary nedre grænse argumenter}
		\item Baase 3.5: \textbf{Adversary Median Problem}
		\item Jørgens noter: \textbf{Noter på nedre grænser}
		\item Weekly Note 12
		\item Video 24-25
	\end{itemize}
\end{frame}

\begin{frame}[allowframebreaks]
  \frametitle{Nedre Grænser for Sammenligningsbaseret Sortering}
  \begin{itemize}
	\item Vi vil gerne bevise at en algoritme mindst vil kunne lave $\Theta(n \log n)$ sammenligninger.
	\item Til at bevise dette bruger vi \textit{decision trees}.
	\item Alle deterministiske sammenligningsbaseret sorteringsalgoritmer $A$  kan associeres med et decision tree.
	\item Et decision tree har en knude som stiller en udtalelse op, e.g. $x_{14} < x_{17}$, og har så to kanter, en for hvis \textit{ja} og en for hvis \textit{nej}.
	\item Disse kanter fører så videre til flere knuder der også har nej/ja kanter, osv. indtil du kommer til bladene.
	\item Det vil også sige at roden af træet er den første sammenligning der bliver lavet.
	\item Hvert blad repræsenterer altså en permutation for hvordan tallene kan se ud.
	\item Det vil sige at alle mulige permutationer er blade, og dermed er der mindst $n!$ blade, for et input af størrelse $n$. Det er også muligt at en permutation kan være i et blad mere end én gang.
	\item Et decision træ er et binært træ, hvilket betyder at for hvert ``niveau'' i træet, bliver antallet af knuder \textbf{højest} fordoblet.
	\item Dermed er der også højest $2^{h}$ blade, hvor $h$ er højden på træet.
	\item Dermed må $2^{h} \ge n!$.
	\item Dermed må $h \ge \log n! \ge n \log n - cn$
	\item Vi får dette fra at $2^{h} \ge n!$, hvor vi kan tage $\log_{2}$ på begge sider og få $\log_{2}(2^{h}) \ge \log_{2}(n!)$ som giver $h \ge \log_{2}(n!)$.
	\item Den anden del, at $\log n! \ge n \log n - cn$ håber jeg ikke vi skal kunne til eksamen, for ChatGPT's bevis bruger både $\pi$ og $e$, og er i øvrigt ultra langt.
	\item Hver vej fra roden til et blad svarer til sammenligninger lavet af $A$ til at sortere et input. Dermed er antallet af sammenligninger lig med længden af vejen.
	\item Dermed bruger $A$ \textbf{mindst} $n \log n - cn$ sammenligninger på et input.
	\item Vi kan også finde en nedre grænse for den gennemsnitlige mængde af sammenligninger via decision trees.
	\item Givet en mængde $P$ som indeholder alle veje fra roden til bladene, kan vi definere $epl$ (externaexternal path length) til at være:
		  \begin{equation*}
			epl = \sum_{p \in P} \text{længde}(p)
		  \end{equation*}
	\item Lemma 2.8 i Baase, siger at $epl$ er minimeret når $T$ er ``almost balanced'', som vil sige at forskellen i distancen mellem rodden og alle blade enten er den samme for alle blade, eller er højest 1.
  \end{itemize}
  % 17:34
\end{frame}

%%% Local Variables:
%%% mode: latex
%%% TeX-engine: xetex
%%% TeX-command-extra-options: "-shell-escape"
%%% TeX-master: "main"
%%% End:
