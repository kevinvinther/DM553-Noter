\section{Approximationsalgoritmer}%
\label{sec:Approximationsalgoritmer}

\begin{frame}
  \frametitle{Pensum}
  \begin{itemize}
    \item CLRS 35: \textbf{Approximationsalgoritmer}
    \item Weekly Note 10 (igen)
    \item Video 19-21
  \end{itemize}
\end{frame}

\begin{frame}
  \frametitle{NP-Komplette Problemer i Praksis}
  \begin{itemize}
    \item I praksis er der $NP$-problemer der er vigtige at have løsninger til.
    \item Hvis inputtet er småt nok, kan selv en algoritme med eksponentiel køretid være fin.
    \item Der er mulighed for at der er nogle specielle cases vi kan løse i polynomiel tid.
    \item Vi kan finde \textit{nær-optimale} løsninger i polynomiel tid. Nær-optimal er i praksis ofte godt \textit{nok}.
    \item En nær-optimal løsning kalder vi en \textit{approksimationsalgoritmer}.
  \end{itemize}
\end{frame}

\begin{frame}[allowframebreaks]
  \frametitle{Præstationsforhold}
  \begin{itemize}
    \item Hvis vi arbejder på et problem som har en positiv kost (e.g. skridt i en algoritme), kan vi definere en nær-optimal løsning til enten at være:
          \begin{itemize}
            \item En løsning der er maksimum mulig
            \item En løsning der er minimum mulig
          \end{itemize}
    \item Altså er problemet enten et maksimerings eller minimeringsproblem.
    \item Vi siger at en algoritme har et approksimeringsforhold af $\rho(n)$ hvis, for at input af størrelse $n$, så er kosten $C$ af en løsning indenfor en faktor af \(\rho(n)\) af kosten $C^{*}$ af en optimal løsning:
          \begin{equation}
\max \left( \frac{C}{C^{*}}, \frac{C^{*}}{C} \right) \le \rho(n)
          \end{equation}
    \item Hvis en algoritme får et approksimationsforhold af \(\rho(n)\), kalder vi det en $\rho(n)$ approksimationsalgoritme.
    \item Det betyder altså at, for eksempel ved Vertex-cover problemet, hvis vi har en 2-approksimationsalgoritme, så vil en løsning være \textbf{højest} en faktor af 2 væk fra den optimale løsning.
    \item For et maksimeringsproblem gælder det at $0 < C \le C^{{*}}$ og forholdet $\frac{C^{*}}{C}$ giver faktoren som kosten af den optimale løsning er større end.
    \item For et maksimeringsproblem gælder det at $0 < C^{*} \le C$, og forholdet $\frac{C}{C^{*}}$ giver faktoren som den approksimale løsning er større end kosten på den optimale løsning.
    \item Approksimeringsforholdet af en approksimationsalgoritme er aldrig mindre end 1, da en $1$-approksimeringsalgoritme er optimal.
    \item Et approksimationsskema (approximation scheme) til et optimeringsproblem er en approksimationsalgoritme som tager som input:
          \begin{itemize}
            \item En instans af problemet, OG
            \item En værdi $\epsilon > 0$ for en konstant \(\epsilon\).
          \end{itemize}

    \item Et sådant skema er en $(1+\epsilon)$-approksimationsalgoritme.
  \end{itemize}
\end{frame}

\begin{frame}[allowframebreaks]
  \frametitle{Vertex-cover problemet}
\begin{itemize}
  \item Husk: Et vertex cover af en urettet graf $G = (V,E)$ er en ægte delmængde $V' \subseteq V$ således at hvis en kant $(u,v)$ er en kant af $G$, så er enten $u \in V'$ eller $v \in V'$ eller begge.
  \item Størrelsen af et vertex cover er $|V'|$, i.e., antallet af knuder i det.
  \item Vertex-cover problemet siger at vi finde et vertex cover af minimum størrelse i en given urettet graf.
  \item Vi kalder et sådan vertex cover for et \textit{optimalt vertex cover}.
  \item Altså er dette optimiseringsversionen af decision problemet.
  \item Den følgende algoritme returnerer et vertex cover hvis størrelse er \textbf{højest} dobbelt det af det optimale.
\end{itemize}
\begin{center}
  \includegraphics[scale=0.45]{figur/approxvertexcover.png}
  \includegraphics[scale=0.35]{figur/figur3501.png}
\end{center}
\begin{itemize}
  \item Køretiden af denne algoritme er $O(V+E)$, hvor vi bruger en adjacency list til at repræsentere $E'$.
\end{itemize}

\begin{theorem}
\texttt{Approx-Vertex-Cover} er en polynomiel tids $2$-approksimationsalgoritme.
\end{theorem}
\begin{itemize}
  \item \textbf{Polynomiel tid:}
        \begin{itemize}
          \item Linje 1: Initialiserer $C$ til at være den tomme mængde: $O(1)$
          \item Linje 2: Lader $E'$ være en kopi af mængden $E \in G$: $O(1)$
          \item Linjer 3-6: Vælger en kant $(u,v) \in E'$, tilføjer $u$ og $v$ til $C$, og sletter alle kanter i $E'$ som coveres af enten $u$ eller $v$: $O(V+E)$
          \item Linje 7: Returnerer vertex coveret $C$: $O(1)$
        \end{itemize}

  \item Mængden $C$ af knuder der er returneret af AVC (Approx-Vertex-Cover) er et vertex cover, fordi algoritmen looper indtil alle kanter er i $C$.
  \item For at se at det er en $2$-approksimationsalgoritme:
  \item Lad $A$ være mængden af kanter som linje 4 i AVC valgte.
  \item For at cover kantene i $A$, skal ethvert vertex cover, specifikt det optimale, $C^{*}$ inkludere mindst én endpoint af hver kant i $A$.
  \item Der er ikke to kanter i $A$ som deler en endpoint, siden alle kanter der deler endpoint fjernes i linje 6.
  \item Dermed er der ikke to kanter i $A$ som er covered af den samme knude fra $C^{*}$, og dermed har vi en nedre grænse:
\end{itemize}
\begin{equation}
|C^{*}| \ge |A|
\end{equation}
\begin{itemize}
  \item Da linje $4$ vælger en kant hvor ingen af endpointsne allerede er i $C$, har vi et upper bound på størrelsen af vertex coveret vi får af AVC:
\end{itemize}
\begin{equation}
|C| = 2|A|
\end{equation}
\begin{itemize}
  \item Tallet $2|A|$ kommer fra de to knuder fra hver kant vi har udvalgt.
  \item Når vi kombinerer ligningerne får vi følgende:
\end{itemize}
\begin{align*}
  |C| &= 2|A|\\
  &\le 2|C^{*}|
\end{align*}
\begin{itemize}
  \item Som beviser sætningen.
  \item (Jeg har selv lidt brug for overbevisning her, jeg synes ikke Cormen's er specielt god.)
\end{itemize}
\end{frame}

\begin{frame}[allowframebreaks]
  \frametitle{Traveling-salesman Problemet}
  \begin{itemize}
    \item Ved TSP (Traveling-salesman problemet) får vi tildelt en urettet graf $G = (V,E)$.
    \item $\forall (u,v) \in E : c(u,v) \ge 0$: Hver kant har et heltals ``kostefunktion''.
    \item Målet med TSP er at finde en hamiloniansk kreds af $G$ med minimums kost.
    \item Lad $c(A)$ være den totale kost af kanter i delmængden $A \subseteq E$.
  \end{itemize}

  \begin{equation*}
c(A) = \sum_{(u,v) \in A} c(u,v)
  \end{equation*}
  \begin{itemize}
    \item I mange praktiske situationer er den billigste vej fra $u$ til $v$ at gå direkte.
    \item Vi formaliserer denne og siger at kostefunktionen $c$ satisfier trekantsuligheden (triangle inequality) hvis, $\forall u,v, w \in V$ gælder følgende:
  \end{itemize}
  \begin{equation*}
c(u,w) \le c(u,v) + c(v,w)
  \end{equation*}
  \begin{itemize}
    \item Vi vil starte med at vise en poly-tids 2-approksimationsalgoritme til TSP med trekantsuligheden.
    \item Efter dette vil vi se at en konstant approksimationsalgoritme til TSP ikke er muligt uden trekantsuligheden undtagen hvis $P = NP$.
  \end{itemize}
\end{frame}


%%% Local Variables:
%%% mode: latex
%%% TeX-engine: xetex
%%% TeX-command-extra-options: "-shell-escape"
%%% TeX-master: "main"
%%% End:
