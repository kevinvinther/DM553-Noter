\documentclass{beamer}

\title{DM553 Emner}
\subtitle{Til mundtlig præsentation af hele pensum.}

\author{Kevin Vinther}
\institute{IMADA}
\date{6. Maj - 14 Juni (?)}


% Theme setting
\usetheme{default}
\usecolortheme{dove}

\usepackage{blindtext}
\usepackage{hyperref}
\usepackage[T1]{fontenc} %thanks's daleif
\usepackage[utf8]{inputenc}
\usepackage{amsthm}

% Sæt sproget til dansk
\usepackage{polyglossia}
\setdefaultlanguage{danish}
\setotherlanguages{english}



% Fonts and layout
\usefonttheme{professionalfonts}
\setbeamerfont{frametitle}{series=\bfseries}
\setbeamertemplate{navigation symbols}{}
\setbeamertemplate{footline}[frame number]

\usepackage{tikz} % Til grafer
\usetikzlibrary{automata} % Til automatgrafer
\usetikzlibrary{positioning} % Positionering af knuder
\usetikzlibrary{arrows.meta, arrows} % arrows til at ændre kanter
\tikzset{
	node distance=2.5cm, % minimumsdistancen mellem to knuderj
	every state/.style={
			semithick,
			fill=gray!10
		},
	initial text={}, % Ingen tekst, bare kanter
	double distance=2pt, % Udseende på accept states
	every edge/.style={ % Egenskaber for hver transition
			draw,
			>=stealth', % bold arrowheads
			auto,
			semithick
		}
}
% Tak til: https://www3.nd.edu/~dchiang/teaching/theory/2018/www/tikz_tutorial.pdf

\begin{document}

%%% Local variables:
%%% Mode: latex
%%% Tex-engine: xetex
%%% Tex-command-extra-options: "-shell-escape"
%%% Tex-master: "main"
%%% End:



\section*{Pumping Lemmas For Regular And Context-Free Languages}

\begin{itemize}
	\item Regulære Sprog
	\item Nonregulæritet
	\item $\{0^{n}1^{n} \mid n \ge 0\}$
	      \begin{itemize}
		      \item Pumpelemmaet for regulære sprog
		            \begin{itemize}
			            \item $xy^{i}z$
			            \item $y \ne \varepsilon$
			            \item $|xy| \le p$
		            \end{itemize}
		      \item Bevis
		            \begin{itemize}
			            \item $p = $antal tilstande
			            \item Hvis $y = \varepsilon$ så er sætningen trivielt sand, da $\varepsilon^{i} = \varepsilon \forall i$
			            \item En tilstand må gentages
			                  \begin{itemize}
				                  \item Fordi vi tager den første gentagne tilstand kan vi sige at $|xy| \le p$
			                  \end{itemize}
		            \end{itemize}
	      \end{itemize}
	\item Kontekstfrie Sprog
	\item Nonkontekstfrie Sprog
	\item $\{a^{n}b^{n}c^{n} \mid n \ge 0\}$
	      \begin{itemize}
		      \item Pumpelemmaet for kontekstfrie sprog
		            \begin{itemize}
			            \item $uv^{i}xy^{i}z$
			            \item $y \ne \varepsilon$
			            \item $|vxy| \le p$
		            \end{itemize}
		      \item Bevis
		            \begin{itemize}
			            \item $p = 2^{|V|+1}$
			            \item For en streng $|w| \ge p$ vil parsetræet have $\ge 2^{|V|+1}$ blade.
			            \item Højden af træet er mindst $|V|+2$ fordi en ekstra til at gå fra variabel til terminal.
			            \item Én af dise $|V|+2$ variabler må være gentagen.
			            \item Betingelser:
			                  \begin{itemize}
				                  \item 1. gælder da den gentagne variabel kan gentages igen og igen, og man kan også bare gå videre til at udlede $x$ fremfor $v$ og $y$.
				                  \item 2. gælder da sætningen er trivielt sand ellers
				                  \item $|vxy| \le p$ gælder, da vi ellers skulle have mere end $2^{|V|+1}$ blade.
			                  \end{itemize}
		            \end{itemize}
	      \end{itemize}
\end{itemize}

\section*{Pushdown Automata And Context-Free Languages}

\begin{itemize}
	\item Introduktion
	      \begin{itemize}
		      \item Forklar hvad kontekstfrie sprog er, CFG og PDA
	      \end{itemize}
	\item Kontekstfrie Sprog
	      \begin{itemize}
		      \item CFG - 4 Tuple ($V, \Sigma, R, S$)
		      \item Chomsky Normal Form, $2|w|-1$
		      \item Pushdown Automat - som en NFA (skal være nondeterministisk!) men med en stak, man kan nemt se hvordan $a^{n}b^{n}$ kan afgøres her.
	      \end{itemize}
	\item Ækvivalens mellem PDA og CFG
	\item Fra CFG til PDA:
	      \begin{itemize}
		      \item Start med pushing af start-symbolet
		      \item Venstremest afledning
		      \item Accep tinput ved at tjekke terminalerne
	      \end{itemize}
	\item Fra PDA til CFG:
	      \begin{itemize}
		      \item Skridt 1:
		      \item Forklar først om den begrænset PDA
		      \item 1. Én accept tilstand
		      \item 2. Der skal enten pushes eller poppes
		      \item 3. Stakken tømmes altid før accept
		      \item Vis at man sagtens kan gøre dette

		      \item Skridt 2:
		      \item Hvis den går fra $\mathdollar$ i stakken til $\mathdollar$ i stakken på et læst $w$, vil den gøre det samme med niveau $\ell$.
		      \item Skridt 3:
		      \item \(\forall p, q \in Q\) vil der være en variabel $A_{pq}$ som vil generere alle strenge $w \in \Sigma^{*}$ hvor stakken ikke går ned
		      \item Der er to scenarier: enten pusher den et symbol der først poppes til sidst, in that case: $aA_{rs}b$
		      \item Ellers kommer den ned på et andet tidpsunkt, in that case: $A_{pr}A_{sq}$.
		      \item Grammatikken er så som følger: $V = \{A_{pq} \mid p, q \in Q\}$
		      \item $S = A_{q_{0}q_{accept}}$
		      \item Regler: $\forall p, q, r, s \in Q, \forall t \in \Gamma, \forall a, b \in \Sigma_{\varepsilon}$ hvor den læser et a, pusher et t laver alt muligt, så læser et b og popper det t laver vi reglen $A_{pq} \rightarrow aA_{rs}b$ til $R$
		      \item $\forall p,q,r \in Q$ lad $A_{pq} \rightarrow A_{pr}A_{rq}$
		      \item $\forall p \in Q $ lad $A_{pp} \rightarrow \varepsilon$.
	      \end{itemize}
\end{itemize}

\section*{Turing Maskiner}
\begin{itemize}
	\item Kort introduktion til hvad en Turingmaskine er: som en DFA med ubegrænset bånd
	\item En 7-tuple: $(\Sigma, \Gamma, Q, q_{0}, q_{accept}, q_{reject}, \delta)$
	\item Grundlæggende om hvordan mskinen læser, skriver og bevæger åndet
	\item \textit{Konfigurationer}
	      \begin{itemize}
		      \item Hvad er en konfiguration
		      \item Hvad vil det sige at en konfiguration giver en andnen? $C_{1} \rightarrow C_2$
		            \begin{itemize}
			            \item At den kan gå fra $C_{1}$ til $C_{2}$ på ét skridt
		            \end{itemize}
		      \item Specielle konfigurationer: start, accept, afvisning
	      \end{itemize}
	\item \textit{Nondeterministiske Turingmaskiner}
	      \begin{itemize}
		      \item Forskelle
		            \begin{itemize}
			            \item Den kan gætte sig frem, og kan derfor have $B = |Q| \cdot |\Gamma| \cdot 3$ overføringer, fremfor kun én.
			            \item God til for eksempel at finde ud af primtal, da den kan gætte på tal der skal kunne ganges i.
		            \end{itemize}
		      \item Konvertering til deterministisk
		            \begin{itemize}
			            \item Lad hver overføring være $B$ eller $0$, og repræsenter valget som en streng i base $B$.
			            \item $283$ vil sige at det første skridt har taget anden overføring, andet skridt har taget ottende overføring og tredje skridt har taget tredje overføring.
			            \item Den deterministiske Turingmaskine har så 3 bånd: et bånd med input, et bånd der ændrer i forhold til hvad NDTM'en ville gøre, og et bånd til hvor vi er med strengen.
			            \item Vi bruger breadth first search, i en shortlex metode.
			            \item $B + B^2 + B^{3} + B^{4} \cdots B^{r}$: \textit{Eksponentielt}
		            \end{itemize}
	      \end{itemize}
	\item \textit{Multibånds Turingmaskiner}
\end{itemize}

\section*{Afgørlighed}

\section*{NP-Komplethedsbeviser - Eksempler}

\section*{Cook-Levin}

\section*{Approksimationsalgoritme}

\section*{Informationsteoretiske Nedre Grænser}

\section*{Modstanderargumenter - Teknikker og Eksempler}

\section*{Algoritmer med faste parametre, parameteriseret kompleksitet og ekaskte eksponentielle algoritmer}


\newpage

\end{document}
%%% Local Variables:
%%% mode: latex
%%% TeX-engine: xetex
%%% TeX-command-extra-options: "-shell-escape"
%%% End:
